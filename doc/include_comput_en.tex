\part{Computational tools}

\chapter{Choice of computational tools}

The least-squares method was utilized for the nonlinear fitting of the models
to the experimental data sets. This was achieved with a program in the \texttt{C}
programming language, utilizing the functions and routines of the GNU Scientific
Library (GSL), a library for numerical computations under the GNU General Public
License \cite{galassi_book}. The choice of this library for nonlinear optimizations
was due to the ease of access of the intermediate steps in the iterative calculations
of the desired parameters.

\chapter{Property conversion}

Unfortunately, not all data sets utilized were available as series of $a_w$ values
as a function of composition; as seen in the tables \ref{tab_dados_pontos} and
\ref{tab_dados_multi_pontos}, one must, initially, convert all data to the same
basis, before supplying the data to the program responsible for fitting. The basis
of choice was simply water activity as a function of molar fraction, with all
conversion of the activity data to osmotic coefficient data being internal to the
fitting program.

\chapter{Algorithms and cost functions}

The Levenberg-Marquardt algorithm with geodesic acceleration was used to solve
the least-squares problem as shown in the equation \ref{eq:min_quad_lev},
fitting a model of $N$ parameters to set of $n>N$ experimental data points.
The algorithm was chosen due to its fast convergence to a solution, for reasonably
small values of $N$, as they are in the examples analysed.
One must observe that, for the exhibition of the fitting parameters and comparison
of the deviations, the value actually being shown is the square root (so the value
is dimensionally correct) of the ratio between cost function and number of
data points (so the comparisons between different data series are fair); as $n>0 %
\implies \frac{\partial}{\partial f}\sqrt{\frac{f}{n}} >0$, this scaling is not a
problem, since the fitting evaluation parameters are monotonically dependent on
the cost function.

\begin{equation}
	\label{eq:min_quad_lev}
	\minimize_{(A_j^*)_{j \in \{1,2,\ldots,N\}} \in \mathbb{R}^N}%
	\left(\sum_{k \in \{1,2,\ldots,K\}}\left[\phi^k - \Phi((A^*_j)_{j%
	\in \{1, 2, \ldots, N\}}, (x^k_i)_\text{$i$ componente},%
	T^k, P^k)\right]^2\right)
\end{equation}

Clearly, the function $\Phi$ depends on the model to be fitted, ranging from simple
expressions as the one derived from Norrish's equation, to relatively complex
ones, as the one derived from the UNIQUAC model. Besides, the initial values for
the iterative process were obtained through values registered in the literature for
similar substances.

For a few data sets, the property supplied was the composition of a binary mixture
in osmotic equilibrium with the solution of interest. In this situation, a high
degree polynomial was fitted through a reference data series, obtaining values of
water activity for analysis.

The reference substances were sodium chloride (to which the reference data were
obtained from Scatchard\footnote{\cite{scatchard1938}}) and sucrose (to which
the reference data were obtained from Stokes \& Robinson\footnote{\cite{%
stokes1961}}).

Apart from the cost function above, one may also be interested in other fitting
evaluation parameters, such as the determination coefficient $R^2$, adjusted
to account for the amount of parameters $N$ and the amount of data points $n$,
obtaining the adjusted coefficient of determination $\bar{R^2}$.

\begin{equation}
	\bar{R^2} = 1 - \cfrac{\left( 1 - R^2 \right) \left( n - 1 \right)}
		{n - N - 1}
\end{equation}

One of the main problems in the optimization of the models, beyond the previously
listed ones, lies on the big values of $\sigma_\phi$, compared to the values of
$\sigma_{x_w}$ and $\sigma_{a_w}$. Even if one assumes the activity measurements as
greatest source of uncertainty (\textit{e.g} $\sigma_{a_w} \gg \sigma_{x_w}$), the
uncertainty in the osmotic coefficient calculations goes to infinity as $x_w$
approaches 1.

\chapter{Code and data}

All the code developed and all the data utilized are available at the project
\href{https://github.com/pedro-callil/IC_2021_Modelagem_Termodinamica}{repository}
\footnote{Available at
	\url{https://github.com/pedro-callil/IC_2021_Modelagem_Termodinamica}.}
in which are available more details about the implementation of each model and
compilation, installation and execution of each binary.

If one is interested in verifying the source code and the executables, one needs
only to clone the repository and compile the program. To do this, are required
git, GCC, the GSL and the numerical libraries BLAS/CBLAS.

\begin{verbatim}
$ git clone https://github.com/pedro-callil/IC_2021_Modelagem_Termodinamica
$ cd IC_2021_Modelagem_Termodinamica
$ make # Desnecessário caso o interesse seja pelos dados
\end{verbatim}

If the interest is restricted to the data utilized, they can also be obtained,
simply cloning the repository:

\begin{verbatim}
$ git clone https://github.com/pedro-callil/IC_2021_Modelagem_Termodinamica
$ cd IC_2021_Modelagem_Termodinamica/data
\end{verbatim}

\section{License}

As the use of the GNU Scientific Library requires, the source code is available
under the GNU General Public License, 3\textsuperscript{rd} version, as the
library itself is.

