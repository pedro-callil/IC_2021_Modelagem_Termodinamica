\part{Ferramentas computacionais}

\chapter{Escolha das ferramentas}

Foi utilizado o método dos mínimos quadrados para o ajuste não-linear dos
modelos aos dados obtidos. Para isso, foi escrito programa em linguagem
\texttt{C}, de modo a utilizar as rotinas fornecidas pela biblioteca
\textit{GNU Scientific Library} \cite{galassi_book} para otimização não-linear.
A escolha dessa biblioteca em específico se deu devido ao fácil acesso aos
passos intermediários dos processos iterativos utilizados para obter os
parâmetros desejados.

\chapter{Conversão de propriedades}

Infelizmente, nem todos os dados utilizados estão disponíveis nas mesmas
propriedades; como visto nas tabelas \ref{tab_dados_pontos} e
\ref{tab_dados_multi_pontos}, é necessário inicialmente converter todos os dados
para a mesma base. Mesmo que o programa necessite dos dados de coeficiente
osmótico para a regressão, todos os dados são alimentados ao programa na forma
de dados de atividade da água e frações molares, sendo a conversão dos valores de
$a_w$ em valores de $\phi$ puramente interna.

\chapter{Algoritmos e funções-objetivo}

Foi utilizado o algoritmo de Levenberg-Marquardt com aceleração geodésica para a
solução do problema de mínimos quadrados para um conjunto de $n$ dados e $N$
parâmetros dado pela expressão \ref{eq:min_quad_lev}.
Esse algoritmo foi escolhido pois converge de forma rápida para sistemas com
uma quantidade de parâmetros razoável, como é o caso aqui (número de parâmetros
$N$, para minimizar a função objetivo, menor que uma dezena).
Perceba que para a apresentação dos dados e avaliação do ajuste, foi calculada a
raiz (de modo a tornar o valor dimensionalmente correto) da razão entre o valor da
função e o número de dados do ajuste (para possibilitar a comparação entre
diferentes séries de dados); como
$n>0 \implies \frac{\partial}{\partial f}\sqrt{\frac{f}{n}} >0$, temos que os
parâmetros de ajuste dependem monotonicamente da função objetivo, sendo, portanto,
válidos.

\begin{equation}
	\label{eq:min_quad_lev}
	\minimize_{(A_j^*)_{j \in \{1,2,\ldots,N\}} \in \mathbb{R}^N}%
	\left(\sum_{k \in \{1,2,\ldots,K\}}\left[\phi^k - \Phi((A^*_j)_{j%
	\in \{1, 2, \ldots, N\}}, (x^k_i)_\text{$i$ componente},%
	T^k, P^k)\right]^2\right)
\end{equation}

Claramente, a forma da função $\Phi$ depende do modelo a ser utilizado, variando
entre expressões simples como a obtida a partir da equação de Norrish, a
expressões significantemente mais complexas, como a obtida a partir de modelos
como UNIQUAC. Além disso, os valores iniciais para o processo iterativo serão
obtidos a partir de consulta à literatura para obtenção de dados já obtidos para
substâncias semelhantes.

Para alguns conjuntos de dados, foram informados dados de composição de duas
soluções, com a mesmo valor de $a_w$. Nesse caso, foi ajustado um polinômio
de grau alto a uma série de dados de referência, e foram, dessa forma, obtidos
valores de atividade para comparação.

As substâncias de referência foram NaCl (cuja curva de ajuste foi obtida a
partir de dados de Scatchard\footnote{\cite{scatchard1938}}) e sacarose (cuja
curva de ajuste foi obtida a partir de dados de Stokes \& Robinson\footnote{%
\cite{stokes1961}}).

Além da função objetivo explicitada acima, também é interessante avaliar o
comportamento e utilidade do coeficiente de determinação $R^2$, ajustado para
considerarmos a quantidade de parâmetros $N$ e a quantidade de dados $n$, como
parâmetro de ajuste.

\begin{equation}
	\bar{R^2} = 1 - \cfrac{\left( 1 - R^2 \right) \left( n - 1 \right)}
		{n - N - 1}
\end{equation}

O principal problema para a otimização dos modelos é, além dos problemas
intrínsecos aos modelos, como já discutido, a enorme imprecisão em
$\sigma_\phi$ obtida ao se calcular o coeficiente osmótico, frente ao erro
$\sigma_{a_w}$ e $\sigma_{x_w}$ das medições. Mesmo se assumindo que a maior
fonte de erro seja a medição das atividades (\textit{e.g.} $\sigma_{a_w} \gg
\sigma_{x_w}$), temos que a incerteza nos cálculos de coeficiente osmótico
tende ao infinito quando o valor de $x_w$ tende a 1.

\chapter{Código e Dados}

Todo o código desenvolvido e os dados utilizados estão disponíveis no
\href{https://github.com/pedro-callil/IC_2021_Modelagem_Termodinamica}{repositório}
\footnote{Disponível em
	\url{https://github.com/pedro-callil/IC_2021_Modelagem_Termodinamica}.}
do projeto, no qual estão disponíveis mais detalhes sobre a implementação de
cada modelo e compilação/execução do programa.

Caso haja interesse em verificar o código-fonte e os executáveis basta clonar o
repositório e compilar o programa. São necessários, para isso, git, GCC, GSL e as
bibliotecas numéricas BLAS e CBLAS.

\begin{verbatim}
$ git clone https://github.com/pedro-callil/IC_2021_Modelagem_Termodinamica
$ cd IC_2021_Modelagem_Termodinamica
$ make # Desnecessário caso o interesse seja pelos dados
\end{verbatim}

Caso não haja interesse em obter os executáveis, os dados podem ser obtidos
clonando o repositório, simplesmente:

\begin{verbatim}
$ git clone https://github.com/pedro-callil/IC_2021_Modelagem_Termodinamica
$ cd IC_2021_Modelagem_Termodinamica/data
\end{verbatim}

\section{Licença}

Como exigido pelo uso da biblioteca científica da GNU (disponível sob a Licença
Pública Geral, 3\textordfeminine\ versão), o código-fonte também está disponível
sob os termos dessa licença.

