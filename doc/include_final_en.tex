\part{Conclusions}

Despite having the greatest correlative power of the models analysed, the UNIQUAC
model is not always useful, as even with a series of simplifications, it requires
a large amount of adjustable parameters, and, for a successful regression, one may
lack the quantity required of experimental data. Besides, there are difficulties in
the convergence of the algorithms for the least-squares nonlinear fitting when
analysing some data sets.

Therefore, for relatively simple mixtures, models as Norrish's and the virial model
find great practical usefulness; moreover, the mathematical modelling employed is
less complex, which facilitates numerical analysis. However, one must pay attention
to pitfalls arising from applicability problems: while the virial model is more
flexible, Norrish's model fails to be useful when the solution being examined
exhibits osmotic coefficient greater than one.

Finally, one must keep in mind that, despite the superiority of these models (and,
usually, due to the above discussed reasons, of the virial model), the ideal
solution hypothesis (Raoult's law) and its corrections (Caurie's and Zdanovskii's)
predict the water activity reasonably well; when the accuracy of $a_w$ values is
not critical, in mixtures in which the effects of complex solute-solute interactions
are negligible and in which the dilutions are relatively high, their adoption is
valid. Besides, one could, perhaps, use a simplified virial model as a replacement
to the one with solute-solute interaction parameters, in a situation in which the
accuracy of the full virial (or Norrish's, \textit{etc.}) model is not a hard
requirement, but Raoult's Law and its non-parametric corrections fail.

\phantompart

\part{Schedule}

The original schedule (table \ref{tab:cronograma}) proposed for the project
was followed. Previous research on the subject was already compiled and
interpreted, the computational routines were already programmed and the selected
models were fitted to the data.

\begin{table}[h]
	\centering
	\caption{Project schedule}
	\label{tab:cronograma}
	\begin{tabular}{l C C C C}\hline
		\multirow{2}*{Task} & \multicolumn{4}{c}{Trimester}\\
			& 1 & 2 & 3 & 4\\\hline
		Bibliographic revision& \multicolumn{4}{|c|}%
		{\cellcolor{pdone}Done}\\\hhline{~----}
		Previous research analysis &
		\multicolumn{2}{|c|}{\cellcolor{pdone}Done} & & \\\hhline{~---~}
		Computational routines programming & &
		\multicolumn{2}{|c|}{\cellcolor{pdone}Done} & \\\hhline{~~---}
		Model fitting & & &
			\multicolumn{2}{|c|}{\cellcolor{pdone}Done}\\
		\hhline{~~---}
		Report writing & &
			\multicolumn{1}{|c|}{\cellcolor{pdone}Done}
		& & \multicolumn{1}{|c|}{\cellcolor{pdone}Done}\\\hline
	\end{tabular}
\end{table}

