\part{Conclusões}

Apesar de o modelo UNIQUAC apresentar maior poder correlativo, comparado com os
outros modelos apresentados, não é sempre possível fazer uma regressão entre os
dados experimentais de composição e atividade, já que o modelo exige uma grande
quantidade de parâmetros, mesmo adotando uma série de simplificações; de fato,
para obtermos os valores dos coeficientes $q_i$ e $u_{ii}$ precisamos de uma grande
quantidade de pontos $a_w=a_w(x_1,x_2,\ldots,x_n)$, que muitas vezes não está
disponível. Além disso, existem maiores dificuldades para a convergência do método
dos mínimos quadrados nesse modelo, comparando aos outros analisados.

Dessa forma, modelos mais simples como o de Norrish e o modelo virial estudado, que,
para misturas que não sejam excessivamente complexas, apresentam poucos parâmetros
a serem ajustados, encontram grande utilidade prática; além disso, o equacionamento
mais simples desses modelos facilita a regressão através de métodos numéricos. Outra
observação a ser feita reside na aplicabilidade desses modelos: enquanto o modelo
de Norrish é relativamente mais restrito, o modelo virial (assim como UNIQUAC)
consegue se adaptar a uma grande gama de valores possíveis de coeficiente osmótico
$\phi$.

Por fim, é importante ressaltar que, apesar de esses modelos (Norrish, Virial e
UNIQUAC) serem superiores ao caso ideal (Lei de Raoult) e às correções (Zdanovskii
e Caurie) propostas, a Lei de Raoult (e, quando aplicáveis, os outros dois modelos)
apresenta grande concordância com os dados experimentais de $a_w$ em uma fração
considerável das séries de dados analisadas: assim, para misturas não muito
complexas, em altas diluições, em situações nas quais a precisão no valor
de $a_w$ não é crítica, é válido, portanto, adotá-la. Além disso, em situações
nas quais a Lei de Raoult e suas correções são insuficientes, mas não há
importância crítica em obter a precisão oferecida pelo modelo virial, a utilização
de uma simplificação desse modelo, sem parâmetros de interação soluto-soluto, é
razoável.

\phantompart

\part{Cronograma}

O cronograma original (tabela \ref{tab:cronograma}) proposto para o projeto
está sendo obedecido: já foi feita a compilação e interpretação dos dados da
literatura, e estão sendo escritos os códigos-fonte dos programas responsáveis
por ajustar os modelos aos dados.

\begin{table}[h]
	\centering
	\caption{Cronograma e etapas já cumpridas}%IC_GOES_HERE
	\label{tab:cronograma}
	\begin{tabular}{l C C C C}\hline
		\multirow{2}*{Etapa} & \multicolumn{4}{c}{Trimestre}\\
			& 1 & 2 & 3 & 4\\\hline
		Revisão bibliográfica & \multicolumn{4}{|c|}%
		{\cellcolor{pdone}Concluída}\\\hhline{~----}
		Análise e interpretação da literatura &
		\multicolumn{2}{|c|}{\cellcolor{pdone}Concluída} & & \\\hhline{~---~}
		Programação das rotinas computacionais & &
		\multicolumn{2}{|c|}{\cellcolor{pdone}Concluída} & \\\hhline{~~---}
		Ajuste do modelo & & &
			\multicolumn{2}{|c|}{\cellcolor{pdone}Concluída}\\
		\hhline{~~---}
		Redação de relatório & &
			\multicolumn{1}{|c|}{\cellcolor{pdone}Concluída}
		& & \multicolumn{1}{|c|}{\cellcolor{pdone}Concluída}\\\hline
	\end{tabular}
\end{table}

%Nesse módulo de estágio, englobado no terceiro e quarto trimestres
do cronograma exibido na tabela \ref{tab:cronograma}, foram realizadas as
seguintes atividades:

\begin{table}[h]
	\centering
	\caption{Cronograma e etapas já cumpridas - Módulo de Estágio}
	\label{tab:cronograma_me}
	\begin{tabular}{l C C C C}\hline
		\multirow{2}*{Etapa} & \multicolumn{4}{c}{Mês}\\
			& Maio & Junho & Julho & Agosto\\\hline
		Revisão bibliográfica & \multicolumn{4}{|c|}%
		{\cellcolor{pdone}Concluída}\\\hhline{~----}
		Programação: conversões de dados & \multicolumn{1}{|c|}%
		{\cellcolor{pdone}Concluída}\\\hhline{~-~~~}
		Programação: regressões & \multicolumn{1}{|c|}%
		{\cellcolor{pdone}Concluída}\\\hhline{~---~}
		Programação: análise das regressões & & \multicolumn{2}{|c|}%
		{\cellcolor{pdone}Concluída}\\\hhline{~~--~}
		Ajuste do modelo & &
			\multicolumn{2}{|c|}{\cellcolor{pdone}Concluída} & \\
		\hhline{~----}
		Redação de relatórios &
		\multicolumn{4}{|c|}{\cellcolor{pdoing}Em conclusão}\\\hline
	\end{tabular}
\end{table}



