\part{Introduction}

The activity $a$ of a substance in a system, defined \cite{sandler2017} as
the ratio between the fugacity of the substance in the system and its fugacity
in a reference system (equation \ref{eq:def_atv}), acts as its ``effective
concentration'', thermodynamically. Therefore, the measurement and analysis of
the water activity in solutions allows better predictions of phenomena related to
the water contents of a system. To the food industry, the most important of those
phenomena is the microbial-induced food spoilage.

Indeed, the property controlling microbial growth in a food product is the
water activity. In the graph exhibited at the figure
\ref{fig:germ}\footnote{\cite{canovas2007}}, one can see the values of water
activity required for the growth of a few microorganisms of interest.

\begin{figure}[h]
	\centering
	\begin{tikzpicture}[>=Stealth]
		\draw[->] (5,-0.5) -- (5,11);
		\draw (5,0.0) node[anchor=west] {$a_w$};
		\draw (5,8.75) -- (6.3,8.75);
		\draw (6.3,8.75) -- (13.5,8.75);
		\draw (5,7.75) -- (6.3,7.25);
		\draw (6.3,7.25) -- (13.5,7.25);
		\draw (5,6.75) -- (6.3,5.8);
		\draw (6.3,5.8) -- (13.5,5.8);
		\draw (5,5) -- (6.3,4.2);
		\draw (6.3,4.2) -- (13.5,4.2);
		\draw (5,3.75) -- (6.3,3);
		\draw (6.3,3) -- (13.5,3);
		\draw (5,1.25) -- (6.3,1.85);
		\draw (6.3,1.85) -- (13.5,1.85);
		\draw (5,0.25) -- (6.3,0.35);
		\draw (6.3,0.35) -- (13.5,0.35);
		\fill[black] (5,0.25) circle (0.07);
		\fill[black] (5,1.25) circle (0.07);
		\fill[black] (5,3.75) circle (0.07);
		\fill[black] (5,5) circle (0.07);
		\fill[black] (5,6.75) circle (0.07);
		\fill[black] (5,7.75) circle (0.07);
		\fill[black] (5,8.75) circle (0.07);
		\fill[black] (5,10) circle (0.07);
		\draw (10,10.4) node[anchor=south, text width=9cm]
		{\tiny{Microorganisms inhibited at the smallest value of
		$a_w$ in the range}};
		\draw (6.3,9.5) node[anchor=west, text width=7cm]
		{\tiny{\textit{Pseudomonas}, \textit{Escherichia},
		\textit{Proteus}, \textit{Shigella}, \textit{Klebsiella},
		\textit{Bacillus}, \textit{Clostridium perfringens},
		\textit{C. botulinum} E, G, a few yeasts}};
		\draw (6.3,8) node[anchor=west, text width=7cm]
		{\tiny{\textit{Salmonella}, \textit{Vibrio parahaemolyticus},
		\textit{Clostridium botulinum} A, B, \textit{Listeria
		monocytogenes}, \textit{Bacillus cereus}}};
		\draw (6.3,6.5) node[anchor=west, text width=7cm]
		{\tiny{\textit{Staphylococcus aureus} (aerobic), several
		yeasts (as \textit{Candida, Torulopsis, Hansenula}),
		\textit{Micrococcus}}};
		\draw (6.3,5) node[anchor=west, text width=7cm]
		{\tiny{Most molds (mycotoxigenic \textit{Penicillium}),
		\textit{Staphylococcus aureus},
		most \textit{Saccharomyces}, \textit{Debaryomyces}}};
		\draw (6.3,3.5) node[anchor=west, text width=7cm]
		{\tiny{Most halophile bacteria,
		mycotoxigenic \textit{Aspergillus}}};
		\draw (6.3,2.4) node[anchor=west, text width=7cm]
		{\tiny{Xerophile molds (\textit{Aspergillus chevalieri, A.
		candidus, Wallemia sebi}), \textit{Saccharomyces bisporus}}};
		\draw (6.3,1.1) node[anchor=west, text width=7cm]
		{\tiny{Osmophile yeasts (\textit{Sacharomyces rouxii},
		a few molds (\textit{Aspergillus echinulatus, Monascus
		bisporus})}};
		\draw (6.3,0.1) node[anchor=west, text width=7cm]
		{\tiny{No microbial proliferation}};
		\draw (5,0.25) node[anchor=east] {0.61};
		\draw (5,1.25) node[anchor=east] {0.65};
		\draw (5,3.75) node[anchor=east] {0.75};
		\draw (5,5) node[anchor=east] {0.80};
		\draw (5,6.75) node[anchor=east] {0.87};
		\draw (5,7.75) node[anchor=east] {0.91};
		\draw (5,8.75) node[anchor=east] {0.95};
		\draw (5,10) node[anchor=east] {1.00};
	\end{tikzpicture}
	\caption{$a_w$ values for microbial growth}
	\label{fig:germ}
\end{figure}

In the conditions usually found in the food industry, is reasonable to assume
that the gas phase of the system behaves as an ideal gas\cite{canovas2007},
in most situations. Furthermore, the reference condition is taken as pure and liquid
water at the same pressure and temperature as the system. Therefore, one can obtain:

\begin{equation}
	\label{eq:def_atv}
	a_w \equiv \left(\cfrac{f_w}{f^\text{ref}_w}\right)_T%
		\approx \left(\cfrac{p^\text{vap}_w}{p^\text{vap,ref}_w}\right)_T
\end{equation}

For ideal solutions, the evaluation of water activity is straightforward: for all
component $u$ in a mixture, its activity $a_i$ is simply its molar fraction $x_i$;
hence, $a_w = x_w$. However, not all systems are ideal, and therefore one needs to
adopt another property, the activity coefficient of the water, $\gamma_w$:

\begin{equation}
	a_w = \gamma_wx_w \implies \gamma_w = \cfrac{a_w}{x_w}
\end{equation}

Yet, even if one now has an indicator of the deviation of a solution from
ideality , another problem arises: the activity coefficient is not a good
parameter to evaluate the deviations when applied to very diluted solutions;
after all, $a_w=a_w(x_w)$ is a continuous function, and as the solution is diluted,
$a_w$ must go to 1 (because the limit $x_w=1$ is the reference solution), and
therefore one will observe that:

\begin{equation}
	\lim_{x_w \to 1}\gamma_w = 1
\end{equation}

And for many solutions of interest, one can observe not only high deviations from
ideal behavior, but also extremely diluted solutions. Thus, one adopts another
property, the osmotic coefficient $\phi$, defined as:

\begin{equation}
	\phi = \cfrac{\ln(a_w)}{\ln(x_w)}
\end{equation}

One can compare these two properties and visualize a possible behavior of
the water activity as the composition of a mixture changes by observing
the data plotted in the figure \ref{fig_atv_gamma_gluc} \cite{ebrahimi2016}.

\begin{figure}[h]
	\centering
	\begin{tikzpicture}
		\newsavebox{\minigrafglic}
		\savebox{\minigrafglic}{%
		\scalebox{0.5}{%
		\begin{tikzpicture}
			\begin{axis}[
			xmin=0.97,xmax=1.0,
			ymin=0.97,ymax=1.0,
			x dir=reverse,
			width=7cm,
			height=7cm,
			xlabel = {$x_w$},
			ylabel = {$a_w$},
			]
			\addplot+[
				color=pverydarkred,
				mark=o,
				only marks,
			]
			table[x={xw},y={aw}]{glucose_a_w_and_phi.dat};
			\addlegendentry{$a_w$ (experimental)};
			\addplot+[
				color=black,
				no marks,
				domain=0.96:1.0,
				samples=5,
			]{x};
			\addlegendentry{$a_w=x_w$};
		\end{axis}
	\end{tikzpicture}
	}}
	\begin{axis}[
			xmin = 0.97, xmax = 1.0, xlabel = {$x_w$},
			ymin = 0.96, ymax = 1.05, ylabel = {$\phi$},
			x tick label style={
				/pgf/number format/.cd,
				fixed,
				fixed zerofill,
				precision=3,
				/tikz/.cd
			},
			y tick label style={
				/pgf/number format/.cd,
				fixed,
				fixed zerofill,
				precision=3,
				/tikz/.cd
			},
			legend pos = south west,
			axis y line=left,
			x axis line style={-},
			xlabel near ticks,
			ylabel near ticks,
			x dir=reverse,
			ytick={0.97,0.985,1.00,1.015,1.03,1.045},
		]
		\addplot+[
			color=pdarkblue,
			mark=square,
			very thick,
			only marks,
		]
		table[x={xw},y={phi}]{glucose_a_w_and_phi.dat};
		\addlegendentry{$\phi$};
		\draw (axis cs: 0.981,0.99)node{\usebox{\minigrafglic}};
		\end{axis}
		\begin{axis}[
			xmin = 0.97, xmax = 1.0, xlabel = {$x_w$},
			ymin = 0.9977, ymax = 1.0001, ylabel = {$\gamma_w$},
			y tick label style={
				/pgf/number format/.cd,
				fixed,
				fixed zerofill,
				precision=4,
				/tikz/.cd
			},
			axis y line=right,
			axis x line=none,
			legend pos = south east,
			x dir=reverse,
		]
		\addplot+[
			color=pverybrightred,
			mark=o,
			very thick,
			only marks,
		]
		table[x={xw},y={gammaw}]{glucose_a_w_and_phi.dat};
		\addlegendentry{$\gamma_w$};
		\end{axis}
	\end{tikzpicture}
	\caption{Activity ($a_w$), Activity coefficient ($\gamma_w$) and%
	osmotic coefficient ($\phi$) of water in D-glucose solutions. %
	One should observe the difference in the $\gamma_w$ and $\phi$ axis scales.}
	\label{fig_atv_gamma_gluc}
\end{figure}

In addition, one must remember that water activity measurements are sometimes
obtained indirectly; instead of measurements of vapor pressure in the system
and comparison with the same property in the reference system (as shown in the
equation \ref{eq:def_atv}), one can measure properties as freezing point
depression, boiling point elevation, or the composition of another solution
in osmotic equilibrium with the system under analysis.

For instance, a few data sets obtained for $n$-ary solutions of carbohydrates
\cite{abderafi1994} are, in fact, boiling point elevation data. Hence, the
experimental measurements must first be converted to osmotic coefficient data,
through the equations \ref{eq:eleb_to_osco} \cite{ge2009,ge2009err}, which are
plotted in the figure \ref{fig_ge_and_wang}.

\begin{equation}
	\label{eq:eleb_to_osco}
	\begin{cases}
		\ln(a_w) = -\cfrac{\Delta H^\text{vap}_{0,T_B}\left(\cfrac{1}{T_B}%
			-\cfrac{1}{T_B+\Delta T_B}\right)-%
			\Delta C_p^\text{vap}\Bigg[\ln\left(\cfrac{T_B+%
			\Delta T_B}{T_B}\right) - \cfrac{\Delta T_B}{T_B+%
			\Delta T_B}\Bigg]}{R}\\
		\ln(a_w) = \cfrac{\Delta H^\text{fus}_{0,T_F}\left(\cfrac{1}{T_F}-%
		\cfrac{1}{T_F-\Delta T_F}\right)+\Delta C_p^\text{fus}%
		\Bigg[\ln\left(\cfrac{T_F-\Delta T_F}{T_F}\right) -%
		\cfrac{\Delta T_F}{T_F-\Delta T_F}\Bigg]}{R}
	\end{cases}
\end{equation}

\begin{figure}[h]
	\centering
	\begin{subfigure}{0.5\textwidth}
		\begin{tikzpicture}[scale=0.75]
			\begin{axis}[
				ylabel={$a_w$},
				xlabel={$T_B$[\textcelsius]},
				ymax=1, xmax=115, xmin=100,
			]
				\addplot [
					color=pred,
					no marks,
					very thick,
				]
					table [x={BPE},y={aw},col sep=comma]
						{ge_and_wang_bpe.csv};
			\end{axis}
		\end{tikzpicture}
		\caption{Boiling Point}
	\end{subfigure}%
	\hfill%
	\begin{subfigure}{0.5\textwidth}
		\begin{tikzpicture}[scale=0.75]
			\begin{axis}[
				ylabel={$a_w$},
				xlabel={$T_F$[\textcelsius]},
				ymax=1, xmin=-15, xmax=0,
			]
				\addplot [
					color=pblue,
					no marks,
					very thick,
				]
					table [x={FPD},y={aw},col sep=comma]
						{ge_and_wang_fpd.csv};
			\end{axis}
		\end{tikzpicture}
		\caption{Freezing Point}
	\end{subfigure}
	\caption{Relation between $T_B$/$T_F$ (in \textcelsius) and $a_w$, %
		$P=P_\text{atm}$}
	\label{fig_ge_and_wang}
\end{figure}


With $\Delta C_p$ defined as the difference in specific heat between the
phases: $\Delta C_p^\text{vap} = C_p^\text{vapor} -%
C_p^\text{liquid}$ and $\Delta C_p^\text{fus} = C_p^\text{liquid} -%
C_p^\text{solid}$, $\Delta T_F$ defined as the freezing point depression,
$\Delta T_B$ the boiling point elevation and $\Delta H_{0,T_B}^\text{vap}$ and
$\Delta H_{0,T_F}^\text{fus}$ representing the enthalpies of vaporization and fusion
of the solvent at $T_B$/$T_F$.

