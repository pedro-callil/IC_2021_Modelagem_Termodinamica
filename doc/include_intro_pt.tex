\part{Introdução}

A atividade $a$ de uma substância, definida \cite{sandler2017} como
a razão entre a fugacidade de uma substância nas condições de um sistema
e sua fugacidade em uma situação de referência (equação \ref{eq:def_atv}),
age como sua ``concentração efetiva'', do ponto de vista termodinâmico.
Portanto, a medição e análise da atividade da água em soluções nos permite
prever de modo mais exato os fenômenos que dependem do teor de água em
um sistema. No caso da indústria de alimentos, o principal desses fenômenos
é a degradação dos alimentos por ação microbiana.

De fato, a grandeza que controla o crescimento microbiano em um alimento é a
atividade da água. No gráfico da figura \ref{fig:germ}\footnote{\cite{canovas2007}}
temos os valores de atividade da água necessários para a proliferação microbiana
para alguns microorganismos.

\begin{figure}[h]
	\centering
	\begin{tikzpicture}[>=Stealth]
		\draw[->] (5,-0.5) -- (5,11);
		\draw (5,0.0) node[anchor=west] {$a_w$};
		\draw (5,8.75) -- (6.3,8.75);
		\draw (6.3,8.75) -- (13.5,8.75);
		\draw (5,7.75) -- (6.3,7.25);
		\draw (6.3,7.25) -- (13.5,7.25);
		\draw (5,6.75) -- (6.3,5.8);
		\draw (6.3,5.8) -- (13.5,5.8);
		\draw (5,5) -- (6.3,4.2);
		\draw (6.3,4.2) -- (13.5,4.2);
		\draw (5,3.75) -- (6.3,3);
		\draw (6.3,3) -- (13.5,3);
		\draw (5,1.25) -- (6.3,1.85);
		\draw (6.3,1.85) -- (13.5,1.85);
		\draw (5,0.25) -- (6.3,0.35);
		\draw (6.3,0.35) -- (13.5,0.35);
		\fill[black] (5,0.25) circle (0.07);
		\fill[black] (5,1.25) circle (0.07);
		\fill[black] (5,3.75) circle (0.07);
		\fill[black] (5,5) circle (0.07);
		\fill[black] (5,6.75) circle (0.07);
		\fill[black] (5,7.75) circle (0.07);
		\fill[black] (5,8.75) circle (0.07);
		\fill[black] (5,10) circle (0.07);
		\draw (10,10.4) node[anchor=south, text width=9cm]
		{\tiny{Microorganismos inibidos pelo menor valor de
		$a_w$ nessa faixa}};
		\draw (6.3,9.5) node[anchor=west, text width=7cm]
		{\tiny{\textit{Pseudomonas}, \textit{Escherichia},
		\textit{Proteus}, \textit{Shigella}, \textit{Klebsiella},
		\textit{Bacillus}, \textit{Clostridium perfringens},
		\textit{C. botulinum} E, G, algumas leveduras}};
		\draw (6.3,8) node[anchor=west, text width=7cm]
		{\tiny{\textit{Salmonella}, \textit{Vibrio parahaemolyticus},
		\textit{Clostridium botulinum} A, B, \textit{Listeria
		monocytogenes}, \textit{Bacillus cereus}}};
		\draw (6.3,6.5) node[anchor=west, text width=7cm]
		{\tiny{\textit{Staphylococcus aureus} (aeróbico), diversas
		leveduras (como \textit{Candida, Torulopsis, Hansenula}),
		\textit{Micrococcus}}};
		\draw (6.3,5) node[anchor=west, text width=7cm]
		{\tiny{Maior parte dos bolores (\textit{Penicillium}
		micotoxigênicos), \textit{Staphylococcus aureus},
		maior parte dos \textit{Saccharomyces}, \textit{Debaryomyces}}};
		\draw (6.3,3.5) node[anchor=west, text width=7cm]
		{\tiny{Maior parte das bactérias halófilas, \textit{Aspergillus}
		micotoxigênicos}};
		\draw (6.3,2.4) node[anchor=west, text width=7cm]
		{\tiny{Bolores xerófilos (\textit{Aspergillus chevalieri, A.
		candidus, Wallemia sebi}), \textit{Saccharomyces bisporus}}};
		\draw (6.3,1.1) node[anchor=west, text width=7cm]
		{\tiny{Leveduras osmófilas (\textit{Sacharomyces rouxii},
		alguns bolores (\textit{Aspergillus echinulatus, Monascus
		bisporus})}};
		\draw (6.3,0.1) node[anchor=west, text width=7cm]
		{\tiny{Não ocorre proliferação microbiana}};
		\draw (5,0.25) node[anchor=east] {0.61};
		\draw (5,1.25) node[anchor=east] {0.65};
		\draw (5,3.75) node[anchor=east] {0.75};
		\draw (5,5) node[anchor=east] {0.80};
		\draw (5,6.75) node[anchor=east] {0.87};
		\draw (5,7.75) node[anchor=east] {0.91};
		\draw (5,8.75) node[anchor=east] {0.95};
		\draw (5,10) node[anchor=east] {1.00};
	\end{tikzpicture}
	\caption{Valores de $a_w$ para o crescimento microbiano}
	\label{fig:germ}
\end{figure}

Nas condições encontradas na indústria de alimentos, não é absurdo assumir a
hipótese de gás ideal para a fase gás dos sistemas
considerados \cite{canovas2007}, em muitas situações. Além disso, a condição de
referência é tomada como água pura, líquida, à mesma temperatura $T$. Assim
obtemos:

\begin{equation}
	\label{eq:def_atv}
	a_w \equiv \left(\cfrac{f_w}{f^\text{ref}_w}\right)_T%
		\approx \left(\cfrac{p^\text{vap}_w}{p^\text{vap,ref}_w}\right)_T
\end{equation}

Para soluções ideais, obter $a_w$ é simples: para todo componente $i$ da
mistura, a sua atividade é dada, simplesmente por sua fração molar, e, portanto,
$a_w = x_w$. Entretanto, essa hipótese nem sempre se verifica; assim sendo,
para soluções ideais adotamos o coeficiente de atividade $\gamma_w$:

\begin{equation}
	a_w = \gamma_wx_w \implies \gamma_w = \cfrac{a_w}{x_w}
\end{equation}

Entretanto, ainda que o coeficiente de atividade seja um indicador para o
desvio da idealidade em uma solução, apresenta um problema: é uma grandeza ruim
para avaliarmos os desvios da idealidade em soluções diluídas, já que, sendo
$a_w$ função contínua da fração molar de água, e dado que a atividade da água
pura é de 1 (é, afinal, a solução de referência), temos que:

\begin{equation}
	\lim_{x_w \to 1}\gamma_w = 1
\end{equation}

Entretanto, para muitas soluções de interesse, não somente temos altos desvios da
idealidade, como também temos soluções extremamente diluídas. Então adotamos
uma nova grandeza, o coeficiente osmótico $\phi$, dado por:

\begin{equation}
	\phi = \cfrac{\ln(a_w)}{\ln(x_w)}
\end{equation}

Para comparar essas grandezas e visualizar um possível comportamento da
atividade da água com a mudança na composição de uma mistura, podemos
observar o gráfico na figura \ref{fig_atv_gamma_gluc} \cite{ebrahimi2016}.

\begin{figure}[h]
	\centering
	\begin{tikzpicture}
		\newsavebox{\minigrafglic}
		\savebox{\minigrafglic}{%
		\scalebox{0.5}{%
		\begin{tikzpicture}
			\begin{axis}[
			xmin=0.97,xmax=1.0,
			ymin=0.97,ymax=1.0,
			x dir=reverse,
			width=7cm,
			height=7cm,
			xlabel = {$x_w$},
			ylabel = {$a_w$},
			]
			\addplot+[
				color=pverydarkred,
				mark=o,
				only marks,
			]
			table[x={xw},y={aw}]{glucose_a_w_and_phi.dat};
			\addlegendentry{$a_w$ experimental};
			\addplot+[
				color=black,
				no marks,
				domain=0.96:1.0,
				samples=5,
			]{x};
			\addlegendentry{$a_w=x_w$};
		\end{axis}
	\end{tikzpicture}
	}}
	\begin{axis}[
			xmin = 0.97, xmax = 1.0, xlabel = {$x_w$},
			ymin = 0.96, ymax = 1.05, ylabel = {$\phi$},
			x tick label style={
				/pgf/number format/.cd,
				fixed,
				fixed zerofill,
				precision=3,
				/tikz/.cd
			},
			y tick label style={
				/pgf/number format/.cd,
				fixed,
				fixed zerofill,
				precision=3,
				/tikz/.cd
			},
			legend pos = south west,
			axis y line=left,
			x axis line style={-},
			xlabel near ticks,
			ylabel near ticks,
			x dir=reverse,
			ytick={0.97,0.985,1.00,1.015,1.03,1.045},
		]
		\addplot+[
			color=pdarkblue,
			mark=square,
			very thick,
			only marks,
		]
		table[x={xw},y={phi}]{glucose_a_w_and_phi.dat};
		\addlegendentry{$\phi$};
		\draw (axis cs: 0.981,0.99)node{\usebox{\minigrafglic}};
		\end{axis}
		\begin{axis}[
			xmin = 0.97, xmax = 1.0, xlabel = {$x_w$},
			ymin = 0.9977, ymax = 1.0001, ylabel = {$\gamma_w$},
			y tick label style={
				/pgf/number format/.cd,
				fixed,
				fixed zerofill,
				precision=4,
				/tikz/.cd
			},
			axis y line=right,
			axis x line=none,
			legend pos = south east,
			x dir=reverse,
		]
		\addplot+[
			color=pverybrightred,
			mark=o,
			very thick,
			only marks,
		]
		table[x={xw},y={gammaw}]{glucose_a_w_and_phi.dat};
		\addlegendentry{$\gamma_w$};
		\end{axis}
	\end{tikzpicture}
	\caption{Atividade ($a_w$), coeficiente de atividade ($\gamma_w$) e%
	coeficiente osmótico ($\phi$) da água em soluções de D-glicose. %
	Perceba a diferença entre as escalas utilizadas para $\gamma_w$ e $\phi$.}
	\label{fig_atv_gamma_gluc}
\end{figure}

Devemos, além disso, lembrar que a medida de atividade da água é, por vezes,
realizada de modo mais indireto; podemos medir para uma determinada solução ao
invés de seu valor de atividade da água (através da medição de sua pressão de
vapor e comparação com a pressão de vapor de água pura nas mesmas condições), o
seu valor de elevação do ponto de ebulição, de depressão de ponto de início de
congelamento, ou a composição de outra solução em equilíbrio osmótico.

Por exemplo, alguns conjuntos de dados obtidos para soluções $n$-árias de
carboidratos \cite{abderafi1994} apresentam dados de elevação do ponto de
ebulição. Essas medidas devem ser convertidas em dados de coeficiente osmótico,
através das equações \ref{eq:eleb_to_osco} \cite{ge2009,ge2009err}, por
exemplo, que estão exibidas na figura \ref{fig_ge_and_wang}.

\begin{equation}
	\label{eq:eleb_to_osco}
	\begin{cases}
		\ln(a_w) = -\cfrac{\Delta H^\text{vap}_{0,T_B}\left(\cfrac{1}{T_B}%
			-\cfrac{1}{T_B+\Delta T_B}\right)-%
			\Delta C_p^\text{vap}\Bigg[\ln\left(\cfrac{T_B+%
			\Delta T_B}{T_B}\right) - \cfrac{\Delta T_B}{T_B+%
			\Delta T_B}\Bigg]}{R}\\
		\ln(a_w) = \cfrac{\Delta H^\text{fus}_{0,T_F}\left(\cfrac{1}{T_F}-%
		\cfrac{1}{T_F-\Delta T_F}\right)+\Delta C_p^\text{fus}%
		\Bigg[\ln\left(\cfrac{T_F-\Delta T_F}{T_F}\right) -%
		\cfrac{\Delta T_F}{T_F-\Delta T_F}\Bigg]}{R}
	\end{cases}
\end{equation}

\begin{figure}[h]
	\centering
	\begin{subfigure}{0.5\textwidth}
		\begin{tikzpicture}[scale=0.75]
			\begin{axis}[
				ylabel={$a_w$},
				xlabel={$T_B$[\textcelsius]},
				ymax=1, xmax=115, xmin=100,
			]
				\addplot [
					color=pred,
					no marks,
					very thick,
				]
					table [x={BPE},y={aw},col sep=comma]
						{ge_and_wang_bpe.csv};
			\end{axis}
		\end{tikzpicture}
		\caption{Ponto de Ebulição}
	\end{subfigure}%
	\hfill%
	\begin{subfigure}{0.5\textwidth}
		\begin{tikzpicture}[scale=0.75]
			\begin{axis}[
				ylabel={$a_w$},
				xlabel={$T_F$[\textcelsius]},
				ymax=1, xmin=-15, xmax=0,
			]
				\addplot [
					color=pblue,
					no marks,
					very thick,
				]
					table [x={FPD},y={aw},col sep=comma]
						{ge_and_wang_fpd.csv};
			\end{axis}
		\end{tikzpicture}
		\caption{Ponto de Fusão}
	\end{subfigure}
	\caption{Relações entre $T_B$/$T_F$ (em \textcelsius) e $a_w$, %
		$P=P_\text{atm}$}
	\label{fig_ge_and_wang}
\end{figure}

Com $\Delta C_p$ sendo definido como a diferença entre os calores
específicos, entre as fases: $\Delta C_p^\text{vap} = C_p^\text{vapor} -%
C_p^\text{líquido}$ e $\Delta C_p^\text{fus} = C_p^\text{líquido} -%
C_p^\text{sólido}$, $\Delta T_F$ sendo a depressão do ponto de congelamento,
$\Delta T_B$ a elevação do ponto de ebulição e $\Delta H_{0,T_B}^\text{vap}$ e
$\Delta H_{0,T_F}^\text{fus}$ sendo as entalpias de vaporização e ebulição do
solvente puro a $T_B$/$T_F$.

