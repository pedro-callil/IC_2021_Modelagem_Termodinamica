\chapter{Computational Analysis}

\section{Discrimination between models: $a_w$ \textit{vs.} $\phi$ and %
fitting evaluation parameters}

As one would expect, fitting the models to values of $\phi$ gives us more
information than fitting the models to values of $a_w$: while $\bar{R^2}$ values
may be to close to 1 when one analyses water activity data, the analysis of
the very same dataset, converted to osmotic coefficient data, yields a very
different picture, with pronouncedly different values of $\bar{R²}$, which
better discriminates between a good or bad fit; the problems with an analysis
of $\bar{R^2}_{a_w}$ arise from the fact that $\lim_{x_w \to 1}\gamma_w = 1$,
and, therefore, for high dilutions, even relatively trivial models seem to be
an excellent fit.

In the same way, one may also observe that the (adjusted) coefficient of
determination is not a good fitting evaluation parameter: a low (even negative)
value of $\bar{R^2}_\phi$ is not a certain indicator of a bad fit: indeed, one
is usually not interested in the behavior of the osmotic coefficient; instead, the
property of interest is the water activity.

Therefore, to evaluate the quality of the fit, one must find a different parameter;
from the three under scrutiny, the best is, actually, the scaled cost function.
In the figure \ref{fig_mannitol_sucrose}\footnote{\cite{stokes1961}} one can easily
visualize the phenomena: the fitted activity models lie much closer to the
experimental activity data than the osmotic coefficient models derived from them
lie to the experimental osmotic coefficient data. The fit evaluation parameters are
exhibited (not only for models that require least-squares fitting) at the table
\ref{tab_mannitol_sucrose}, in which one can observe the advantage in the use of
the scaled cost function ($\sqrt{\frac{1}{n}\sum_{i=1}^N(\phi_{\text{exp}}-%
\phi_{\text{calc}})^2}$) as a fit evaluation parameter.

\begin{figure}[h]
\centering
\begin{subfigure}{.5\textwidth}
	\centering
	\begin{tikzpicture}[scale = 0.75]
		\begin{axis} [
				xlabel={$x_\text{sucrose}$},
				ylabel={$x_\text{manitol}$},
				zlabel={$\phi$},
				restrict z to domain=0.4:0.8,
				view={-70}{40},
				xticklabel style={
					/pgf/number format/fixed,
					/pgf/number format/precision=3,
					/pgf/number format/fixed zerofill
				},
				scaled x ticks=false,
				xtick={0.020,0.040,0.060,0.080},
				yticklabel style={
					/pgf/number format/fixed,
					/pgf/number format/precision=3,
					/pgf/number format/fixed zerofill
				},
				scaled y ticks=false,
				legend pos=north west,
			]
			\addplot3+ [
				color=pbrightred,
				only marks,
				mark = *,
				mark options={fill=pbrightred},
				thick,
			] table [
				x={sucrose},y={mannitol},z={phi_exp},col sep=comma
			] {./stokes_61_sucrose_mannitol_uni.csv};
			\addlegendentry{$\phi_\text{experimental}$};
			\addplot3+ [
				mesh,
				no marks,
				color=pverybrightblue,
			] gnuplot [raw gnuplot] {
				set dgrid 50,50 spline;
				set datafile separator ',';
				splot 'stokes_61_sucrose_mannitol_nor.csv' u 5:6:1;
			};
			\addlegendentry{$\phi_\text{Norrish}$};
			\addplot3+ [
				mesh,
				no marks,
				color=pblue,
			] gnuplot [raw gnuplot] {
				set dgrid 25,25 spline;
				set datafile separator ',';
				splot 'stokes_61_sucrose_mannitol_vir.csv' u 5:6:1;
			};
			\addlegendentry{$\phi_\text{virial}$};
			\addplot3+ [
				mesh,
				no marks,
				color=pverydarkblue,
			] gnuplot [raw gnuplot] {
				set dgrid 24,24 spline;
				set datafile separator ',';
				splot 'stokes_61_sucrose_mannitol_uni.csv' u 5:6:1;
			};
			\addlegendentry{$\phi_\text{UNIQUAC}$};
		\end{axis}
	\end{tikzpicture}
	\caption{Osmotic coefficients}
	\label{fig_mannitol_sucrose_osm}
\end{subfigure}%
\hfill%
\begin{subfigure}{.5\textwidth}
	\centering
	\begin{tikzpicture}[scale = 0.75]
		\begin{axis} [
				xlabel={$x_\text{sucrose}$},
				ylabel={$x_\text{manitol}$},
				zlabel={$a_w$},
				%view={-60}{20},
				xticklabel style={
					/pgf/number format/fixed,
					/pgf/number format/precision=3,
					/pgf/number format/fixed zerofill
				},
				scaled x ticks=false,
				xtick={0.020,0.040,0.060,0.080},
				yticklabel style={
					/pgf/number format/fixed,
					/pgf/number format/precision=3,
					/pgf/number format/fixed zerofill
				},
				scaled y ticks=false,
				legend pos=north east,
			]
			\addplot3+ [
				color=pbrightred,
				only marks,
				mark = *,
				mark options={fill=pbrightred},
				thick,
			] table [
				x={sucrose},y={mannitol},z={aw_calc},col sep=comma
			] {./stokes_61_sucrose_mannitol_uni.csv};
			\addlegendentry{$a_{w,\text{experimental}}$};
			\addplot3 [
				mesh,
				no marks,
				color=pverybrightblue,
			] gnuplot [raw gnuplot] {
				set dgrid 15,15 spline;
				set datafile separator ',';
				splot 'stokes_61_sucrose_mannitol_nor.csv' u 5:6:3;
			};
			\addlegendentry{$a_{w,\text{Norrish}}$};
			\addplot3 [
				mesh,
				no marks,
				color=pblue,
			] gnuplot [raw gnuplot] {
				set dgrid 25,25 spline;
				set datafile separator ',';
				splot 'stokes_61_sucrose_mannitol_vir.csv' u 5:6:3;
			};
			\addlegendentry{$a_{w,\text{Virial}}$};
			\addplot3 [
				mesh,
				no marks,
				color=pverydarkblue,
			] gnuplot [raw gnuplot] {
				set dgrid 24,24 spline;
				set datafile separator ',';
				splot 'stokes_61_sucrose_mannitol_uni.csv' u 5:6:3;
			};
			\addlegendentry{$a_{w,\text{UNIQUAC}}$};
		\end{axis}
	\end{tikzpicture}
	\caption{Water activities}
	\label{fig_mannitol_sucrose_atv}
\end{subfigure}
\caption{Values of $\phi$ and $a_w$ (obtained experimentally and from the models) %
	for a solution with mannitol and sucrose.}
\label{fig_mannitol_sucrose}
\end{figure}

\begin{tabularx}{\textwidth}{ X  r  r  r  r }
	\caption{Indicators for a data set}
	\label{tab_mannitol_sucrose}\\
	\toprule
	Model & $\sqrt{\frac{1}{n}\sum_{i=1}^N(\phi_{\text{exp}}-%
		\phi_{\text{calc}})^2}$ &
		$\bar{R}^2_{\phi}$ & $\bar{R}^2_{a_w}$ &
		\textnumero\ parameters\\
	\midrule
	\endfirsthead
	\toprule
	Model & $\sqrt{\frac{1}{n}\sum_{i=1}^N(\phi_{\text{exp}}-%
		\phi_{\text{calc}})^2}$ &
		$\bar{R}^2_{\phi}$ & $\bar{R}^2_{a_w}$ &
		\textnumero\ parameters\\\hline
	\midrule
	\endhead
	\midrule
	\multicolumn{5}{r}{\footnotesize(Continue in the following page)}
	\endfoot
	\endlastfoot
	Raoult & 0.422001 & -25.947383 & 0.440301 & 0 \\
	Caurie & 0.429705 & -26.940271 & 0.402607 & 0 \\
	Zdanovskii & 0.422505 & -26.012114 & 0.423267 & 0 \\
	Norrish & 0.211922 & -5.987272 & 0.814903 & 2\\
	Virial & 0.055878 & 0.514218 & 0.975995 & 3\\
	UNIQUAC & 0.055712 & 0.517119 & 0.977335 & 6\\\hline
\end{tabularx}

\section{Required number of iterations and computer time}

In figure \ref{fig_violin_iter} are exhibited the numbers of iterations required
for the convergence of Levenberg-Marquardt algorithm (with geodesic acceleration),
for the data sets analysed. One can observe the small number of iterations required
for the convergence when fitting the virial model to the data, due to its relative
simplicity (especially when comparing data from binary systems alone). Another point
to be observed is the relatively large number of iterations for the UNIQUAC model,
several magnitudes larger than the required for the other two. Besides, a few data
sets lead to extremely long computer times (a couple seconds), compared to the
majority of data sets (a few hundredths of seconds).


\begin{figure}[h]
	\centering
	\begin{tikzpicture}
		\newsavebox{\violinnorrish}
		\savebox{\violinnorrish}{%
		\begin{tikzpicture}
			\begin{axis}[
				height=0.6\textwidth,
				width=0.6\textwidth,
				ymin=0,ymax=15,
				xmin=-1,xmax=3,
				axis line style={draw=none},
				tick style={draw=none},
				xticklabels={,,},
				yticklabels={,,},
			]
			\draw[pviolinbrightblue,thick]
				(axis cs:0,0) -- (axis cs:0,10);
			\addplot+[
				color=black,
				fill=pviolinbrightblue,
				no marks,
				thick,
				smooth,
			]
			table[x={yplus},y={x},col sep=comma]
				{./norrish_violin.csv};
			\addplot+[
				color=black,
				fill=pviolinbrightblue,
				no marks,
				thick,
				smooth,
			]
			table[x={yminus},y={x},col sep=comma]
				{./norrish_violin.csv};
			\addplot+[
				color=black,
				very thick,
				mark=o,
			] coordinates {(0.0,3.98797)};
			\end{axis}
		\end{tikzpicture}
		}
		\newsavebox{\violinvirial}
		\savebox{\violinvirial}{%
		\begin{tikzpicture}
			\begin{axis}[
				height=0.6\textwidth,
				width=0.6\textwidth,
				ymin=0,ymax=15,
				xmin=-2,xmax=2,
				axis line style={draw=none},
				tick style={draw=none},
				xticklabels={,,},
				yticklabels={,,},
			]
			\draw[pviolinblue,thick]
				(axis cs:0,0) -- (axis cs:0,5);
			\addplot+[
				color=black,
				fill=pviolinblue,
				no marks,
				thick,
				smooth,
			]
			table[x={yplus},y={x},col sep=comma]
				{./virial_violin.csv};
			\addplot+[
				color=black,
				fill=pviolinblue,
				no marks,
				thick,
				smooth,
			]
			table[x={yminus},y={x},col sep=comma]
				{./virial_violin.csv};
			\addplot+[
				color=black,
				very thick,
				mark=o,
			] coordinates {(0.0,2.06878)};
			\end{axis}
		\end{tikzpicture}
		}
		\newsavebox{\violinuniquac}
		\savebox{\violinuniquac}{%
		\begin{tikzpicture}
			\begin{axis}[
				height=0.6\textwidth,
				width=0.6\textwidth,
				ymin=0,ymax=15,
				xmin=-3,xmax=1,
				axis line style={draw=none},
				tick style={draw=none},
				xticklabels={,,},
				yticklabels={,,},
			]
			\draw[pviolinblue,thick]
				(axis cs:0,0) -- (axis cs:0,15);
			\addplot+[
				color=black,
				fill=pviolindarkblue,
				no marks,
				thick,
				smooth,
			]
			table[x={yplus},y={x},col sep=comma]
				{./uniquac_violin.csv};
			\addplot+[
				color=black,
				fill=pviolindarkblue,
				no marks,
				thick,
				smooth,
			]
			table[x={yminus},y={x},col sep=comma]
				{./uniquac_violin.csv};
			\addplot+[
				color=black,
				very thick,
				mark=o,
			] coordinates {(0.0,7.8168)};
			\end{axis}
		\end{tikzpicture}
		}
	\begin{axis}[
			xmin = 1.0, xmax = 3.0, xlabel = {Model},
			ymin = 0, ymax = 15, ylabel = {\textnumero\ of iterations},
			height=0.59\textwidth,
			width=0.6\textwidth,
			xtick style={draw=none},
			xtick={1.5,2,2.5},
			xticklabels={Norrish,Virial,UNIQUAC},
			ytick={0,3,6,9,12,15},
			yticklabels={1,8,64,512,4096,32768},
		]
		\draw (axis cs:2,7.5) node {\usebox{\violinnorrish}};
		\draw (axis cs:2,7.5) node {\usebox{\violinvirial}};
		\draw (axis cs:2,7.5) node {\usebox{\violinuniquac}};
	\end{axis}
	\end{tikzpicture}
	\caption{Number of iterations required for the convergence of each model}
	\label{fig_violin_iter}
\end{figure}

One may visualize the behavior of the (scaled) cost function as the iterative
process runs, for the three models that require it, in figure \ref{fig_obj_iter},
a plot of a scaled cost function at each step of the process to obtain the fit
shown in figure \ref{fig_mannitol_sucrose}. One may observe, again, the smaller
computer time required and smaller cost function associated with the virial
model, even for data originating from a ternary system.

\begin{figure}[h]
	\centering
	\begin{tikzpicture}
		\begin{semilogyaxis} [
			ylabel={$\sqrt{\frac{1}{n}\sum_{i=1}^N(\phi_{\text{exp}}-%
			\phi_{\text{calc}})^2}$},
			xlabel={Iteração},
			xmax=27.5, xmin=-0.5,
		]
			\addplot [
				mark=o,
				smooth,
				color=pverybrightblue,
				very thick,
				only marks,
			]
			table [x={iter},y={cost},col sep=comma]
				{iter_norrish.csv};
			\addlegendentry{Norrish};
			\addplot [
				mark=o,
				smooth,
				color=pblue,
				very thick,
				only marks,
			]
			table [x={iter},y={cost},col sep=comma]
				{iter_virial.csv};
			\addlegendentry{Virial};
			\addplot [
				mark=o,
				smooth,
				color=pverydarkblue,
				very thick,
				only marks,
			]
			table [x={iter},y={cost},col sep=comma]
				{iter_uniquac.csv};
			\addlegendentry{UNIQUAC};
			\draw (axis cs: 6,0.0559) -- (axis cs: 10,5.59) -- %
				(axis cs: 18,5.59);
			\fill[pblue] (axis cs: 6,0.0559) circle (0.6mm);
			\draw (axis cs: 8,0.212) -- (axis cs: 10,5.59);
			\fill[pverybrightblue] (axis cs: 8,0.212) circle (0.6mm);
			\draw (axis cs: 12,0.2) -- (axis cs: 10,5.59);
			\draw[-{Latex[length=2mm]}]
				(axis cs: 12,0.2) -- (axis cs: 27,0.2);
			\draw (axis cs: 12,0.2) node [anchor=south west]%
				{\tiny{179 iterations for convergence}};
			\draw (axis cs: 14,5.59) node [anchor=south]%
				{\tiny{Last iteration}};
		\end{semilogyaxis}
	\end{tikzpicture}
	\caption{Scaled cost function for each step of the iterative process}
	\label{fig_obj_iter}
\end{figure}

\section{Observations about Caurie's and Zdanovskii's models}

Zdanovskii's relation is useless to analyse binary solutions; therefore, it was
only considered and compared when analysing ternary solutions. The required
isotherms were obtained through polynomial fitting over binary $a_w$ data.

Caurie's model requires a secondary model for the computation of the water activity
of binary systems. In this report this model was Raoult's law, and the equation
\ref{eq_caurie} was replaced with the equation \ref{eq_caurie_raoult}.

\begin{equation}
	\label{eq_caurie_raoult}
	a_w = \prod_{i \neq w}x_{wi}-\left[\cfrac{n}%
	{55.5^2}\sum_{\substack{i \neq j \\ i,j \neq w}}%
	m_im_j + \cfrac{(n+1)}{55.5^3}\sum_{%
	\substack{i\neq j,k \\ j \neq k \\  i,j,k \neq w}}%
	m_im_jm_k\right]
\end{equation}

