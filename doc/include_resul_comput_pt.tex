\chapter{Análise Computacional}

\section{Discriminação entre modelos: $a_w$ \textit{vs.} $\phi$ e %
parâmetros de ajuste}

Como esperado, os ajustes em $\phi$ nos fornecem muito mais informação que os
obtidos em dados de $a_w$: enquanto os valores para $\bar{R}^2$ podem estar
muito próximos a 1 ao analisarmos dados de atividade, dados de coeficiente
osmótico nos retornam valores de coeficiente (ajustado) de determinação com
capacidade muito maior de discriminação entre os modelos.

Podemos ver, também, que o coeficiente de determinação não é um bom parâmetro para
avaliar os modelos: um coeficiente baixo de determinação (ou mesmo negativo) com
relação aos dados de coeficiente osmótico não implica que o modelo não se aplica
bem aos dados de $a_w$, de interesse; da mesma forma, um valor de $\bar{R}^2$ muito
próximo a 1 não é indício, necessariamente de que um modelo seja excepcional para
predição de valores de atividade: dado que $\lim_{x_w \to 1}\gamma_w = 1$, temos
que para grandes diluições, mesmo modelos relativamente triviais (como a Lei de
Raoult) se comportam de forma excelente.

Assim sendo, para avaliarmos o ajuste, o melhor parâmetro é a média dos desvios
dos valores de $\phi$. Na figura \ref{fig_mannitol_sucrose}
\footnote{\cite{stokes1961}} temos uma visão clara do fenômeno: o ajuste é muito
mais próximo para dados de $a_w$ que para $\phi$. Os dados desses ajustes (não
somente para modelos que necessitam de regressão, como na figura) estão na tabela
\ref{tab_mannitol_sucrose}, na qual é possível perceber a vantagem de se tomar
$\sqrt{\frac{1}{n}\sum_{i=1}^N(\phi_{\text{exp}}-\phi_{\text{calc}})^2}$
como medida de exatidão dos modelos.

\begin{figure}[h]
\centering
\begin{subfigure}{.5\textwidth}
	\centering
	\begin{tikzpicture}[scale = 0.75]
		\begin{axis} [
				xlabel={$x_\text{sacarose}$},
				ylabel={$x_\text{manitol}$},
				zlabel={$\phi$},
				restrict z to domain=0.4:0.8,
				view={-70}{40},
				xticklabel style={
					/pgf/number format/fixed,
					/pgf/number format/precision=3,
					/pgf/number format/fixed zerofill
				},
				scaled x ticks=false,
				xtick={0.020,0.040,0.060,0.080},
				yticklabel style={
					/pgf/number format/fixed,
					/pgf/number format/precision=3,
					/pgf/number format/fixed zerofill
				},
				scaled y ticks=false,
				legend pos=north west,
			]
			\addplot3+ [
				color=pbrightred,
				only marks,
				mark = *,
				mark options={fill=pbrightred},
				thick,
			] table [
				x={sucrose},y={mannitol},z={phi_exp},col sep=comma
			] {./stokes_61_sucrose_mannitol_uni.csv};
			\addlegendentry{$\phi_\text{experimental}$};
			\addplot3+ [
				mesh,
				no marks,
				color=pverybrightblue,
			] gnuplot [raw gnuplot] {
				set dgrid 50,50 spline;
				set datafile separator ',';
				splot 'stokes_61_sucrose_mannitol_nor.csv' u 5:6:1;
			};
			\addlegendentry{$\phi_\text{Norrish}$};
			\addplot3+ [
				mesh,
				no marks,
				color=pblue,
			] gnuplot [raw gnuplot] {
				set dgrid 25,25 spline;
				set datafile separator ',';
				splot 'stokes_61_sucrose_mannitol_vir.csv' u 5:6:1;
			};
			\addlegendentry{$\phi_\text{virial}$};
			\addplot3+ [
				mesh,
				no marks,
				color=pverydarkblue,
			] gnuplot [raw gnuplot] {
				set dgrid 24,24 spline;
				set datafile separator ',';
				splot 'stokes_61_sucrose_mannitol_uni.csv' u 5:6:1;
			};
			\addlegendentry{$\phi_\text{UNIQUAC}$};
		\end{axis}
	\end{tikzpicture}
	\caption{Coeficientes Osmóticos}
	\label{fig_mannitol_sucrose_osm}
\end{subfigure}%
\hfill%
\begin{subfigure}{.5\textwidth}
	\centering
	\begin{tikzpicture}[scale = 0.75]
		\begin{axis} [
				xlabel={$x_\text{sacarose}$},
				ylabel={$x_\text{manitol}$},
				zlabel={$a_w$},
				%view={-60}{20},
				xticklabel style={
					/pgf/number format/fixed,
					/pgf/number format/precision=3,
					/pgf/number format/fixed zerofill
				},
				scaled x ticks=false,
				xtick={0.020,0.040,0.060,0.080},
				yticklabel style={
					/pgf/number format/fixed,
					/pgf/number format/precision=3,
					/pgf/number format/fixed zerofill
				},
				scaled y ticks=false,
				legend pos=north east,
			]
			\addplot3+ [
				color=pbrightred,
				only marks,
				mark = *,
				mark options={fill=pbrightred},
				thick,
			] table [
				x={sucrose},y={mannitol},z={aw_calc},col sep=comma
			] {./stokes_61_sucrose_mannitol_uni.csv};
			\addlegendentry{$a_{w,\text{experimental}}$};
			\addplot3 [
				mesh,
				no marks,
				color=pverybrightblue,
			] gnuplot [raw gnuplot] {
				set dgrid 15,15 spline;
				set datafile separator ',';
				splot 'stokes_61_sucrose_mannitol_nor.csv' u 5:6:3;
			};
			\addlegendentry{$a_{w,\text{Norrish}}$};
			\addplot3 [
				mesh,
				no marks,
				color=pblue,
			] gnuplot [raw gnuplot] {
				set dgrid 25,25 spline;
				set datafile separator ',';
				splot 'stokes_61_sucrose_mannitol_vir.csv' u 5:6:3;
			};
			\addlegendentry{$a_{w,\text{Virial}}$};
			\addplot3 [
				mesh,
				no marks,
				color=pverydarkblue,
			] gnuplot [raw gnuplot] {
				set dgrid 24,24 spline;
				set datafile separator ',';
				splot 'stokes_61_sucrose_mannitol_uni.csv' u 5:6:3;
			};
			\addlegendentry{$a_{w,\text{UNIQUAC}}$};
		\end{axis}
	\end{tikzpicture}
	\caption{Atividades da água}
	\label{fig_mannitol_sucrose_atv}
\end{subfigure}
\caption{Valores de $\phi$ e $a_w$ (experimentais e calculados com os modelos) para %
	uma solução de manitol e sacarose}
\label{fig_mannitol_sucrose}
\end{figure}

\begin{tabularx}{\textwidth}{ X  r  r  r  r }
	\caption{Indicadores para um conjunto de dados}
	\label{tab_mannitol_sucrose}\\
	\toprule
	Modelo & $\sqrt{\frac{1}{n}\sum_{i=1}^N(\phi_{\text{exp}}-%
		\phi_{\text{calc}})^2}$ &
		$\bar{R}^2_{\phi}$ & $\bar{R}^2_{a_w}$ &
		\textnumero\ parâmetros\\
	\midrule
	\endfirsthead
	\toprule
	Modelo & $\sqrt{\frac{1}{n}\sum_{i=1}^N(\phi_{\text{exp}}-%
		\phi_{\text{calc}})^2}$ &
		$\bar{R}^2_{\phi}$ & $\bar{R}^2_{a_w}$ &
		\textnumero\ parâmetros\\\hline
	\midrule
	\endhead
	\midrule
	\multicolumn{5}{r}{\footnotesize(Continua na página seguinte)}
	\endfoot
	\endlastfoot
	Raoult & 0.422001 & -25.947383 & 0.440301 & 0 \\
	Caurie & 0.429705 & -26.940271 & 0.402607 & 0 \\
	Zdanovskii & 0.422505 & -26.012114 & 0.423267 & 0 \\
	Norrish & 0.211922 & -5.987272 & 0.814903 & 2\\
	Virial & 0.055878 & 0.514218 & 0.975995 & 3\\
	UNIQUAC & 0.055712 & 0.517119 & 0.977335 & 6\\\hline
\end{tabularx}

\section{Número de iterações e tempo de computação}

Na figura \ref{fig_violin_iter} estão exibidos os números de iterações necessárias
para a convergência do método de Levenberg-Marquardt com aceleração geodésica,
para os conjuntos de dados avaliados.
Perceba que o modelo virial (que, de fato, é mais simples, ao menos quando se
consideram conjuntos de dados de soluções binárias) apresenta um número de
iterações ordens de grandeza menor que o necessário para a convergência dos
outros modelos, em média. Outro ponto a ser considerado é o número de iterações
para o modelo UNIQUAC, que é, por sua vez, ordens de grandeza maior que o número
de iterações necessárias para a convergência do modelo de Norrish. Além disso, é
importante ressaltar que alguns conjuntos de dados levaram a tempos de máquina
excepcionalmente longos.

\begin{figure}[h]
	\centering
	\begin{tikzpicture}
		\newsavebox{\violinnorrish}
		\savebox{\violinnorrish}{%
		\begin{tikzpicture}
			\begin{axis}[
				height=0.6\textwidth,
				width=0.6\textwidth,
				ymin=0,ymax=15,
				xmin=-1,xmax=3,
				axis line style={draw=none},
				tick style={draw=none},
				xticklabels={,,},
				yticklabels={,,},
			]
			\draw[pviolinbrightblue,thick]
				(axis cs:0,0) -- (axis cs:0,10);
			\addplot+[
				color=black,
				fill=pviolinbrightblue,
				no marks,
				thick,
				smooth,
			]
			table[x={yplus},y={x},col sep=comma]
				{./norrish_violin.csv};
			\addplot+[
				color=black,
				fill=pviolinbrightblue,
				no marks,
				thick,
				smooth,
			]
			table[x={yminus},y={x},col sep=comma]
				{./norrish_violin.csv};
			\addplot+[
				color=black,
				very thick,
				mark=o,
			] coordinates {(0.0,3.98797)};
			\end{axis}
		\end{tikzpicture}
		}
		\newsavebox{\violinvirial}
		\savebox{\violinvirial}{%
		\begin{tikzpicture}
			\begin{axis}[
				height=0.6\textwidth,
				width=0.6\textwidth,
				ymin=0,ymax=15,
				xmin=-2,xmax=2,
				axis line style={draw=none},
				tick style={draw=none},
				xticklabels={,,},
				yticklabels={,,},
			]
			\draw[pviolinblue,thick]
				(axis cs:0,0) -- (axis cs:0,5);
			\addplot+[
				color=black,
				fill=pviolinblue,
				no marks,
				thick,
				smooth,
			]
			table[x={yplus},y={x},col sep=comma]
				{./virial_violin.csv};
			\addplot+[
				color=black,
				fill=pviolinblue,
				no marks,
				thick,
				smooth,
			]
			table[x={yminus},y={x},col sep=comma]
				{./virial_violin.csv};
			\addplot+[
				color=black,
				very thick,
				mark=o,
			] coordinates {(0.0,2.06878)};
			\end{axis}
		\end{tikzpicture}
		}
		\newsavebox{\violinuniquac}
		\savebox{\violinuniquac}{%
		\begin{tikzpicture}
			\begin{axis}[
				height=0.6\textwidth,
				width=0.6\textwidth,
				ymin=0,ymax=15,
				xmin=-3,xmax=1,
				axis line style={draw=none},
				tick style={draw=none},
				xticklabels={,,},
				yticklabels={,,},
			]
			\draw[pviolinblue,thick]
				(axis cs:0,0) -- (axis cs:0,15);
			\addplot+[
				color=black,
				fill=pviolindarkblue,
				no marks,
				thick,
				smooth,
			]
			table[x={yplus},y={x},col sep=comma]
				{./uniquac_violin.csv};
			\addplot+[
				color=black,
				fill=pviolindarkblue,
				no marks,
				thick,
				smooth,
			]
			table[x={yminus},y={x},col sep=comma]
				{./uniquac_violin.csv};
			\addplot+[
				color=black,
				very thick,
				mark=o,
			] coordinates {(0.0,7.8168)};
			\end{axis}
		\end{tikzpicture}
		}
	\begin{axis}[
			xmin = 1.0, xmax = 3.0, xlabel = {Modelo},
			ymin = 0, ymax = 15, ylabel = {\textnumero\ de iterações},
			height=0.59\textwidth,
			width=0.6\textwidth,
			xtick style={draw=none},
			xtick={1.5,2,2.5},
			xticklabels={Norrish,Virial,UNIQUAC},
			ytick={0,3,6,9,12,15},
			yticklabels={1,8,64,512,4096,32768},
		]
		\draw (axis cs:2,7.5) node {\usebox{\violinnorrish}};
		\draw (axis cs:2,7.5) node {\usebox{\violinvirial}};
		\draw (axis cs:2,7.5) node {\usebox{\violinuniquac}};
	\end{axis}
	\end{tikzpicture}
	\caption{Número de iterações para a convergência de cada modelo}
	\label{fig_violin_iter}
\end{figure}

Podemos ver, como exemplo, o comportamento da função objetivo com as
iterações, para os três modelos, na figura \ref{fig_obj_iter}, obtida
ao fazermos o gráfico da função objetivo ao longo das iterações necessárias
para a obtenção do ajuste exibido na figura \ref{fig_mannitol_sucrose}. Aqui
percebemos, novamente, a maior exatidão e menor tempo de máquina exigido pelo
modelo virial, mesmo para um conjunto de dados relativos a uma mistura ternária.

\begin{figure}[h]
	\centering
	\begin{tikzpicture}
		\begin{semilogyaxis} [
			ylabel={$\sqrt{\frac{1}{n}\sum_{i=1}^N(\phi_{\text{exp}}-%
			\phi_{\text{calc}})^2}$},
			xlabel={Iteração},
			xmax=27.5, xmin=-0.5,
		]
			\addplot [
				mark=o,
				smooth,
				color=pverybrightblue,
				very thick,
				only marks,
			]
			table [x={iter},y={cost},col sep=comma]
				{iter_norrish.csv};
			\addlegendentry{Norrish};
			\addplot [
				mark=o,
				smooth,
				color=pblue,
				very thick,
				only marks,
			]
			table [x={iter},y={cost},col sep=comma]
				{iter_virial.csv};
			\addlegendentry{Virial};
			\addplot [
				mark=o,
				smooth,
				color=pverydarkblue,
				very thick,
				only marks,
			]
			table [x={iter},y={cost},col sep=comma]
				{iter_uniquac.csv};
			\addlegendentry{UNIQUAC};
			\draw (axis cs: 6,0.0559) -- (axis cs: 10,5.59) -- %
				(axis cs: 18,5.59);
			\fill[pblue] (axis cs: 6,0.0559) circle (0.6mm);
			\draw (axis cs: 8,0.212) -- (axis cs: 10,5.59);
			\fill[pverybrightblue] (axis cs: 8,0.212) circle (0.6mm);
			\draw (axis cs: 12,0.2) -- (axis cs: 10,5.59);
			\draw[-{Latex[length=2mm]}]
				(axis cs: 12,0.2) -- (axis cs: 27,0.2);
			\draw (axis cs: 12,0.2) node [anchor=south west]%
				{\tiny{179 iterações para convergência}};
			\draw (axis cs: 14,5.59) node [anchor=south]%
				{\tiny{Última iteração}};
		\end{semilogyaxis}
	\end{tikzpicture}
	\caption{Função objetivo em função da iteração}
	\label{fig_obj_iter}
\end{figure}

\section{Observações sobre os modelos de Caurie e Zdanovskii}

Para análise de dados de soluções binárias, a relação de Zdanovskii não
apresenta utilidade; assim sendo, foi considerada apenas na análise de dados
provenientes de soluções ternárias. Para obter as isotermas necessárias,
simplesmente se ajustou um polinômio de grau menor que 4 sobre dados de $a_w$
de soluções binárias de cada soluto.

O modelo de Caurie exige um modelo secundário para o cálculo da atividade da
água de solução binária com molalidade equivalente a de cada soluto. Foi
utilizada a Lei de Raoult, nesse caso, sendo a equação \ref{eq_caurie},
na prática, adotada na forma \ref{eq_caurie_raoult}

\begin{equation}
	\label{eq_caurie_raoult}
	a_w = \prod_{i \neq w}x_{wi}-\left[\cfrac{n}%
	{55.5^2}\sum_{\substack{i \neq j \\ i,j \neq w}}%
	m_im_j + \cfrac{(n+1)}{55.5^3}\sum_{%
	\substack{i\neq j,k \\ j \neq k \\  i,j,k \neq w}}%
	m_im_jm_k\right]
\end{equation}

