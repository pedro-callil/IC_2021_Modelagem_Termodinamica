\chapter{Experimental Data Analysis}

\section{Carbohydrates and Aminoacids}

More data were obtained for carbohydrate solutions (1208 data points,
distributed among 91 data sets) than for aminoacid solutions (511 data
points in 45 data sets). Besides, the number of distinct substances was
greater for carbohydrates (21, between carbohydrates and similar compounds
\footnote{%
	Namely, arabinose, butanetetrol, dextrose, erythritol, ethanediol,
	fructose, galactose, glucose, glycerol, lactose, maltitol, maltose,
	mannitol, mannose, propanetriol, raffinose, ribose, sorbitol, sucrose,
	xylitol and xylose.
}) than it was for aminoacids (16 different substances\footnote{%
	Namely, alanine, $\alpha$-aminobutyric acid, arginine, glutamic acid,
	glycine, glycylglycine, histidine, hydroxyproline, isoleucine,
	lactamide, lisine, methionine, proline, serine, threonine and valine.
}); multicomponent data were only available for carbohydrate solutions.

The aminoacid solutions analysed exhibit smaller concentrations, in average,
than the carbohydrate solutions, as one can see in the histogram in figure
\ref{fig_hist_dilui}.

\begin{figure}[h]
	\centering
	\begin{subfigure}{0.5\textwidth}
		\begin{tikzpicture} [scale=0.75]
			\begin{axis} [
				ybar,
				ylabel={\textnumero\ of solutions},
				xlabel={$a_w$},
				ymin=0, xmax=1,
				legend pos=north west,
			]
				\addplot+ [
					hist={bins=20},
					color=black,
					fill=pdarkred,
				] table [y index=0] {aminoacid_aw_avgs.csv};
				\addlegendentry{Aminoacids ($\bar{a_w}=0.9861$)};
			\end{axis}
		\end{tikzpicture}
		\caption{Aminoacids}
	\end{subfigure}%
	\hfill%
	\begin{subfigure}{0.5\textwidth}
		\begin{tikzpicture} [scale=0.75]
			\begin{axis} [
				ybar,
				ylabel={\textnumero\ of solutions},
				xlabel={$a_w$},
				ymin=0, xmax=1,
				legend pos=north west,
			]
				\addplot+ [
					hist={bins=40},
					color=black,
					fill=pdarkblue,
				] table [y index=0] {./carbohydrate_aw_avgs.csv};
				\addlegendentry{Carbohydrates ($\bar{a_w}=0.9352$)};
			\end{axis}
		\end{tikzpicture}
		\caption{Carbohydrates}
	\end{subfigure}
	\caption{Distribution of $a_w$ data}
	\label{fig_hist_dilui}
\end{figure}

The scaled cost function was compared between each model, fitted over data from
binary solutions; therefore, Zdanovskii's relation is not applicable and Caurie's
equation is simply Raoult's law.

In average, the cost function was smaller for data from aminoacid solutions,
compared to the cost functions observed for carbohydrate data. Indeed, one
may expect larger deviations from ideality (and, therefore, of the assumptions
of the models) as the dilution diminishes, and, therefore, is natural that the
models are less precise.\footnote{%
	An opposite tendency to the uncertainties originated from measurement,
	that, as commented before, grow with the dilution.
}

The models behaved differently in each group of substances; while the virial and UNIQUAC models yielded similar values for aminoacid solutions, the UNIQUAC model was the
best, by far, for carbohydrate systems. This may be observed in the table
\ref{tab_model_amino_carb}, that presents the average of the scaled cost functions
(weighted with the size of each data set) of the fitting of each model to each
bynary data series.

\begin{tabularx}{\textwidth}{ X r r }
	\caption{Model performance for different substances}
	\label{tab_model_amino_carb}\\
	\toprule
	\multirow{2}*{Model} & \multicolumn{2}{c}{%
			$\sqrt{\frac{1}{n}\sum_{i=1}^N(\phi_{\text{exp}}-%
			\phi_{\text{calc}})^2}_\text{average}$} \\
		& Aminoacid solutions & Carbohydrate solutions \\
	\midrule
	\endfirsthead
	\toprule
	\multirow{2}*{Model} & \multicolumn{2}{c}{%
			$\sqrt{\frac{1}{n}\sum_{i=1}^N(\phi_{\text{exp}}-%
			\phi_{\text{calc}})^2}_\text{average}$} \\
		& Aminoacid solutions & Carbohydrate solutions \\\hline
	\midrule
	\endhead
	\midrule
	\multicolumn{3}{r}{\footnotesize(Continue in the following page)}
	\endfoot
	\endlastfoot
	Raoult & 0.153428 & 0.479665 \\
	Norrish & 0.093486 & 0.448270 \\
	Virial & 0.049592 & 0.247174 \\
	UNIQUAC & 0.045613 & 0.197069 \\\hline
\end{tabularx}

Furthermore, one may also observe that the performance of Norrish's model,
compared to the other three, was significantly better for aminoacid solutions;
this is even clearer from scaled cost function comparisons between aminoacids
and carbohydrates for the other models.

\begin{figure}[h]
	\centering
	\begin{subfigure}{0.5\textwidth}%
		\centering
		\begin{tikzpicture}[scale=0.75]
			\newsavebox{\minigrafarg}
			\savebox{\minigrafarg}{%
			\scalebox{0.5}{%
			\begin{tikzpicture}
				\begin{axis}[
				xmin=0.0,xmax=0.015,
				ymin=0.985,ymax=1.0,
				width=7cm,
				height=7cm,
				xlabel = {$x$},
				ylabel = {$a_w$},
				xlabel near ticks,
				ylabel near ticks,
				xticklabel style={
					/pgf/number format/fixed,
					/pgf/number format/precision=3,
					/pgf/number format/fixed zerofill
				},
				scaled x ticks=false,
				yticklabel style={
					/pgf/number format/fixed,
					/pgf/number format/precision=3,
					/pgf/number format/fixed zerofill
				},
				scaled y ticks=false,
				legend pos=south west,
				]
				\addplot+[
					no marks,
					color=black,
					samples=2,
					domain=-0.001:0.016,
				] plot {1-x};
				\addlegendentry{$a_w=x_w$};
				\addplot+[
					color=pbrightred,
					mark=o,
					only marks,
					thick,
				]
				table[x={l_arginine},y={aw_exp},col sep=comma]
					{ninni_arginine_norrish.csv};
				\addplot+[
					color=pverybrightblue,
					no marks,
					thick,
				]
				table[x={l_arginine},y={aw_calc},col sep=comma]
					{ninni_arginine_norrish.csv};
				\addplot+[
					color=pblue,
					no marks,
					thick,
				]
				table[x={l_arginine},y={aw_calc},col sep=comma]
					{ninni_arginine_virial.csv};
				\addplot+[
					color=pverydarkblue,
					no marks,
					thick,
				]
				table[x={l_arginine},y={aw_calc},col sep=comma]
					{ninni_arginine_uniquac.csv};
			\end{axis}
		\end{tikzpicture}
		}}
		\begin{axis}[
				xmin = 0.002, xmax = 0.015,
				xlabel = {$x_\text{arginina}$},
				ymin = 0.62, ymax = 0.85, ylabel = {$\phi$},
				legend pos = north west,
				xlabel near ticks,
				ylabel near ticks,
				xticklabel style={
					/pgf/number format/fixed,
					/pgf/number format/precision=3,
					/pgf/number format/fixed zerofill
				},
				scaled x ticks=false,
				xtick = {0.003,0.006,0.009,0.012},
			]
			\addplot+[
				smooth,
				color=pverybrightblue,
				no marks,
				thick,
			]
			table[x={l_arginine},y={phi_calc},col sep=comma]
				{ninni_arginine_norrish.csv};
			\addlegendentry{Norrish};
			\addplot+[
				smooth,
				color=pblue,
				no marks,
				thick,
			]
			table[x={l_arginine},y={phi_calc},col sep=comma]
				{ninni_arginine_virial.csv};
			\addlegendentry{virial};
			\addplot+[
				smooth,
				color=pverydarkblue,
				no marks,
				thick,
			]
			table[x={l_arginine},y={phi_calc},col sep=comma]
				{ninni_arginine_uniquac.csv};
			\addlegendentry{UNIQUAC};
			\addplot+[
				color=pbrightred,
				mark=o,
				very thick,
				only marks,
			]
			table[x={l_arginine},y={phi_exp},col sep=comma]
				{ninni_arginine_uniquac.csv};
			\addlegendentry{experimental};
			\draw (axis cs:0.0055,0.68) node{\usebox{\minigrafarg}};
			\end{axis}
		\end{tikzpicture}
		\caption{Arginine -- aminoacid example}
		\label{fig_arginine_phi}
	\end{subfigure}%
	\hfill%
	\begin{subfigure}{0.5\textwidth}%
		\centering
		\begin{tikzpicture}[scale=0.75]
			\newsavebox{\minigrafxyl}
			\savebox{\minigrafxyl}{%
			\scalebox{0.5}{%
			\begin{tikzpicture}
				\begin{axis}[
				xmin=0.0,xmax=0.061,
				ymin=0.94,ymax=1.0,
				width=7cm,
				height=7cm,
				xlabel = {$x$},
				ylabel = {$a_w$},
				xlabel near ticks,
				ylabel near ticks,
				xticklabel style={
					/pgf/number format/fixed,
					/pgf/number format/precision=3,
					/pgf/number format/fixed zerofill
				},
				scaled x ticks=false,
				yticklabel style={
					/pgf/number format/fixed,
					/pgf/number format/precision=3,
					/pgf/number format/fixed zerofill
				},
				scaled y ticks=false,
				legend pos=south west,
				]
				\addplot+[
					no marks,
					color=black,
					samples=2,
					domain=-0.001:0.061,
				] plot {1-x};
				\addlegendentry{$a_w=x_w$};
				\addplot+[
					color=pbrightred,
					mark=o,
					only marks,
					thick,
				]
				table[x={xylose},y={aw_exp},col sep=comma]
					{./ebrahimi_xylose_norrish.csv};
				\addplot+[
					color=pverybrightblue,
					no marks,
					thick,
				]
				table[x={xylose},y={aw_calc},col sep=comma]
					{./ebrahimi_xylose_norrish.csv};
				\addplot+[
					color=pblue,
					no marks,
					thick,
				]
				table[x={xylose},y={aw_calc},col sep=comma]
					{./ebrahimi_xylose_virial.csv};
				\addplot+[
					color=pverydarkblue,
					no marks,
					thick,
				]
				table[x={xylose},y={aw_calc},col sep=comma]
					{./ebrahimi_xylose_uniquac.csv};
			\end{axis}
		\end{tikzpicture}
		}}
		\begin{axis}[
				xmin = 0.000, xmax = 0.061,
				xlabel = {$x_\text{xilose}$},
				ymin = 0.978, ymax = 1.02, ylabel = {$\phi$},
				legend pos = south east,
				xlabel near ticks,
				ylabel near ticks,
				xticklabel style={
					/pgf/number format/fixed,
					/pgf/number format/precision=3,
					/pgf/number format/fixed zerofill
				},
				scaled x ticks=false,
				yticklabel style={
					/pgf/number format/fixed,
					/pgf/number format/precision=3,
					/pgf/number format/fixed zerofill
				},
				scaled y ticks=false,
				xtick = {0.015,0.030,0.045,0.060},
			]
			\addplot+[
				smooth,
				color=pverybrightblue,
				no marks,
				thick,
			]
			table[x={xylose},y={phi_calc},col sep=comma]
				{ebrahimi_xylose_norrish.csv};
			\addlegendentry{Norrish};
			\addplot+[
				smooth,
				color=pblue,
				no marks,
				thick,
			]
			table[x={xylose},y={phi_calc},col sep=comma]
				{ebrahimi_xylose_virial.csv};
			\addlegendentry{virial};
			\addplot+[
				smooth,
				color=pverydarkblue,
				no marks,
				thick,
			]
			table[x={xylose},y={phi_calc},col sep=comma]
				{ebrahimi_xylose_uniquac.csv};
			\addlegendentry{UNIQUAC};
			\addplot+[
				color=pbrightred,
				mark=o,
				very thick,
				only marks,
			]
			table[x={xylose},y={phi_exp},col sep=comma]
				{ebrahimi_xylose_uniquac.csv};
			\addlegendentry{experimental};
			\draw (axis cs:0.017,1.008) node{\usebox{\minigrafxyl}};
			\end{axis}
		\end{tikzpicture}
		\caption{Xylose -- carbohydrate example}
		\label{fig_xylose_phi}
	\end{subfigure}
	\caption{Examples of data sets and model fitting}
	\label{fig_tipi_datasets_xylose_arginine}
\end{figure}

An example of the behavior of the data (and the models) is shown in figure
\ref{fig_tipi_datasets_xylose_arginine}\footnote{\cite{ninni2001,ebrahimi2016}}:
one can observe a larger deviation from ideality in the aminoacid solutions,
making Raoult's law even worse than the other models, in comparison with its
performance for the carbohydrate solution; besides, $\phi$ is never greater than 1,
as Norrish's model mathematically requires, for the aminoacid solution, but not
for the carbohydrate data, exemplifying the difference shown in the table. Another
observation can be done about the similarities between UNIQUAC and Virial models,
with both approaching the average of the data in the first plot, and presenting
similar curves for dilutions higher than $x_w=0.99$, in the second one.

\begin{figure}[h]
	\centering
	\begin{subfigure}{0.5\textwidth}%
		\centering
		\begin{tikzpicture}[scale=0.75]
			\newsavebox{\minigrafglyc}
			\savebox{\minigrafglyc}{%
			\scalebox{0.5}{%
			\begin{tikzpicture}
				\begin{axis}[
				xmin=0.0,xmax=0.030,
				ymin=0.985,ymax=1.0,
				width=7cm,
				height=3.5cm,
				xlabel = {$x$},
				ylabel = {$a_w$},
				xlabel near ticks,
				ylabel near ticks,
				xticklabel style={
					/pgf/number format/fixed,
					/pgf/number format/precision=3,
					/pgf/number format/fixed zerofill
				},
				scaled x ticks=false,
				yticklabel style={
					/pgf/number format/fixed,
					/pgf/number format/precision=3,
					/pgf/number format/fixed zerofill
				},
				scaled y ticks=false,
				legend pos=south west,
				]
				\addplot+[
					no marks,
					color=black,
					samples=2,
					domain=-0.001:0.03,
				] plot {1-x};
				\addlegendentry{$a_w=x_w$};
				\addplot+[
					color=pbrightred,
					mark=o,
					only marks,
					thick,
				]
				table[x={glycylglycine},y={aw_exp},col sep=comma]
					{ellerton_glycylglycine_norrish.csv};
				\addplot+[
					color=pverybrightblue,
					no marks,
					thick,
				]
				table[x={glycylglycine},y={aw_calc},col sep=comma]
					{ellerton_glycylglycine_norrish.csv};
				\addplot+[
					color=pblue,
					no marks,
					thick,
				]
				table[x={glycylglycine},y={aw_calc},col sep=comma]
					{ellerton_glycylglycine_virial.csv};
				\addplot+[
					color=pverydarkblue,
					no marks,
					thick,
				]
				table[x={glycylglycine},y={aw_calc},col sep=comma]
					{ellerton_glycylglycine_uniquac.csv};
			\end{axis}
		\end{tikzpicture}
		}}
		\begin{axis}[
				xmin = 0.0, xmax = 0.03,
				xlabel = {$x_\text{glicilglicina}$},
				ymin = 0.42, ymax = 0.5, ylabel = {$\phi$},
				legend pos = north east,
				xlabel near ticks,
				ylabel near ticks,
				xticklabel style={
					/pgf/number format/fixed,
					/pgf/number format/precision=3,
					/pgf/number format/fixed zerofill
				},
				scaled x ticks=false,
				xtick = {0.006,0.012,0.018,0.024},
			]
			\addplot+[
				smooth,
				color=pverybrightblue,
				no marks,
				thick,
			]
			table[x={glycylglycine},y={phi_calc},col sep=comma]
				{ellerton_glycylglycine_norrish.csv};
			\addlegendentry{Norrish};
			\addplot+[
				smooth,
				color=pblue,
				no marks,
				thick,
			]
			table[x={glycylglycine},y={phi_calc},col sep=comma]
				{ellerton_glycylglycine_virial.csv};
			\addlegendentry{virial};
			\addplot+[
				color=pverydarkblue,
				no marks,
				thick,
			]
			table[x={glycylglycine},y={phi_calc},col sep=comma]
				{ellerton_glycylglycine_uniquac.csv};
			\addlegendentry{UNIQUAC};
			\addplot+[
				color=pbrightred,
				mark=o,
				very thick,
				only marks,
			]
			table[x={glycylglycine},y={phi_exp},col sep=comma]
				{ellerton_glycylglycine_uniquac.csv};
			\addlegendentry{experimental};
			\draw (axis cs:0.009,0.432) node{\usebox{\minigrafglyc}};
			\end{axis}
		\end{tikzpicture}
		\caption{Glycylglycine -- aminoacid example}
		\label{fig_glycylglycine_phi}
	\end{subfigure}%
	\hfill%
	\begin{subfigure}{0.5\textwidth}%
		\centering
		\begin{tikzpicture}[scale=0.75]
			\newsavebox{\minigrafsuc}
			\savebox{\minigrafsuc}{%
			\scalebox{0.5}{%
			\begin{tikzpicture}
				\begin{axis}[
				xmin=0.0,xmax=0.1,
				ymin=0.9,ymax=1.0,
				width=7cm,
				height=7cm,
				xlabel = {$x$},
				ylabel = {$a_w$},
				xlabel near ticks,
				ylabel near ticks,
				xticklabel style={
					/pgf/number format/fixed,
					/pgf/number format/precision=3,
					/pgf/number format/fixed zerofill
				},
				scaled x ticks=false,
				yticklabel style={
					/pgf/number format/fixed,
					/pgf/number format/precision=3,
					/pgf/number format/fixed zerofill
				},
				scaled y ticks=false,
				legend pos=south west,
				]
				\addplot+[
					no marks,
					color=black,
					samples=2,
					domain=-0.001:0.1,
				] plot {1-x};
				\addlegendentry{$a_w=x_w$};
				\addplot+[
					color=pbrightred,
					mark=o,
					only marks,
					thick,
				]
				table[x={sucrose},y={aw_exp},col sep=comma]
					{./stokes_sucrose_uniquac_orig.csv};
				\addplot+[
					color=pverybrightblue,
					no marks,
					thick,
				]
				table[x={sucrose},y={aw_calc},col sep=comma]
					{./stokes_sucrose_norrish.csv};
				\addplot+[
					color=pblue,
					no marks,
					thick,
				]
				table[x={sucrose},y={aw_calc},col sep=comma]
					{./stokes_sucrose_virial.csv};
				\addplot+[
					color=pverydarkblue,
					no marks,
					thick,
				]
				table[x={sucrose},y={aw_calc},col sep=comma]
					{./stokes_sucrose_uniquac.csv};
			\end{axis}
		\end{tikzpicture}
		}}
		\begin{axis}[
				xmin = 0.000, xmax = 0.1,
				xlabel = {$x_\text{sacarose}$},
				ymin = 0.45, ymax = 1.0, ylabel = {$\phi$},
				legend pos = south east,
				xlabel near ticks,
				ylabel near ticks,
				xticklabel style={
					/pgf/number format/fixed,
					/pgf/number format/precision=3,
					/pgf/number format/fixed zerofill
				},
				scaled x ticks=false,
				yticklabel style={
					/pgf/number format/fixed,
					/pgf/number format/precision=3,
					/pgf/number format/fixed zerofill
				},
				scaled y ticks=false,
				xtick = {0.016,0.032,0.048,0.064,0.080},
			]
			\addplot+[
				smooth,
				color=pverybrightblue,
				no marks,
				thick,
			]
			table[x={sucrose},y={phi_calc},col sep=comma]
				{stokes_sucrose_norrish.csv};
			\addlegendentry{Norrish};
			\addplot+[
				smooth,
				color=pblue,
				no marks,
				thick,
			]
			table[x={sucrose},y={phi_calc},col sep=comma]
				{stokes_sucrose_virial.csv};
			\addlegendentry{virial};
			\addplot+[
				smooth,
				color=pverydarkblue,
				no marks,
				thick,
			]
			table[x={sucrose},y={phi_calc},col sep=comma]
				{stokes_sucrose_uniquac.csv};
			\addlegendentry{UNIQUAC};
			\addplot+[
				color=pbrightred,
				mark=o,
				very thick,
				only marks,
			]
			table[x={sucrose},y={phi_exp},col sep=comma]
				{stokes_sucrose_uniquac_orig.csv};
			\addlegendentry{experimental};
			\draw (axis cs:0.07,0.870) node{\usebox{\minigrafsuc}};
			\end{axis}
		\end{tikzpicture}
		\caption{Sucrose -- carbohydrate example}
		\label{fig_sucrose_phi}
	\end{subfigure}
	\caption{Examples of data sets and model fitting}
	\label{fig_tipi_datasets_sucrose_glycylglycine}
\end{figure}

One may observe in figure \ref{fig_tipi_datasets_sucrose_glycylglycine}
\footnote{\cite{ellerton1964,stokes1961}}
a different reason for the large differences between fitting Norrish's model over
aminoacid or carbohydrate data; for carbohydrate solutions, usually the data
resembles a decreasing function $a_w=a_w(x_w)$ in a range of its domain, which is not
as usual for data from aminoacids. This is a problem for the model, since it can't
handle negative values of $\frac{\partial \phi}{\partial x_w}$.

\section{Dilution effects}

The dilution doesn't appear as a significant factor in the values of the
scaled cost function, when fitting the models analysed. This can be seen in
figure \ref{fig_dilution_all}.

\begin{figure}[h]
	\centering
	\begin{tikzpicture}
		\begin{semilogyaxis} [
			ylabel={$\sqrt{\frac{1}{n}\sum_{i=1}^N(\phi_{\text{exp}}-%
			\phi_{\text{calc}})^2}$},
			xlabel={$x_w$},
			legend pos=south west,
			xmin=0.85,
			xmax=1.0,
		]
			\addplot [
				mark=*,
				color=black,
				fill=pred,
				only marks,
				mark size=3pt,
			]
			table [x={xw},y={raoult},col sep=comma]
				{dilution_mono.csv};
			\addlegendentry{Raoult};
			\addplot [
				mark=*,
				color=black,
				fill=pverybrightblue,
				only marks,
				mark size=3pt,
			]
			table [x={xw},y={norrish},col sep=comma]
				{dilution_mono.csv};
			\addlegendentry{Norrish};
			\addplot [
				mark=*,
				color=black,
				fill=pblue,
				only marks,
				mark size=3pt,
			]
			table [x={xw},y={virial},col sep=comma]
				{dilution_mono.csv};
			\addlegendentry{Virial};
			\addplot [
				mark=*,
				color=black,
				fill=pverydarkblue,
				only marks,
				mark size=3pt,
			]
			table [x={xw},y={uniquac},col sep=comma]
				{dilution_mono.csv};
			\addlegendentry{UNIQUAC};
		\end{semilogyaxis}
	\end{tikzpicture}
	\caption{Scaled cost function and dilution data (binary systems)}
	\label{fig_dilution_all}
\end{figure}


\section{Effects originated from differences in experimental procedures}

As seen in section \ref{sec_selec_data}, the data were obtained through
a plethora of experimental procedures, such as vapor pressure measurements,
freezing point depression, boiling point elevation, \textit{etc.} Among the
data sets obtained from similar substances (\textit{i.e.} aminoacids and
carbohydrates), one may compare the differences in the scaled cost function
between data sets obtained through distinct experimental procedures.

\subsection{Aminoacids}

\label{sec_metod_amin}

The four models suitable for binary data were fitted to data obtained through
osmotic coefficient, water activity, freezing point depression and vapor pressure
measurements, as listed in the sources in section \ref{sec_selec_data}, and the
differences in the scaled cost function were analysed. The only statistically
significant difference among the groups was between the scaled cost functions of
fitting an UNIQUAC model to data obtained through freezing point depression and
to data originated from osmotic coefficients measurements.

However, since the data sets comprising the former group were rather small, the
most reasonable explanation lies in the ocurrence of overfitting, due to the small
diferences between the number $N$ of fitting parameters and the number $n$ of
experimental data points.

\subsection{Carbohydrates}

The four models suitable for binary data were fitted to the data sets obtained
through osmotic coefficient, water activity, freezing point depression, boiling
point elevation and equilibrium relative humidity measurements. For coherence with
section \ref{sec_metod_amin}, only data obtained from binary systems were analysed.

A statistically significant difference exists, between the scaled cost functions
obtained from data from boiling point elevation, osmotic coefficient measurements
and the remaining data sets. However, this can, again, be more reasonably explained
through differences between studies; indeed, a large part of first of these
groups consists data from Abderafi \& Bounahmidi \footnote{Namely, the experiments
were performed under higher temperatures.} \cite{abderafi1994} which may lead to
a few peculiarities in the group, arising from characteristics of one specific
experiment.

\section{Temperature influence}

To assess the temperature effect in the quality of the fit, one needs to
evaluate the average of the scaled cost functions when fitting the models
to data obtained at different temperatures; there are data sets suitable
for this analysis, for carbohydrates \cite{velezmoro2000} and aminoacids
\cite{romero2006,tsurko2007} alike.

Through this procedure, one may observe that the effects of the temperature
over the quality of the fit are not universal, ocurring in a few specific
conditions dependent on the chemical nature of the components and their dilution.

\subsection{Aminoacids}

Data from two articles were analysed; the first \cite{tsurko2007} presents a
comparison of three aminoacids (glutamic acid, glycine and histidine) under
temperatures of 298.15K and 310.15K. The other \cite{romero2006} compares
another three (alanine, $\alpha$-aminobutyric acid and glycine) under
temperatures ranging from 288.15K to 303.15K. As an example, four data sets from
the latter were plotted in figure \ref{fig_temp_amins}.

No statistically significant difference was found through the scaled cost
functions obtained when fitting the models to data obtained under each
temperature value, for data from both articles.

The reason for this is the same reason for the better adjustment of Norrish's
model to aminoacid data: as $\phi$ is smaller than one, an adjustment remains
possible as the temperature changes.

\begin{figure}[h]
	\centering
	\begin{tikzpicture}
		\begin{axis} [
			ylabel={$\phi$},
			xlabel={$x_\text{glycine}$},
			xtick={0.002,0.006,0.010,0.014,0.018,0.022},
			xlabel near ticks,
			ylabel near ticks,
			xticklabel style={
				/pgf/number format/fixed,
				/pgf/number format/precision=3,
				/pgf/number format/fixed zerofill
			},
			scaled x ticks=false,
		]
			\addplot [
				mark=*,
				color=black,
				fill=pverybrightred,
				only marks,
				mark size=3pt,
			]
			table [x={glycine},y={phi_exp},col sep=comma]
				{./romero_glycine_288_15_K.csv};
			\addlegendentry{288.15K};
			\addplot [
				mark=*,
				color=black,
				fill=pred,
				only marks,
				mark size=3pt,
			]
			table [x={glycine},y={phi_exp},col sep=comma]
				{./romero_glycine_293_15_K.csv};
			\addlegendentry{293.15K};
			\addplot [
				mark=*,
				color=black,
				fill=pverydarkred,
				only marks,
				mark size=3pt,
			]
			table [x={glycine},y={phi_exp},col sep=comma]
				{./romero_glycine_298_15_K.csv};
			\addlegendentry{288.15K};
			\addplot [
				mark=*,
				color=black,
				fill=black,
				only marks,
				mark size=3pt,
			]
			table [x={glycine},y={phi_exp},col sep=comma]
				{./romero_glycine_303_15_K.csv};
			\addlegendentry{303.15K};
		\end{axis}
	\end{tikzpicture}
	\caption{Experimental values of $\phi$ obtained from glycine solution data}
	\label{fig_temp_amins}
\end{figure}

\subsection{Carbohydrates}

Through an analysis of water activity data as a function of composition
\cite{velezmoro2000}, one may not assert the existence of a significant
difference between the average scaled cost functions, when fitting virial
and UNIQUAC models. However, both Norrish's and Raoult's models exhibit
greater cost function values when fitting experiments at 35\textcelsius,
in comparison with the quality of the fit for data obtained at 25 or
30\textcelsius. Besides, the scaled cost functions were very similar for
the two models; therefore, one may deduce that for the data sets analysed,
the Norrish model converges to Raoult's law, and, even with $\phi > 1$ at
25\textcelsius, the rise in temperatures must cause the rise in the values
of $\phi$ as well. This can, indeed, be visualized in figure \ref{fig_temp_carbs}.

\begin{figure}[h]
	\centering
	\begin{tikzpicture}
		\begin{axis} [
			ylabel={$\phi$},
			xlabel={$x_\text{maltose}$},
			xtick={0.004,0.008,0.012,0.016,0.020},
			xlabel near ticks,
			ylabel near ticks,
			xticklabel style={
				/pgf/number format/fixed,
				/pgf/number format/precision=3,
				/pgf/number format/fixed zerofill
			},
			scaled x ticks=false,
			xmin=0.0000,
		]
			\draw[black,thick] (axis cs:0.0000,1) -- (axis cs:0.022,1);
			\draw (axis cs: 0.016,0.9)
				node[anchor=north] {\tiny{Raoult's Law}};
			\draw (axis cs: 0.0135,0.9) -- (axis cs:0.0185,0.9);
			\draw (axis cs: 0.0125,1.0) -- (axis cs: 0.0135,0.9);
			\draw (axis cs: 0.0128,1.0) -- (axis cs: 0.0138,0.9);
			\draw (axis cs: 0.0131,1.0) -- (axis cs: 0.0141,0.9);
			\fill [
				color=lightgray,
				opacity=0.5,
			]
				(axis cs: 0.0000,1.0) rectangle (axis cs: 0.022,2.8);
			\draw (rel axis cs: 0.5,0.75)
				node[anchor=south]
				{\tiny{Region unapproachable}};
			\draw (rel axis cs: 0.5,0.75)
				node[anchor=north]
				{\tiny{to Norrish's model}};
			\addplot [
				mark=*,
				color=black,
				fill=pverybrightred,
				only marks,
				mark size=3pt,
			]
			table [x={maltose},y={phi_exp},col sep=comma]
				{./velezmoro_maltose_25.csv};
			\addlegendentry{25\textcelsius};
			\addplot [
				mark=*,
				color=black,
				fill=pred,
				only marks,
				mark size=3pt,
			]
			table [x={maltose},y={phi_exp},col sep=comma]
				{./velezmoro_maltose_30.csv};
			\addlegendentry{30\textcelsius};
			\addplot [
				mark=*,
				color=black,
				fill=pverydarkred,
				only marks,
				mark size=3pt,
			]
			table [x={maltose},y={phi_exp},col sep=comma]
				{./velezmoro_maltose_35.csv};
			\addlegendentry{35\textcelsius};
		\end{axis}
	\end{tikzpicture}
	\caption{Experimental $\phi$ values obtained from maltose solutions}
	\label{fig_temp_carbs}
\end{figure}

\section{Pressure influence}

The virial and Norrish's models were fitted to the data of water activity as
a function of composition of glucose and fructose solutions at different
pressures, obtained from Maximo, Meirelles \& Batista \cite{maximo2010}.
As the size of each data set was small (5 points), the UNIQUAC model was not
applied.

As Norrish's model kept its trend to simply converge to Raoult's law,
the only two models actually analysed were the virial model and Raoult's law,
as exhibited in figure \ref{fig_pressure}.

One can observe a slight downwards trend of the cost function with pressure
increases, but the trend is not statistically significant.

\begin{figure}[h]
	\centering
	\begin{subfigure}{0.5\textwidth}{%
		\begin{tikzpicture}[scale=0.75]
			\begin{axis} [
				ylabel={$\sqrt{\frac{1}{n}\sum_{i=1}^N(%
					\phi_{\text{exp}}-%
				\phi_{\text{calc}})^2}$},
				xlabel={Pressure (kPa)},
				xlabel near ticks,
				ylabel near ticks,
				scaled x ticks=false,
				ymin=0.3,ymax=1.5,
			]
				\addplot [
					mark=*,
					color=black,
					fill=pblue,
					only marks,
					mark size=3pt,
				]
				table [x={pressure},y={virial},col sep=comma]
					{./pressure_glucose.csv};
				\addlegendentry{Virial Model};
				\addplot [
					no marks,
					color=pblue,
					samples=3,
					domain=20:100,
					ultra thick,
				] plot {-0.000585*x+0.7056};
				\addlegendentry{$R^2=0.0065$};
				\addplot [
					mark=*,
					color=black,
					fill=pred,
					only marks,
					mark size=3pt,
				]
				table [x={pressure},y={raoult},col sep=comma]
					{./pressure_glucose.csv};
				\addlegendentry{Raoult's Law};
				\addplot [
					no marks,
					color=pred,
					samples=3,
					domain=20:100,
					ultra thick,
				] plot {-0.001272*x+1.0666};
				\addlegendentry{$R^2=0.0204$};
			\end{axis}
		\end{tikzpicture}}
		\caption{Glucose}
		\label{fig_pressure_glic}
	\end{subfigure}%
	\hfill%
	\begin{subfigure}{0.5\textwidth}{%
		\begin{tikzpicture}[scale=0.75]
			\begin{axis} [
				ylabel={$\sqrt{\frac{1}{n}\sum_{i=1}^N(%
					\phi_{\text{exp}}-%
				\phi_{\text{calc}})^2}$},
				xlabel={Pressure (kPa)},
				xlabel near ticks,
				ylabel near ticks,
				scaled x ticks=false,
				ymin=0.5,ymax=1.8,
			]
				\addplot [
					mark=*,
					color=black,
					fill=pblue,
					only marks,
					mark size=3pt,
				]
				table [x={pressure},y={virial},col sep=comma]
					{./pressure_fructose.csv};
				\addlegendentry{Virial Model};
				\addplot [
					no marks,
					color=pblue,
					samples=3,
					domain=20:100,
					ultra thick,
				] plot {-0.00355*x+1.0137};
				\addlegendentry{$R^2=0.3086$};
				\addplot [
					mark=*,
					color=black,
					fill=pred,
					only marks,
					mark size=3pt,
				]
				table [x={pressure},y={raoult},col sep=comma]
					{./pressure_fructose.csv};
				\addlegendentry{Raoult's Law};
				\addplot [
					no marks,
					color=pred,
					samples=3,
					domain=20:100,
					ultra thick,
				] plot {-0.005321*x+1.5911};
				\addlegendentry{$R^2=0.4052$};
			\end{axis}
		\end{tikzpicture}}
		\caption{Fructose}
		\label{fig_pressure_fruc}
	\end{subfigure}
	\caption{Cost functions (scaled) for model fitting over $\phi$ data %
		under different pressures}
	\label{fig_pressure}
\end{figure}

