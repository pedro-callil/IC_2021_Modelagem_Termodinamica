\chapter{Análise dos Dados Experimentais}

\section{Carboidratos e aminoácidos}

Foram obtidos mais dados para soluções de carboidratos (1208, distribuídos em
91 conjuntos de dados) do que para soluções de aminoácidos (511, em 45 conjuntos).
Além disso, os dados para soluções de carboidratos incluem uma variedade maior de
substâncias (21, entre carboidratos e compostos semelhantes\footnote{%
	São elas: arabinose, butanotetrol, dextrose, eritritol, etanodiol,
	frutose, galactose, glicose, glicerol, lactose, maltitol, maltose,
	manitol, manose, propanotriol, rafinose, ribose, sorbitol, sacarose,
	xilitol e xilose.
}
) que os dados para aminoácidos (16 substâncias\footnote{%
	São elas: alanina, ácido $\alpha$-aminobutírico, arginina, ácido
	glutâmico, glicina, glicilglicina, histidina, hidroxiprolina,
	isoleucina, lactamida, lisina, metionina, prolina, serina,
	treonina e valina.
}
); por fim, dados para sistemas multicomponente incluem apenas carboidratos.

Os dados para aminoácidos apresentam diluições maiores, em média, que os de
soluções de carboidratos. Isso pode ser visto no histograma da figura
\ref{fig_hist_dilui}.

\begin{figure}[h]
	\centering
	\begin{subfigure}{0.5\textwidth}
		\begin{tikzpicture} [scale=0.75]
			\begin{axis} [
				ybar,
				ylabel={\textnumero\ de soluções},
				xlabel={$a_w$},
				ymin=0, xmax=1,
				legend pos=north west,
			]
				\addplot+ [
					hist={bins=20},
					color=black,
					fill=pdarkred,
				] table [y index=0] {aminoacid_aw_avgs.csv};
				\addlegendentry{Aminoácidos ($\bar{a_w}=0.9861$)};
			\end{axis}
		\end{tikzpicture}
		\caption{Aminoácidos}
	\end{subfigure}%
	\hfill%
	\begin{subfigure}{0.5\textwidth}
		\begin{tikzpicture} [scale=0.75]
			\begin{axis} [
				ybar,
				ylabel={\textnumero\ de soluções},
				xlabel={$a_w$},
				ymin=0, xmax=1,
				legend pos=north west,
			]
				\addplot+ [
					hist={bins=40},
					color=black,
					fill=pdarkblue,
				] table [y index=0] {./carbohydrate_aw_avgs.csv};
				\addlegendentry{Carboidratos ($\bar{a_w}=0.9352$)};
			\end{axis}
		\end{tikzpicture}
		\caption{Carboidratos}
	\end{subfigure}
	\caption{Distribuição dos valores de $a_w$ real}
	\label{fig_hist_dilui}
\end{figure}

Foram comparados os ajustes dos modelos sobre soluções binárias, de carboidratos
e de aminoácidos. Assim sendo, a Relação de Zdanovskii não se aplica e a Equação
de Caurie não difere do modelo de Raoult, como discutido anteriormente.

Foi observado um desvio médio nos valores de $\phi$ menor para os dados de $a_w$
em soluções de aminoácidos, relativo ao desvio médio obtido para soluções de
carboidratos; de fato, é esperado que soluções com baixas diluições apresentem
maiores desvios da idealidade (e de quaisquer hipóteses que tenham sido adotadas
na dedução dos modelos) e, portanto, é natural que obtenhamos resultados menos
precisos, nesse caso.\footnote{%
	Excetuando-se, claro, a questão dos erros de medição, que aumentam
	conforme aumenta a diluição, como discutido anteriormente.
}

Os modelos se comportaram de forma distinta para cada grupo; enquanto para dados
obtidos a partir de soluções de aminoácidos o modelo virial se destacou, obtendo
desvios similares ao modelo UNIQUAC, para soluções de carboidratos o modelo UNIQUAC
obteve, de longe, os menores desvios. Isso pode ser observado na tabela
\ref{tab_model_amino_carb}, que exibe a média dos desvios (ponderada pela quantidade
de dados em cada experimento) ao ajustarmos cada modelo a cada conjunto de dados
para sistemas binários.


\begin{tabularx}{\textwidth}{ X r r }
	\caption{Performance dos modelos para diferentes substâncias}
	\label{tab_model_amino_carb}\\
	\toprule
	\multirow{2}*{Modelo} & \multicolumn{2}{c}{%
			$\sqrt{\frac{1}{n}\sum_{i=1}^N(\phi_{\text{exp}}-%
			\phi_{\text{calc}})^2}_\text{médio}$} \\
		& Soluções de aminoácidos & Soluções de carboidratos \\
	\midrule
	\endfirsthead
	\toprule
	\multirow{2}*{Modelo} & \multicolumn{2}{c}{%
			$\sqrt{\frac{1}{n}\sum_{i=1}^N(\phi_{\text{exp}}-%
			\phi_{\text{calc}})^2}_\text{médio}$} \\
		& Aminoácidos & Carboidratos \\\hline
	\midrule
	\endhead
	\midrule
	\multicolumn{3}{r}{\footnotesize(Continua na página seguinte)}
	\endfoot
	\endlastfoot
	Raoult & 0.153428 & 0.479665 \\
	Norrish & 0.093486 & 0.448270 \\
	Virial & 0.049592 & 0.247174 \\
	UNIQUAC & 0.045613 & 0.197069 \\\hline
\end{tabularx}

Além disso, podemos observar que o modelo de Norrish apresenta resultados
consideravelmente melhores para aminoácidos, em comparação com os desvios obtidos
para soluções de carboidratos; isso se torna ainda mais claro ao compararmos
as variações nos desvios entre carboidratos e aminoácidos para os modelos
avaliados.

\begin{figure}[h]
	\centering
	\begin{subfigure}{0.5\textwidth}%
		\centering
		\begin{tikzpicture}[scale=0.75]
			\newsavebox{\minigrafarg}
			\savebox{\minigrafarg}{%
			\scalebox{0.5}{%
			\begin{tikzpicture}
				\begin{axis}[
				xmin=0.0,xmax=0.015,
				ymin=0.985,ymax=1.0,
				width=7cm,
				height=7cm,
				xlabel = {$x$},
				ylabel = {$a_w$},
				xlabel near ticks,
				ylabel near ticks,
				xticklabel style={
					/pgf/number format/fixed,
					/pgf/number format/precision=3,
					/pgf/number format/fixed zerofill
				},
				scaled x ticks=false,
				yticklabel style={
					/pgf/number format/fixed,
					/pgf/number format/precision=3,
					/pgf/number format/fixed zerofill
				},
				scaled y ticks=false,
				legend pos=south west,
				]
				\addplot+[
					no marks,
					color=black,
					samples=2,
					domain=-0.001:0.016,
				] plot {1-x};
				\addlegendentry{$a_w=x_w$};
				\addplot+[
					color=pbrightred,
					mark=o,
					only marks,
					thick,
				]
				table[x={l_arginine},y={aw_exp},col sep=comma]
					{ninni_arginine_norrish.csv};
				\addplot+[
					color=pverybrightblue,
					no marks,
					thick,
				]
				table[x={l_arginine},y={aw_calc},col sep=comma]
					{ninni_arginine_norrish.csv};
				\addplot+[
					color=pblue,
					no marks,
					thick,
				]
				table[x={l_arginine},y={aw_calc},col sep=comma]
					{ninni_arginine_virial.csv};
				\addplot+[
					color=pverydarkblue,
					no marks,
					thick,
				]
				table[x={l_arginine},y={aw_calc},col sep=comma]
					{ninni_arginine_uniquac.csv};
			\end{axis}
		\end{tikzpicture}
		}}
		\begin{axis}[
				xmin = 0.002, xmax = 0.015,
				xlabel = {$x_\text{arginina}$},
				ymin = 0.62, ymax = 0.85, ylabel = {$\phi$},
				legend pos = north west,
				xlabel near ticks,
				ylabel near ticks,
				xticklabel style={
					/pgf/number format/fixed,
					/pgf/number format/precision=3,
					/pgf/number format/fixed zerofill
				},
				scaled x ticks=false,
				xtick = {0.003,0.006,0.009,0.012},
			]
			\addplot+[
				smooth,
				color=pverybrightblue,
				no marks,
				thick,
			]
			table[x={l_arginine},y={phi_calc},col sep=comma]
				{ninni_arginine_norrish.csv};
			\addlegendentry{Norrish};
			\addplot+[
				smooth,
				color=pblue,
				no marks,
				thick,
			]
			table[x={l_arginine},y={phi_calc},col sep=comma]
				{ninni_arginine_virial.csv};
			\addlegendentry{virial};
			\addplot+[
				smooth,
				color=pverydarkblue,
				no marks,
				thick,
			]
			table[x={l_arginine},y={phi_calc},col sep=comma]
				{ninni_arginine_uniquac.csv};
			\addlegendentry{UNIQUAC};
			\addplot+[
				color=pbrightred,
				mark=o,
				very thick,
				only marks,
			]
			table[x={l_arginine},y={phi_exp},col sep=comma]
				{ninni_arginine_uniquac.csv};
			\addlegendentry{experimental};
			\draw (axis cs:0.0055,0.68) node{\usebox{\minigrafarg}};
			\end{axis}
		\end{tikzpicture}
		\caption{Arginina -- exemplo de aminoácido}
		\label{fig_arginine_phi}
	\end{subfigure}%
	\hfill%
	\begin{subfigure}{0.5\textwidth}%
		\centering
		\begin{tikzpicture}[scale=0.75]
			\newsavebox{\minigrafxyl}
			\savebox{\minigrafxyl}{%
			\scalebox{0.5}{%
			\begin{tikzpicture}
				\begin{axis}[
				xmin=0.0,xmax=0.061,
				ymin=0.94,ymax=1.0,
				width=7cm,
				height=7cm,
				xlabel = {$x$},
				ylabel = {$a_w$},
				xlabel near ticks,
				ylabel near ticks,
				xticklabel style={
					/pgf/number format/fixed,
					/pgf/number format/precision=3,
					/pgf/number format/fixed zerofill
				},
				scaled x ticks=false,
				yticklabel style={
					/pgf/number format/fixed,
					/pgf/number format/precision=3,
					/pgf/number format/fixed zerofill
				},
				scaled y ticks=false,
				legend pos=south west,
				]
				\addplot+[
					no marks,
					color=black,
					samples=2,
					domain=-0.001:0.061,
				] plot {1-x};
				\addlegendentry{$a_w=x_w$};
				\addplot+[
					color=pbrightred,
					mark=o,
					only marks,
					thick,
				]
				table[x={xylose},y={aw_exp},col sep=comma]
					{./ebrahimi_xylose_norrish.csv};
				\addplot+[
					color=pverybrightblue,
					no marks,
					thick,
				]
				table[x={xylose},y={aw_calc},col sep=comma]
					{./ebrahimi_xylose_norrish.csv};
				\addplot+[
					color=pblue,
					no marks,
					thick,
				]
				table[x={xylose},y={aw_calc},col sep=comma]
					{./ebrahimi_xylose_virial.csv};
				\addplot+[
					color=pverydarkblue,
					no marks,
					thick,
				]
				table[x={xylose},y={aw_calc},col sep=comma]
					{./ebrahimi_xylose_uniquac.csv};
			\end{axis}
		\end{tikzpicture}
		}}
		\begin{axis}[
				xmin = 0.000, xmax = 0.061,
				xlabel = {$x_\text{xilose}$},
				ymin = 0.978, ymax = 1.02, ylabel = {$\phi$},
				legend pos = south east,
				xlabel near ticks,
				ylabel near ticks,
				xticklabel style={
					/pgf/number format/fixed,
					/pgf/number format/precision=3,
					/pgf/number format/fixed zerofill
				},
				scaled x ticks=false,
				yticklabel style={
					/pgf/number format/fixed,
					/pgf/number format/precision=3,
					/pgf/number format/fixed zerofill
				},
				scaled y ticks=false,
				xtick = {0.015,0.030,0.045,0.060},
			]
			\addplot+[
				smooth,
				color=pverybrightblue,
				no marks,
				thick,
			]
			table[x={xylose},y={phi_calc},col sep=comma]
				{ebrahimi_xylose_norrish.csv};
			\addlegendentry{Norrish};
			\addplot+[
				smooth,
				color=pblue,
				no marks,
				thick,
			]
			table[x={xylose},y={phi_calc},col sep=comma]
				{ebrahimi_xylose_virial.csv};
			\addlegendentry{virial};
			\addplot+[
				smooth,
				color=pverydarkblue,
				no marks,
				thick,
			]
			table[x={xylose},y={phi_calc},col sep=comma]
				{ebrahimi_xylose_uniquac.csv};
			\addlegendentry{UNIQUAC};
			\addplot+[
				color=pbrightred,
				mark=o,
				very thick,
				only marks,
			]
			table[x={xylose},y={phi_exp},col sep=comma]
				{ebrahimi_xylose_uniquac.csv};
			\addlegendentry{experimental};
			\draw (axis cs:0.017,1.008) node{\usebox{\minigrafxyl}};
			\end{axis}
		\end{tikzpicture}
		\caption{Xilose -- exemplo de carboidrato}
		\label{fig_xylose_phi}
	\end{subfigure}
	\caption{Exemplos de conjuntos de dados e ajustes dos modelos}
	\label{fig_tipi_datasets_xylose_arginine}
\end{figure}

Podemos ver um exemplo do comportamento dos dados (e do ajuste dos modelos) na
figura \ref{fig_tipi_datasets_xylose_arginine}
\footnote{\cite{ninni2001,ebrahimi2016}}: perceba que o maior desvio da idealidade
para a solução de aminoácido torna a Lei de Raoult muito inferior aos outros
modelos; além disso, $\phi$ é consistentemente menor que 1 nessas soluções, como
requerido pelo modelo de Norrish, matematicamente, mas não para as soluções de
carboidratos, exemplificando a diferença exibida na tabela. Outro ponto a ser
comentado é a semelhança entre os modelos UNIQUAC e virial; enquanto no primeiro
caso ambos se aproximam do caso em que $\phi$ e $x_w$ não estão correlacionados,
no segundo caso temos duas funções semelhantes para diluições menores que
$x_w=0.99$.

\begin{figure}[h]
	\centering
	\begin{subfigure}{0.5\textwidth}%
		\centering
		\begin{tikzpicture}[scale=0.75]
			\newsavebox{\minigrafglyc}
			\savebox{\minigrafglyc}{%
			\scalebox{0.5}{%
			\begin{tikzpicture}
				\begin{axis}[
				xmin=0.0,xmax=0.030,
				ymin=0.985,ymax=1.0,
				width=7cm,
				height=3.5cm,
				xlabel = {$x$},
				ylabel = {$a_w$},
				xlabel near ticks,
				ylabel near ticks,
				xticklabel style={
					/pgf/number format/fixed,
					/pgf/number format/precision=3,
					/pgf/number format/fixed zerofill
				},
				scaled x ticks=false,
				yticklabel style={
					/pgf/number format/fixed,
					/pgf/number format/precision=3,
					/pgf/number format/fixed zerofill
				},
				scaled y ticks=false,
				legend pos=south west,
				]
				\addplot+[
					no marks,
					color=black,
					samples=2,
					domain=-0.001:0.03,
				] plot {1-x};
				\addlegendentry{$a_w=x_w$};
				\addplot+[
					color=pbrightred,
					mark=o,
					only marks,
					thick,
				]
				table[x={glycylglycine},y={aw_exp},col sep=comma]
					{ellerton_glycylglycine_norrish.csv};
				\addplot+[
					color=pverybrightblue,
					no marks,
					thick,
				]
				table[x={glycylglycine},y={aw_calc},col sep=comma]
					{ellerton_glycylglycine_norrish.csv};
				\addplot+[
					color=pblue,
					no marks,
					thick,
				]
				table[x={glycylglycine},y={aw_calc},col sep=comma]
					{ellerton_glycylglycine_virial.csv};
				\addplot+[
					color=pverydarkblue,
					no marks,
					thick,
				]
				table[x={glycylglycine},y={aw_calc},col sep=comma]
					{ellerton_glycylglycine_uniquac.csv};
			\end{axis}
		\end{tikzpicture}
		}}
		\begin{axis}[
				xmin = 0.0, xmax = 0.03,
				xlabel = {$x_\text{glicilglicina}$},
				ymin = 0.42, ymax = 0.5, ylabel = {$\phi$},
				legend pos = north east,
				xlabel near ticks,
				ylabel near ticks,
				xticklabel style={
					/pgf/number format/fixed,
					/pgf/number format/precision=3,
					/pgf/number format/fixed zerofill
				},
				scaled x ticks=false,
				xtick = {0.006,0.012,0.018,0.024},
			]
			\addplot+[
				smooth,
				color=pverybrightblue,
				no marks,
				thick,
			]
			table[x={glycylglycine},y={phi_calc},col sep=comma]
				{ellerton_glycylglycine_norrish.csv};
			\addlegendentry{Norrish};
			\addplot+[
				smooth,
				color=pblue,
				no marks,
				thick,
			]
			table[x={glycylglycine},y={phi_calc},col sep=comma]
				{ellerton_glycylglycine_virial.csv};
			\addlegendentry{virial};
			\addplot+[
				color=pverydarkblue,
				no marks,
				thick,
			]
			table[x={glycylglycine},y={phi_calc},col sep=comma]
				{ellerton_glycylglycine_uniquac.csv};
			\addlegendentry{UNIQUAC};
			\addplot+[
				color=pbrightred,
				mark=o,
				very thick,
				only marks,
			]
			table[x={glycylglycine},y={phi_exp},col sep=comma]
				{ellerton_glycylglycine_uniquac.csv};
			\addlegendentry{experimental};
			\draw (axis cs:0.009,0.432) node{\usebox{\minigrafglyc}};
			\end{axis}
		\end{tikzpicture}
		\caption{Glicilglicina -- exemplo de aminoácido}
		\label{fig_glycylglycine_phi}
	\end{subfigure}%
	\hfill%
	\begin{subfigure}{0.5\textwidth}%
		\centering
		\begin{tikzpicture}[scale=0.75]
			\newsavebox{\minigrafsuc}
			\savebox{\minigrafsuc}{%
			\scalebox{0.5}{%
			\begin{tikzpicture}
				\begin{axis}[
				xmin=0.0,xmax=0.1,
				ymin=0.9,ymax=1.0,
				width=7cm,
				height=7cm,
				xlabel = {$x$},
				ylabel = {$a_w$},
				xlabel near ticks,
				ylabel near ticks,
				xticklabel style={
					/pgf/number format/fixed,
					/pgf/number format/precision=3,
					/pgf/number format/fixed zerofill
				},
				scaled x ticks=false,
				yticklabel style={
					/pgf/number format/fixed,
					/pgf/number format/precision=3,
					/pgf/number format/fixed zerofill
				},
				scaled y ticks=false,
				legend pos=south west,
				]
				\addplot+[
					no marks,
					color=black,
					samples=2,
					domain=-0.001:0.1,
				] plot {1-x};
				\addlegendentry{$a_w=x_w$};
				\addplot+[
					color=pbrightred,
					mark=o,
					only marks,
					thick,
				]
				table[x={sucrose},y={aw_exp},col sep=comma]
					{./stokes_sucrose_uniquac_orig.csv};
				\addplot+[
					color=pverybrightblue,
					no marks,
					thick,
				]
				table[x={sucrose},y={aw_calc},col sep=comma]
					{./stokes_sucrose_norrish.csv};
				\addplot+[
					color=pblue,
					no marks,
					thick,
				]
				table[x={sucrose},y={aw_calc},col sep=comma]
					{./stokes_sucrose_virial.csv};
				\addplot+[
					color=pverydarkblue,
					no marks,
					thick,
				]
				table[x={sucrose},y={aw_calc},col sep=comma]
					{./stokes_sucrose_uniquac.csv};
			\end{axis}
		\end{tikzpicture}
		}}
		\begin{axis}[
				xmin = 0.000, xmax = 0.1,
				xlabel = {$x_\text{sacarose}$},
				ymin = 0.45, ymax = 1.0, ylabel = {$\phi$},
				legend pos = south east,
				xlabel near ticks,
				ylabel near ticks,
				xticklabel style={
					/pgf/number format/fixed,
					/pgf/number format/precision=3,
					/pgf/number format/fixed zerofill
				},
				scaled x ticks=false,
				yticklabel style={
					/pgf/number format/fixed,
					/pgf/number format/precision=3,
					/pgf/number format/fixed zerofill
				},
				scaled y ticks=false,
				xtick = {0.016,0.032,0.048,0.064,0.080},
			]
			\addplot+[
				smooth,
				color=pverybrightblue,
				no marks,
				thick,
			]
			table[x={sucrose},y={phi_calc},col sep=comma]
				{stokes_sucrose_norrish.csv};
			\addlegendentry{Norrish};
			\addplot+[
				smooth,
				color=pblue,
				no marks,
				thick,
			]
			table[x={sucrose},y={phi_calc},col sep=comma]
				{stokes_sucrose_virial.csv};
			\addlegendentry{virial};
			\addplot+[
				smooth,
				color=pverydarkblue,
				no marks,
				thick,
			]
			table[x={sucrose},y={phi_calc},col sep=comma]
				{stokes_sucrose_uniquac.csv};
			\addlegendentry{UNIQUAC};
			\addplot+[
				color=pbrightred,
				mark=o,
				very thick,
				only marks,
			]
			table[x={sucrose},y={phi_exp},col sep=comma]
				{stokes_sucrose_uniquac_orig.csv};
			\addlegendentry{experimental};
			\draw (axis cs:0.07,0.870) node{\usebox{\minigrafsuc}};
			\end{axis}
		\end{tikzpicture}
		\caption{Sacarose -- exemplo de carboidrato}
		\label{fig_sucrose_phi}
	\end{subfigure}
	\caption{Exemplos de conjuntos de dados e ajustes dos modelos}
	\label{fig_tipi_datasets_sucrose_glycylglycine}
\end{figure}

Podemos observar na figura \ref{fig_tipi_datasets_sucrose_glycylglycine}
\footnote{\cite{ellerton1964,stokes1961}}
outro motivo para as grandes diferenças observadas entre o modelo de Norrish no
ajuste de aminoácidos e carboidratos; de forma geral, temos que conforme a diluição
diminui, o coeficiente osmótico tende a aumentar, para uma solução de carboidratos,
a partir de determinado valor de $x_w$, o que não é tão comum em soluções de
aminoácidos; assim, sendo, o modelo de Norrish, que não apresenta meio de lidar com
valores negativos de $\frac{\partial \phi}{\partial x_w}$.

\section{Influência da diluição}

Observando os dados como um todo, podemos observar que a diluição parece não
afetar de forma significativa o valor dos desvios, para os modelos avaliados.
Isso pode ser visto no gráfico exibido na figura \ref{fig_dilution_all}.

\begin{figure}[h]
	\centering
	\begin{tikzpicture}
		\begin{semilogyaxis} [
			ylabel={$\sqrt{\frac{1}{n}\sum_{i=1}^N(\phi_{\text{exp}}-%
			\phi_{\text{calc}})^2}$},
			xlabel={$x_w$},
			legend pos=south west,
			xmin=0.85,
			xmax=1.0,
		]
			\addplot [
				mark=*,
				color=black,
				fill=pred,
				only marks,
				mark size=3pt,
			]
			table [x={xw},y={raoult},col sep=comma]
				{dilution_mono.csv};
			\addlegendentry{Raoult};
			\addplot [
				mark=*,
				color=black,
				fill=pverybrightblue,
				only marks,
				mark size=3pt,
			]
			table [x={xw},y={norrish},col sep=comma]
				{dilution_mono.csv};
			\addlegendentry{Norrish};
			\addplot [
				mark=*,
				color=black,
				fill=pblue,
				only marks,
				mark size=3pt,
			]
			table [x={xw},y={virial},col sep=comma]
				{dilution_mono.csv};
			\addlegendentry{Virial};
			\addplot [
				mark=*,
				color=black,
				fill=pverydarkblue,
				only marks,
				mark size=3pt,
			]
			table [x={xw},y={uniquac},col sep=comma]
				{dilution_mono.csv};
			\addlegendentry{UNIQUAC};
		\end{semilogyaxis}
	\end{tikzpicture}
	\caption{Função objetivo em função da diluição (sistemas binários)}
	\label{fig_dilution_all}
\end{figure}


\section{Influência do Método de Obtenção}

Como visto na seção \ref{sec_selec_data}, os dados foram obtidos através de
diversos métodos, como medidas de pressão de vapor, depressão do ponto de fusão,
elevação do ponto de ebulição, entre outros. Dentre conjuntos de dados obtidos
a partir de substâncias similares, podemos comparar os desvios obtidos entre os
dados e os modelos para diferentes procedimentos experimentais.

\subsection{Aminoácidos}

\label{sec_metod_amin}

Foram comparados dados obtidos a partir de medições de coeficiente osmótico,
medidas diretas de atividade da água, depressão do ponto de fusão e dados de
pressão de vapor, de acordo com as fontes delineadas na seção \ref{sec_selec_data}.
A única diferença estatisticamente significante observada foi entre os desvios
obtidos através do modelo UNIQUAC para dados obtidos a partir de coeficiente
osmótico e para dados obtidos através de depressão de ponto de fusão.

Entretanto, como os experimentos que compõem esse último grupo geraram poucos
dados experimentais, e como tem todos a mesma fonte, a explicação mais provável
é a simples ocorrência de \textit{overfitting} ao utilizar nesses dados o modelo
UNIQUAC, tornando os desvios excepcionalmente pequenos.

\subsection{Carboidratos}

Foram analisados os desvios dos ajustes para dados obtidos a partir de medições
diretas de $a_w$, medições de coeficientes osmóticos, depressões de ponto de fusão,
elevações de ponto de ebulição e umidade relativa. Para manter a coerência com a
seção \ref{sec_metod_amin}, foram avaliados apenas ajustes sobre dados
provenientes de sistemas binários.

Existe diferença estatisticamente significante entre os desvios obtidos a partir
de dados de elevação do ponto de ebulição, de coeficientes osmóticos, e do resto.
De forma similar a anterior, entretanto, a explicação mais provável para essa
diferença são as diferenças entre os estudos: entre os dados de elevação de ponto
de ebulição, por exemplo, se incluem os obtidos, por Abderafi \& Bounahmidi
\cite{abderafi1994}, que, por terem sido obtidos em condições muito diferentes
das condições nos quais foram obtidos os outros dados, apresentam algumas
peculiaridades\footnote{%
	\textit{E.g.} a diferença na temperatura dos experimentos.
}.


\section{Influência da Temperatura}

Para avaliar a influência da temperatura, avaliamos separadamente dados
provenientes de estudos feitos a diferentes temperaturas, tanto para carboidratos
\cite{velezmoro2000} quanto para aminoácidos \cite{romero2006,tsurko2007}.

Com isso, pode ser observado que a influência da temperatura sobre a função
objetivo não é universal. Ocorrendo apenas em situações específicas, a
depender de outros fatores como natureza química dos solutos ou diluição.

\subsection{Aminoácidos}

Foram analisados dois estudos: um comparando três aminoácidos (ácido glutâmico,
glicina e histidina) às temperaturas de 298.15K e 310.15K \cite{tsurko2007}, e
outro comparando três aminoácidos (alanina, ácido $\alpha$-aminobutírico e glicina)
às temperaturas de 288.15K, 293.15K, 298.15K e 303.15K \cite{romero2006}.

Analisando os dados obtidos em cada um, não foi encontrada diferença significativa
entre os desvios a uma temperatura e a outra, para modelo algum, tanto para os
dados do primeiro conjunto de dados, quanto para os dados do segundo.

Isso se explica já que não temos, como no caso a seguir, valores de $\phi$ maiores
que 1; assim sendo, para os valores de temperatura em questão é possível obter
um ajuste, para os modelos avaliados; além disso, o comportamento não se modifica
de forma qualitativa (como no caso anterior, que passa a exibir pontos de mínimo
perceptíveis para maiores temperaturas).

\begin{figure}[h]
	\centering
	\begin{tikzpicture}
		\begin{axis} [
			ylabel={$\phi$},
			xlabel={$x_\text{glicina}$},
			xtick={0.002,0.006,0.010,0.014,0.018,0.022},
			xlabel near ticks,
			ylabel near ticks,
			xticklabel style={
				/pgf/number format/fixed,
				/pgf/number format/precision=3,
				/pgf/number format/fixed zerofill
			},
			scaled x ticks=false,
		]
			\addplot [
				mark=*,
				color=black,
				fill=pverybrightred,
				only marks,
				mark size=3pt,
			]
			table [x={glycine},y={phi_exp},col sep=comma]
				{./romero_glycine_288_15_K.csv};
			\addlegendentry{288.15K};
			\addplot [
				mark=*,
				color=black,
				fill=pred,
				only marks,
				mark size=3pt,
			]
			table [x={glycine},y={phi_exp},col sep=comma]
				{./romero_glycine_293_15_K.csv};
			\addlegendentry{293.15K};
			\addplot [
				mark=*,
				color=black,
				fill=pverydarkred,
				only marks,
				mark size=3pt,
			]
			table [x={glycine},y={phi_exp},col sep=comma]
				{./romero_glycine_298_15_K.csv};
			\addlegendentry{288.15K};
			\addplot [
				mark=*,
				color=black,
				fill=black,
				only marks,
				mark size=3pt,
			]
			table [x={glycine},y={phi_exp},col sep=comma]
				{./romero_glycine_303_15_K.csv};
			\addlegendentry{303.15K};
		\end{axis}
	\end{tikzpicture}
	\caption{Valores experimentais obtidos para $\phi$ em uma solução de glicina}
	\label{fig_temp_amins}
\end{figure}

\subsection{Carboidratos}

Avaliando dados de $a_w$ em função de $(x_\text{frutose},x_\text{glicose},%
x_\text{sacaraose},x_\text{maltose})$ \cite{velezmoro2000}, não foi possível
afirmar que as pequenas diferenças observadas entre os desvios para
$T=25,30,35$ \textcelsius\ são estatisticamente significantes, para
os modelos virial e UNIQUAC.

Entretanto, o modelo de Norrish e a lei de Raoult apresentam desvios
estatisticamente significantes diferentes entre medidas obtidas a
25\textcelsius\ ou 30\textcelsius\ e medidas obtidas a 35\textcelsius.

Além disso, os desvios para Norrish e Raoult foram muito similares; sendo
assim, podemos concluir que para esse conjunto de dados, temos, de forma
geral, $\phi>1$; além disso, o crescimento de $T$ deve levar a um aumento
no valor de $\phi$, o que torna os modelos de Norrish e Raoult ainda menos
adequados. Isso pode ser visto na figura \ref{fig_temp_carbs}\footnote{%
\cite{velezmoro2000}}.

\begin{figure}[h]
	\centering
	\begin{tikzpicture}
		\begin{axis} [
			ylabel={$\phi$},
			xlabel={$x_\text{maltose}$},
			xtick={0.004,0.008,0.012,0.016,0.020},
			xlabel near ticks,
			ylabel near ticks,
			xticklabel style={
				/pgf/number format/fixed,
				/pgf/number format/precision=3,
				/pgf/number format/fixed zerofill
			},
			scaled x ticks=false,
			xmin=0.0000,
		]
			\draw[black,thick] (axis cs:0.0000,1) -- (axis cs:0.022,1);
			\draw (axis cs: 0.016,0.9)
				node[anchor=north] {\tiny{Lei de Raoult}};
			\draw (axis cs: 0.0135,0.9) -- (axis cs:0.0185,0.9);
			\draw (axis cs: 0.0125,1.0) -- (axis cs: 0.0135,0.9);
			\draw (axis cs: 0.0128,1.0) -- (axis cs: 0.0138,0.9);
			\draw (axis cs: 0.0131,1.0) -- (axis cs: 0.0141,0.9);
			\fill [
				color=lightgray,
				opacity=0.5,
			]
				(axis cs: 0.0000,1.0) rectangle (axis cs: 0.022,2.8);
			\draw (rel axis cs: 0.5,0.75)
				node[anchor=south]
				{\tiny{Região inacessível}};
			\draw (rel axis cs: 0.5,0.75)
				node[anchor=north]
				{\tiny{ao modelo de Norrish}};
			\addplot [
				mark=*,
				color=black,
				fill=pverybrightred,
				only marks,
				mark size=3pt,
			]
			table [x={maltose},y={phi_exp},col sep=comma]
				{./velezmoro_maltose_25.csv};
			\addlegendentry{25\textcelsius};
			\addplot [
				mark=*,
				color=black,
				fill=pred,
				only marks,
				mark size=3pt,
			]
			table [x={maltose},y={phi_exp},col sep=comma]
				{./velezmoro_maltose_30.csv};
			\addlegendentry{30\textcelsius};
			\addplot [
				mark=*,
				color=black,
				fill=pverydarkred,
				only marks,
				mark size=3pt,
			]
			table [x={maltose},y={phi_exp},col sep=comma]
				{./velezmoro_maltose_35.csv};
			\addlegendentry{35\textcelsius};
		\end{axis}
	\end{tikzpicture}
	\caption{Valores experimentais obtidos para $\phi$ em uma solução de maltose}
	\label{fig_temp_carbs}
\end{figure}

\section{Influência da Pressão}

Os modelos de Norrish e virial foram ajustados aos dados de $a_w$
\footnote{\cite{maximo2010}} em função da composição à diferentes pressões,
para soluções de frutose e glicose. Como a quantidade de dados para cada
valor de $P$ era pequena (5 pontos), um ajuste do modelo UNIQUAC não resultaria
em dados interessantes.

Foi observado que o modelo de Norrish manteve a tendência de simplesmente se
aproximar da Lei de Raoult, como visto para soluções de carboidratos. Assim sendo,
estão comparados apenas os desvios para a Lei de Raoult e para o modelo virial,
nos gráficos exibidos na figura \ref{fig_pressure}.

Podemos observar que apesar de existir (para dados obtidos a partir de soluções
de frutose) uma leve tendência à diminuição dos desvios com o aumento da pressão,
tal tendência não é estatisticamente significante.

\begin{figure}[h]
	\centering
	\begin{subfigure}{0.5\textwidth}{%
		\begin{tikzpicture}[scale=0.75]
			\begin{axis} [
				ylabel={$\sqrt{\frac{1}{n}\sum_{i=1}^N(%
					\phi_{\text{exp}}-%
				\phi_{\text{calc}})^2}$},
				xlabel={Pressão (kPa)},
				xlabel near ticks,
				ylabel near ticks,
				scaled x ticks=false,
				ymin=0.3,ymax=1.5,
			]
				\addplot [
					mark=*,
					color=black,
					fill=pblue,
					only marks,
					mark size=3pt,
				]
				table [x={pressure},y={virial},col sep=comma]
					{./pressure_glucose.csv};
				\addlegendentry{Modelo Virial};
				\addplot [
					no marks,
					color=pblue,
					samples=3,
					domain=20:100,
					ultra thick,
				] plot {-0.000585*x+0.7056};
				\addlegendentry{$R^2=0.0065$};
				\addplot [
					mark=*,
					color=black,
					fill=pred,
					only marks,
					mark size=3pt,
				]
				table [x={pressure},y={raoult},col sep=comma]
					{./pressure_glucose.csv};
				\addlegendentry{Lei de Raoult};
				\addplot [
					no marks,
					color=pred,
					samples=3,
					domain=20:100,
					ultra thick,
				] plot {-0.001272*x+1.0666};
				\addlegendentry{$R^2=0.0204$};
			\end{axis}
		\end{tikzpicture}}
		\caption{Glicose}
		\label{fig_pressure_glic}
	\end{subfigure}%
	\hfill%
	\begin{subfigure}{0.5\textwidth}{%
		\begin{tikzpicture}[scale=0.75]
			\begin{axis} [
				ylabel={$\sqrt{\frac{1}{n}\sum_{i=1}^N(%
					\phi_{\text{exp}}-%
				\phi_{\text{calc}})^2}$},
				xlabel={Pressão (kPa)},
				xlabel near ticks,
				ylabel near ticks,
				scaled x ticks=false,
				ymin=0.5,ymax=1.8,
			]
				\addplot [
					mark=*,
					color=black,
					fill=pblue,
					only marks,
					mark size=3pt,
				]
				table [x={pressure},y={virial},col sep=comma]
					{./pressure_fructose.csv};
				\addlegendentry{Modelo Virial};
				\addplot [
					no marks,
					color=pblue,
					samples=3,
					domain=20:100,
					ultra thick,
				] plot {-0.00355*x+1.0137};
				\addlegendentry{$R^2=0.3086$};
				\addplot [
					mark=*,
					color=black,
					fill=pred,
					only marks,
					mark size=3pt,
				]
				table [x={pressure},y={raoult},col sep=comma]
					{./pressure_fructose.csv};
				\addlegendentry{Lei de Raoult};
				\addplot [
					no marks,
					color=pred,
					samples=3,
					domain=20:100,
					ultra thick,
				] plot {-0.005321*x+1.5911};
				\addlegendentry{$R^2=0.4052$};
			\end{axis}
		\end{tikzpicture}}
		\caption{Frutose}
		\label{fig_pressure_fruc}
	\end{subfigure}
	\caption{Desvios obtidos para $\phi$ sob diferentes pressões}
	\label{fig_pressure}
\end{figure}

