\chapter{Model Analysis}

\section{Binary and $n$-ary solutions}

Scaled cost functions were compared for model fitting over data from binary and
$n$-ary solutions. When reasonable (a some ternary solutions), Zdanovskii's
relation was also used.

There are 823 data points from data sets obtained from binary solutions only,
235 from data sets from binary and ternary solutions and 213 for binary, ternary
and quaternary solutions. Each model was fitted over each data set, and the weighted
average scaled cost functions are available in table \ref{tab_comp_mono}.

The data, from which the values in the table were calculated, were obtained from
fitting only the models suitable to all data sets; therefore, Caurie's and
Zdanovskii's models are not in it, and must be analysed in another place, since
both exhibit problems for a few data series (such as non-physical results for
concentrated solutions\footnote{\cite{abderafi1994}} -- Caurie -- or necessity
of two auxiliary data series not always available -- Zdanovskii).

\begin{tabularx}{\textwidth}{ X r r r }
	\caption{Model performance for binary, ternary and quaternary solutions}
	\label{tab_comp_mono}\\
	\toprule
	\multirow{2}*{Model} & \multicolumn{3}{c}%
		{$\sqrt{\frac{1}{n}\sum_{i=1}^N(\phi_{\text{exp}}-%
		\phi_{\text{calc}})^2}_\text{average}$}\\
		& Binary systems & Ternary systems &%
			Quaternary systems \\
	\midrule
	\endfirsthead
	\toprule
	\multirow{2}*{Model} & \multicolumn{3}{c}%
		{$\sqrt{\frac{1}{n}\sum_{i=1}^N(\phi_{\text{exp}}-%
		\phi_{\text{calc}})^2}_\text{average}$}\\
		& Binary systems & Ternary systems &%
			Quaternary systems \\\hline
	\midrule
	\endhead
	\midrule
	\multicolumn{4}{r}{\footnotesize(Continue in the following page)}
	\endfoot
	\endlastfoot
	Raoult & 0.479665 & 0.591365 & 0.820056 \\
	Norrish & 0.448270 & 0.274971 & 0.654951 \\
	Virial & 0.247174 & 0.076962 & 0.310939 \\
	UNIQUAC & 0.197069 & 0.073781 & 0.550024 \\\hline
\end{tabularx}

\subsection{Zdanovskii's relation}

The model exhibited a bad behavior for the ternary data analysed; a comparison
with the results obtained fitting Raoult's, Norrish's, virial and UNIQUAC models
to the same data (to which Zdanovskii's model may be applied) is in table
\ref{tab_zdan_multi}. One may observe that the scaled cost function values are
greater for the model, even when compared with the cost functions obtained assuming
ideal behavior (Raoult's law). This is not necessarily indicative of a bad model,
however; as the number of data sets to which the relation may be applied is small
(only four ternary data sets), abnormal behavior of the model in the conditions
of one experiment may result in a huge increase in the cost function.

Indeed, from the 178 experimental data points to which the model could be applied,
103 (57.8\%) were obtained from data at very low dilutions
\footnote{\cite{abderafi1994}}, in which it behaves badly; this was not observed
for the remaining data set, and a comparison with the cost functions obtained fitting
the model to more usual data (as seen in table \ref{tab_mannitol_sucrose}) shows
that it may behave reasonably in certain conditions.

\begin{tabularx}{\textwidth}{ X  r }
	\caption{Comparison Zdanovskii's model}
	\label{tab_zdan_multi}\\
	\toprule
	Model & %
		$\sqrt{\frac{1}{n}\sum_{i=1}^N(\phi_{\text{exp}}-%
		\phi_{\text{calc}})^2}_\text{average}$\\
	\midrule
	\endfirsthead
	\toprule
	Model & %
		$\sqrt{\frac{1}{n}\sum_{i=1}^N(\phi_{\text{exp}}-%
		\phi_{\text{calc}})^2}_\text{average}$\\\hline
	\midrule
	\endhead
	\midrule
	\multicolumn{2}{r}{\footnotesize(Continue in the following page)}
	\endfoot
	\endlastfoot
	Raoult & 0.635824 \\
	Zdanovskii & 0.911042 \\
	Norrish & 0.295073 \\
	Virial & 0.087813 \\
	UNIQUAC & 0.076432 \\\hline
\end{tabularx}

\subsection{Caurie's model}

Caurie's model exhibits non-physical results when applied to a few sets of data, as
previously explained; excluding them and applying the models to the suitable data
sets, one can observe relatively small cost functions (when compared with Raoult's
model). The data are in table \ref{tab_caurie_multi}.

\begin{tabularx}{\textwidth}{ X  r }
	\caption{Comparison with Caurie's model}
	\label{tab_caurie_multi}\\
	\toprule
	Model & %
		$\sqrt{\frac{1}{n}\sum_{i=1}^N(\phi_{\text{exp}}-%
		\phi_{\text{calc}})^2}_\text{average}$\\
	\midrule
	\endfirsthead
	\toprule
	Model & %
		$\sqrt{\frac{1}{n}\sum_{i=1}^N(\phi_{\text{exp}}-%
		\phi_{\text{calc}})^2}_\text{average}$\\\hline
	\midrule
	\endhead
	\midrule
	\multicolumn{2}{r}{\footnotesize(Continue in the following page)}
	\endfoot
	\endlastfoot
	Raoult & 0.668330 \\
	Caurie & 0.654477 \\
	Norrish & 0.600739 \\
	Virial & 0.258034 \\
	UNIQUAC & 0.488533 \\\hline
\end{tabularx}

\section{Observations about the results obtained}

One can observe the smaller values of the scaled cost function obtained when
fitting virial and UNIQUAC models, to binary and $n$-ary data alike and that the
phenomenon is not explainable through the number of parameters used, only,
since Norrish's model requires the same number of adjustable parameters as the
virial model, exhibiting, however, significantly worse performance.

Besides, the (simplified) version of the UNIQUAC model is not consistently better
performant than the virial model, despite the former requiring a maximum of
six adjustable parameters (and usually only one) and the latter requiring a maximum
of eight (and no less than four).

Finally, one might notice that for a few sets of data\footnote{%
	\cite{norrish1966} data obtained from a sucrose and dextrose solution.
}, the size of the sample was insufficiently large to allow fitting the UNIQUAC
model; fortunately, no problems arise from this lack of data, since model analysis
for similar substances, data from the same source and similar experimental conditions
indicate no peculiarities in the data set. Unlike the situations saw applying
Caurie's model in some conditions, the cause of this problem lies only in the
amount of data available.

\section{Observations about the parameters obtained}

\subsection{Norrish's model}

As previously shown, Norrish's model exhibits difficulties when fitted to data
obtained from a system where $\phi$ might be greater than 1. In this situation,
the iterative procedure utilized will lead to values of $K_i$ as close as zero as
possible, maximizing the values of $\phi_\text{calc}$.

\subsection{UNIQUAC model}

The model exhibits numerous adjustable parameters, even for simple
solutions, which may lead to problems arising from overfitting. Besides, since
the properties that the model requires as argument are not linearly independent
(as the sum $x_\text{water} + \sum_{i\text{ solute}}x_i$ is fixed as 1), there
may be colinearity-related uncertainties in the obtained values for the adjustable
parameters. However, a certain degree of coherence was observed among some values;
while parameters as $q_\text{sucrose}$ assumed wildly different values when different
data were fitted, parameters as $u_\text{water}$ remained consistently close
to the mean (for $u_\text{water}$, around 1700J/mol). This behavior can be
seen in figure \ref{fig_violin_uniquac_u}.

\begin{figure}[h]
	\centering
	\begin{tikzpicture}
		\newsavebox{\violinglycineu}
		\savebox{\violinglycineu}{%
		\begin{tikzpicture}
			\begin{axis}[
				height=0.6\textwidth,
				width=0.8\textwidth,
				ymin=2,ymax=6,
				xmin=-5,xmax=2,
				axis line style={draw=none},
				tick style={draw=none},
				xticklabels={,,},
				yticklabels={,,},
			]
			\draw[pviolindarkred,thick]
				(axis cs:0,2) -- (axis cs:0,6);
			\addplot+[
				color=black,
				fill=pviolindarkred,
				no marks,
				thick,
				smooth,
			]
			table[x={yplus},y={x},col sep=comma]
				{./violin_glycine_uniquac.csv};
			\addplot+[
				color=black,
				fill=pviolindarkred,
				no marks,
				thick,
				smooth,
			]
			table[x={yminus},y={x},col sep=comma]
				{./violin_glycine_uniquac.csv};
			\addplot+[
				color=black,
				very thick,
				mark=o,
			] coordinates {(0.0,4.714)};
			\end{axis}
		\end{tikzpicture}
		}
		\newsavebox{\violinalanineu}
		\savebox{\violinalanineu}{%
		\begin{tikzpicture}
			\begin{axis}[
				height=0.6\textwidth,
				width=0.8\textwidth,
				ymin=2,ymax=6,
				xmin=-6,xmax=1,
				axis line style={draw=none},
				tick style={draw=none},
				xticklabels={,,},
				yticklabels={,,},
			]
			%\draw[pviolinbrightred,thin,opacity=0.3]
				%(axis cs:0,2) -- (axis cs:0,6);
			\addplot+[
				color=black,
				fill=pviolinbrightred,
				fill opacity=0.6,
				no marks,
				thick,
				smooth,
			]
			table[x={yplus},y={x},col sep=comma]
				{./violin_alanine_uniquac.csv};
			\addplot+[
				color=black,
				fill=pviolinbrightred,
				fill opacity=0.6,
				no marks,
				thick,
				smooth,
			]
			table[x={yminus},y={x},col sep=comma]
				{./violin_alanine_uniquac.csv};
			\addplot+[
				color=black,
				very thick,
				mark=o,
			] coordinates {(0.0,4.683)};
			\end{axis}
		\end{tikzpicture}
		}
		\newsavebox{\violinfructoseu}
		\savebox{\violinfructoseu}{%
		\begin{tikzpicture}
			\begin{axis}[
				height=0.6\textwidth,
				width=0.8\textwidth,
				ymin=2,ymax=6,
				xmin=-3,xmax=4,
				axis line style={draw=none},
				tick style={draw=none},
				xticklabels={,,},
				yticklabels={,,},
			]
			\draw[pviolinbrightblue,thick]
				(axis cs:0,2) -- (axis cs:0,6);
			\addplot+[
				color=black,
				fill=pviolinbrightblue,
				no marks,
				thick,
				smooth,
			]
			table[x={yplus},y={x},col sep=comma]
				{./violin_fructose_uniquac.csv};
			\addplot+[
				color=black,
				fill=pviolinbrightblue,
				no marks,
				thick,
				smooth,
			]
			table[x={yminus},y={x},col sep=comma]
				{./violin_fructose_uniquac.csv};
			\addplot+[
				color=black,
				very thick,
				mark=o,
			] coordinates {(0.0,4.198)};
			\end{axis}
		\end{tikzpicture}
		}
		\newsavebox{\violinsucroseu}
		\savebox{\violinsucroseu}{%
		\begin{tikzpicture}
			\begin{axis}[
				height=0.6\textwidth,
				width=0.8\textwidth,
				ymin=2,ymax=6,
				xmin=-2,xmax=5,
				axis line style={draw=none},
				tick style={draw=none},
				xticklabels={,,},
				yticklabels={,,},
			]
			\draw[pviolinblue,thick]
				(axis cs:0,2) -- (axis cs:0,6);
			\addplot+[
				color=black,
				fill=pviolinblue,
				no marks,
				thick,
				smooth,
			]
			table[x={yplus},y={x},col sep=comma]
				{./violin_sucrose_uniquac.csv};
			\addplot+[
				color=black,
				fill=pviolinblue,
				no marks,
				thick,
				smooth,
			]
			table[x={yminus},y={x},col sep=comma]
				{./violin_sucrose_uniquac.csv};
			\addplot+[
				color=black,
				very thick,
				mark=o,
			] coordinates {(0.0,4.604)};
			\end{axis}
		\end{tikzpicture}
		}
		\newsavebox{\violinglucoseu}
		\savebox{\violinglucoseu}{%
		\begin{tikzpicture}
			\begin{axis}[
				height=0.6\textwidth,
				width=0.8\textwidth,
				ymin=2,ymax=6,
				xmin=-1,xmax=6,
				axis line style={draw=none},
				tick style={draw=none},
				xticklabels={,,},
				yticklabels={,,},
			]
			\draw[pviolindarkblue,thick]
				(axis cs:0,2) -- (axis cs:0,6);
			\addplot+[
				color=black,
				fill=pviolindarkblue,
				no marks,
				thick,
				smooth,
			]
			table[x={yplus},y={x},col sep=comma]
				{./violin_glucose_uniquac.csv};
			\addplot+[
				color=black,
				fill=pviolindarkblue,
				no marks,
				thick,
				smooth,
			]
			table[x={yminus},y={x},col sep=comma]
				{./violin_glucose_uniquac.csv};
			\addplot+[
				color=black,
				very thick,
				mark=o,
			] coordinates {(0.0,4.714)};
			\end{axis}
		\end{tikzpicture}
		}
	\begin{axis}[
			xmin = 0.0, xmax = 1.4, xlabel = {Substance},
			ymin = 2, ymax = 6, ylabel = {$u_\text{substance}$},
			height=0.59\textwidth,
			width=0.8\textwidth,
			xtick style={draw=none},
			xtick={0.2,0.4,0.6,1.0,1.2},
			xticklabels={Glucose,Sucrose,Fructose,Glycine,Alanine},
			ytick={2,3,4,5,6},
			yticklabels={$10^2$,$10^3$,$10^4$,$10^5$,$10^6$},
		]
		\draw (axis cs:0.7,4) node {\usebox{\violinfructoseu}};
		\draw (axis cs:0.7,4) node {\usebox{\violinsucroseu}};
		\draw (axis cs:0.7,4) node {\usebox{\violinglucoseu}};
		\draw (axis cs:0.7,4) node {\usebox{\violinglycineu}};
		\draw (axis cs:0.7,4) node {\usebox{\violinalanineu}};
	\end{axis}
	\end{tikzpicture}
	\caption{Values obtained for $u$ for different substances}
	\label{fig_violin_uniquac_u}
\end{figure}

\subsection{Virial model}

Estimations for $b_i$ values were coherent, especially between data sets from the
same sources, and reasonable values were obtained for the adjustable parameters.
Examples for three carbohydrates are exhibited in figure
\ref{fig_violin_virial_carb}.

\begin{figure}[h]
	\centering
	\begin{tikzpicture}
		\newsavebox{\violinsucrose}
		\savebox{\violinsucrose}{%
		\begin{tikzpicture}
			\begin{axis}[
				height=0.6\textwidth,
				width=0.6\textwidth,
				ymin=-2,ymax=0,
				xmin=-4,xmax=12,
				axis line style={draw=none},
				tick style={draw=none},
				xticklabels={,,},
				yticklabels={,,},
			]
			\draw[pviolinblue,thick]
				(axis cs:0,0) -- (axis cs:0,-3);
			\addplot+[
				color=black,
				fill=pviolinbrightblue,
				no marks,
				thick,
				smooth,
			]
			table[x={yplus},y={x},col sep=comma]
				{./sucrose_violin.csv};
			\addplot+[
				color=black,
				fill=pviolinbrightblue,
				no marks,
				thick,
				smooth,
			]
			table[x={yminus},y={x},col sep=comma]
				{./sucrose_violin.csv};
			\addplot+[
				color=black,
				very thick,
				mark=o,
			] coordinates {(0.0,-1.065754)};
			\end{axis}
		\end{tikzpicture}
		}
		\newsavebox{\violinglucose}
		\savebox{\violinglucose}{%
		\begin{tikzpicture}
			\begin{axis}[
				height=0.6\textwidth,
				width=0.6\textwidth,
				ymin=-2,ymax=0,
				xmin=-8,xmax=8,
				axis line style={draw=none},
				tick style={draw=none},
				xticklabels={,,},
				yticklabels={,,},
			]
			\draw[pviolinblue,thick]
				(axis cs:0,0) -- (axis cs:0,-3);
			\addplot+[
				color=black,
				fill=pviolinblue,
				no marks,
				thick,
				smooth,
			]
			table[x={yplus},y={x},col sep=comma]
				{./glucose_violin.csv};
			\draw[pviolinblue,thick]
				(axis cs:0,0) -- (axis cs:0,-3);
			\addplot+[
				color=black,
				fill=pviolinblue,
				no marks,
				thick,
				smooth,
			]
			table[x={yminus},y={x},col sep=comma]
				{./glucose_violin.csv};
			\addplot+[
				color=black,
				very thick,
				mark=o,
			] coordinates {(0.0,-1.16826)};
			\end{axis}
		\end{tikzpicture}
		}
		\newsavebox{\violinfructose}
		\savebox{\violinfructose}{%
		\begin{tikzpicture}
			\begin{axis}[
				height=0.6\textwidth,
				width=0.6\textwidth,
				ymin=-2,ymax=0,
				xmin=-12,xmax=4,
				axis line style={draw=none},
				tick style={draw=none},
				xticklabels={,,},
				yticklabels={,,},
			]
			\draw[pviolindarkblue,thick]
				(axis cs:0,0) -- (axis cs:0,-3);
			\addplot+[
				color=black,
				fill=pviolindarkblue,
				no marks,
				thick,
				smooth,
			]
			table[x={yplus},y={x},col sep=comma]
				{./fructose_violin.csv};
			\addplot+[
				color=black,
				fill=pviolindarkblue,
				no marks,
				thick,
				smooth,
			]
			table[x={yminus},y={x},col sep=comma]
				{./fructose_violin.csv};
			\addplot+[
				color=black,
				very thick,
				mark=o,
			] coordinates {(0.0,-1.23545)};
			\end{axis}
		\end{tikzpicture}
		}
	\begin{axis}[
			xmin = 1.0, xmax = 3.0, xlabel = {Carbohydrate},
			ymin = -2, ymax = 0, ylabel = {$b_\text{carbohydrate}$},
			height=0.59\textwidth,
			width=0.6\textwidth,
			xtick style={draw=none},
			xtick={1.5,2,2.5},
			xticklabels={Sucrose,Glucose,Fructose},
		]
		\draw (axis cs:2,-1) node {\usebox{\violinsucrose}};
		\draw (axis cs:2,-1) node {\usebox{\violinglucose}};
		\draw (axis cs:2,-1) node {\usebox{\violinfructose}};
	\end{axis}
	\end{tikzpicture}
	\caption{Values obtained for $b$ for sucrose, glucose and fructose solutions}
	\label{fig_violin_virial_carb}
\end{figure}

However, one observes a different situation for the estimations of $c_{ij}$:
different data sources, even if internally coherent, result in wildly different
values when fitted. This is possibly associated with temperature or dilution
differences\footnote{%
	Indeed, the values estimated for $c_{ij}$ were greater for experiments
	under greater dilutions. \cite{abderafi1994,velezmoro2000}
} among the different experimental conditions. However, the impact of the
solute-solute interaction parameter is small, compared to the values of the
$b_i$, in the scaled cost function: a simplified virial model, assuming $c_{ij} = 0$
for all pairs of solutes $i$ and $j$ yields good water activity estimations, compared
to the other models; as one may observe in table \ref{tab_vir_simpl}, the smallest
values of the cost function are obtained through both virial models.

\begin{tabularx}{\textwidth}{ X  r }
	\caption{ Performance of the virial model simplification}
	\label{tab_vir_simpl}\\
	\toprule
	Model & %
		$\sqrt{\frac{1}{n}\sum_{i=1}^N(\phi_{\text{exp}}-%
		\phi_{\text{calc}})^2}_\text{average}$\\
	\midrule
	\endfirsthead
	\toprule
	Model & %
		$\sqrt{\frac{1}{n}\sum_{i=1}^N(\phi_{\text{exp}}-%
		\phi_{\text{calc}})^2}_\text{average}$\\\hline
	\midrule
	\endhead
	\midrule
	\multicolumn{2}{r}{\footnotesize(Continue in the following page)}
	\endfoot
	\endlastfoot
	Raoult's & 0.765376 \\
	Norrish's & 0.516392 \\
	UNIQUAC & 0.344529 \\
	Virial (simplified) & 0.243334 \\
	Virial (complete) & 0.215693 \\\hline
\end{tabularx}

\section{Predictive capabilities of the models}

One may be interested in the use of estimations of the adjustable fitting
parameters obtained over a data set to predict water activity values of
solutions of the same substances in similar conditions. To assess this possibility,
the ten largest data sets were split in two halves; to one (training subset) the
three models requiring regression were fitted, and the models' adjustable parameters
obtained were inserted in the model, that was then utilized to predict the values of
water activity associated with each listed composition of the other half of the data
(testing subset); with the predicted and real data, we obtained the scaled cost
functions, which were, then, compared.

For all models analysed, the models with the parameters obtained were a better
fit then the ideal solution assumption (Raoult's law); in almost all situations
the differences between the scaled cost functions of the application of the model
with the adjustable parameters obtained from regression over the testing data and
with adjustable parameters obtained from regression over the training data, over
the testing data, were one or more orders of magnitude inferior to the difference
between the scaled cost functions obtained when fitting the model to the testing
data and when applying Raoult's law.\footnote{The exception lies in data sets to
which Norrish's model converges to Raoult's, as previously explained.}

\begin{figure}[h]
	\centering
	\begin{tikzpicture}
		\begin{axis} [
				xlabel={Cost functions differences (ideal/training)},
				ylabel={Cost function differences (testing/training)},
				xticklabel style={
					/pgf/number format/fixed,
					/pgf/number format/precision=3,
					/pgf/number format/fixed zerofill
				},
				scaled x ticks=false,
				yticklabel style={
					/pgf/number format/fixed,
					/pgf/number format/precision=3,
					/pgf/number format/fixed zerofill
				},
				scaled y ticks=false,
				legend pos=north east,
				xmin=0,ymin=0,
				xmax=1.1,ymax=1.1,
			]
			\addplot [
				color=black,
				fill=pverybrightblue,
				only marks,
				mark size=3pt,
			] table [
				x={raoult_minus_train},
				y={test_minus_train},
				col sep=comma
			] {./test_and_train_diff_norrish.csv};
			\addlegendentry{Norrish's model};
			\addplot [
				color=black,
				fill=pblue,
				only marks,
				mark size=3pt,
			] table [
				x={raoult_minus_train},
				y={test_minus_train},
				col sep=comma
			] {./test_and_train_diff_virial.csv};
			\addlegendentry{Virial model};
			\addplot [
				color=black,
				fill=pverydarkblue,
				only marks,
				mark size=3pt,
			] table [
				x={raoult_minus_train},
				y={test_minus_train},
				col sep=comma
			] {./test_and_train_diff_uniquac.csv};
			\addlegendentry{UNIQUAC model};
			\fill[
				color=lightgray,
				opacity=0.5,
			]
				(axis cs:0,0) -- (axis cs:1.1,1.1)
				-- (axis cs:0,1.1) -- (axis cs:0,0);
			\draw (axis cs:0,0) -- (axis cs:1.1,1.1);
			\draw (axis cs:0.025,0.750)
				node[anchor=west,align=left]
					{Better fit\\with Raoult's law};
			\draw (axis cs:1.075,0.500)
				node[anchor=east,align=right]
					{Better fit with\\
					parameters obtained\\
					over training data set};
		\end{axis}
	\end{tikzpicture}
	\caption{Differences between testing and training data sets}
	\label{fig_test_train}
\end{figure}

One may observe this in figure \ref{fig_test_train}; for all data sets analysed
in this section, we plotted the differences between the values of
the scaled cost functions obtained from the model with adjustable parameters
from fitting over the training data and from fitting data as a function of the
differences between the scaled cost functions obtained from Raoult's law
application and from the application of the model with adjustable parameters
from the training data, over the testing data. It's easily seen that almost no points
fall within the dark area, in which Raoult's law is a better estimator for water
activity than the model generated from the parameters obtained from the training
data.

This is indicative that the parameters obtained for the models through the
regressions are good estimations for the real parameters, as desired, instead of
simply information about intrinsical properties of each data series, since, were
this true, a large discrepancy would be observed between the scaled cost functions
obtained from the application of the model with adjustable parameters from the
training data and from the application of the model fitted to the test data.


