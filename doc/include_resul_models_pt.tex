\chapter{Análise dos Modelos}

\section{Soluções binárias e sistemas multicomponente}

Foram comparados os ajustes sobre soluções binárias e $n$-árias de carboidratos.
Quando possível (soluções ternárias) se recorreu à Relação de Zdanovskii.

Estão disponíveis 823 dados para soluções binárias apenas, 235 dados para soluções
binárias e ternárias, apenas e 213 dados para soluções quaternárias. A cada
conjunto de dados foram ajustados cada um dos modelos, e os valores médios dos
desvios dos coeficientes osmóticos estão disponíveis na tabela \ref{tab_comp_mono},
ponderados pela quantidade de dados em cada série.

Nessa tabela, estão inclusos apenas os modelos que são aplicáveis a todas as séries
de dados; assim sendo, foi necessário excluir os modelos de Caurie (que apresentam
resultados não físicos para soluções com baixas diluições\footnote{%
	\cite{abderafi1994}
}, e também o modelo de Zdanovskii, que exige duas séries auxiliares, nem sempre
presentes, de dados sobre sistemas binários. Assim sendo, esses dois modelos devem
ser analisados separadamente.

\begin{tabularx}{\textwidth}{ X r r r }
	\caption{Performance dos modelos para sistemas binários, ternários %
		e quaternários}
	\label{tab_comp_mono}\\
	\toprule
	\multirow{2}*{Modelo} & \multicolumn{3}{c}%
		{$\sqrt{\frac{1}{n}\sum_{i=1}^N(\phi_{\text{exp}}-%
		\phi_{\text{calc}})^2}_\text{médio}$}\\
		& Sistemas Binários & Sistemas Ternários &%
			Sistemas Quaternários \\
	\midrule
	\endfirsthead
	\toprule
	\multirow{2}*{Modelo} & \multicolumn{3}{c}%
		{$\sqrt{\frac{1}{n}\sum_{i=1}^N(\phi_{\text{exp}}-%
		\phi_{\text{calc}})^2}_\text{médio}$}\\
		& Sistemas Binários & Sistemas Ternários &%
			Sistemas Quaternários \\\hline
	\midrule
	\endhead
	\midrule
	\multicolumn{4}{r}{\footnotesize(Continua na página seguinte)}
	\endfoot
	\endlastfoot
	Raoult & 0.479665 & 0.591365 & 0.820056 \\
	Norrish & 0.448270 & 0.274971 & 0.654951 \\
	Virial & 0.247174 & 0.076962 & 0.310939 \\
	UNIQUAC & 0.197069 & 0.073781 & 0.550024 \\\hline
\end{tabularx}

\subsection{Relação de Zdanovskii}

O comportamento desse modelo foi ruim para os dados ternários analisados; uma
comparação com os resultados obtidos com Raoult, Norrish, Virial e UNIQUAC está
na tabela \ref{tab_zdan_multi}, que compara os desvios quadrados médios dos
coeficientes osmóticos para os conjuntos de dados para os quais a Relação de
Zdanovskii pode ser aplicada. Perceba que os desvios obtidos com a Relação de
Zdanovskii são maiores que os obtidos assumindo solução ideal (Lei de Raoult),
por exemplo. Isso não indica necessariamente que o modelo seja ruim; como,
de todos os dados para soluções binárias disponíveis, apenas 4 conjuntos de
dados estavam acompanhados por séries de dados para modelos binários para cada
componente, comportamentos aberrantes do modelo para as condições de algum desses
experimentos podem ser responsabilizados pelo aumento dos desvios.

De fato, dos 178 pontos experimentais em questão, 103 (57.8\%) são provenientes de
dados obtidos a baixíssimas diluições\footnote{\cite{abderafi1994}}, e de fato, o
modelo de Zdanovskii é inferior justamente nesses dados; uma comparação com o ajuste
obtido para diluições mais baixas (visto na tabela \ref{tab_mannitol_sucrose}, por
exemplo) mostra que existem situações nas quais o modelo se porta de forma razoável.

\begin{tabularx}{\textwidth}{ X  r }
	\caption{Comparação com o modelo de Zdanovskii}
	\label{tab_zdan_multi}\\
	\toprule
	Modelo & %
		$\sqrt{\frac{1}{n}\sum_{i=1}^N(\phi_{\text{exp}}-%
		\phi_{\text{calc}})^2}_\text{médio}$\\
	\midrule
	\endfirsthead
	\toprule
	Modelo & %
		$\sqrt{\frac{1}{n}\sum_{i=1}^N(\phi_{\text{exp}}-%
		\phi_{\text{calc}})^2}_\text{médio}$\\\hline
	\midrule
	\endhead
	\midrule
	\multicolumn{2}{r}{\footnotesize(Continua na página seguinte)}
	\endfoot
	\endlastfoot
	Raoult & 0.635824 \\
	Zdanovskii & 0.911042 \\
	Norrish & 0.295073 \\
	Virial & 0.087813 \\
	UNIQUAC & 0.076432 \\\hline
\end{tabularx}

\subsection{Modelo de Caurie}

O modelo de Caurie retorna resultados não-físicos para alguns conjuntos de dados.
Excluindo esses conjuntos e comparando os resultados, percebemos que o modelo
apresenta resultados razoavelmente bons. Esses dados estão na tabela
\ref{tab_caurie_multi}; observe que os desvios para o modelo de Caurie são
ligeiramente menores que os desvios para a Lei de Raoult.

\begin{tabularx}{\textwidth}{ X  r }
	\caption{Comparação com o modelo de Caurie}
	\label{tab_caurie_multi}\\
	\toprule
	Modelo & %
		$\sqrt{\frac{1}{n}\sum_{i=1}^N(\phi_{\text{exp}}-%
		\phi_{\text{calc}})^2}_\text{médio}$\\
	\midrule
	\endfirsthead
	\toprule
	Modelo & %
		$\sqrt{\frac{1}{n}\sum_{i=1}^N(\phi_{\text{exp}}-%
		\phi_{\text{calc}})^2}_\text{médio}$\\\hline
	\midrule
	\endhead
	\midrule
	\multicolumn{2}{r}{\footnotesize(Continua na página seguinte)}
	\endfoot
	\endlastfoot
	Raoult & 0.668330 \\
	Caurie & 0.654477 \\
	Norrish & 0.600739 \\
	Virial & 0.258034 \\
	UNIQUAC & 0.488533 \\\hline
\end{tabularx}

\section{Observações sobre os resultados}

Podemos observar que os modelos virial e UNIQUAC são, tanto para sistemas
binários quanto para sistemas multicomponente, os melhores em termos de
reduzir o valor da função objetivo, e que isso se dá não somente devido ao maior
número de parâmetros, já que, para sistemas binários tanto o modelo de Norrish
quanto o modelo virial apresentam apenas um parâmetro, apesar de apresentarem desvios
claramente diferentes.

Além disso, a versão (simplificada) do modelo UNIQUAC (que chega a contar com até 8
parâmetros) não apresenta valores consistentemente menores da função objetivo que o
modelo virial (que apresenta no máximo 6 parâmetros).

Por fim, vale ressaltar que, para alguns conjuntos de dados\footnote{%
	\cite{norrish1966}, dados para uma mistura de sacarose e dextrose.
}, a quantidade de pontos foi insuficiente para ajuste via modelo UNIQUAC;
entretanto, isso não é um grande problema, já que análises do modelo para
dados da mesma fonte não apresentaram peculiaridades. Ao contrário das situações
vistas com Caurie, isso foi decorrente da simples quantidade de dados.

\section{Valores Ajustados \textendash\ Observações Gerais}

\subsection{Modelo de Norrish}

Como discutido anteriormente, o modelo de Norrish apresenta dificuldades para
lidar com sistemas que apresentem $\phi > 1$. Nesse caso, o processo iterativo
deixará os parâmetros $K_i$ tão próximos quanto o possível de 0, de modo a
maximizar o valor de $\phi$.

\subsection{Modelo UNIQUAC}

O modelo UNIQUAC apresenta, mesmo para soluções binárias, um grande número
de parâmetros. Isso pode levar a problemas com \textit{overfitting}. Além disso,
existem problemas relativos a colinearidade, já que o conjunto de valores
possíveis $(x_1,x_2,\ldots,x_p)$ para a composição de uma mistura com $p$
componentes (incluindo a água) pertence a um subespaço de dimensão $p-1$
de $\mathbb{R}^p$. Entretanto, ainda foram observados valores coerentes
para alguns parâmetros, analisando diversos experimentos; por exemplo,
$u_\text{água}$ se manteve consistentemente em torno de 1700J/mol, enquanto
outras substâncias como a sacarose apresentaram grandes variações. Isso
pode ser visto claramente na figura \ref{fig_violin_uniquac_u}.

\begin{figure}[h]
	\centering
	\begin{tikzpicture}
		\newsavebox{\violinglycineu}
		\savebox{\violinglycineu}{%
		\begin{tikzpicture}
			\begin{axis}[
				height=0.6\textwidth,
				width=0.8\textwidth,
				ymin=2,ymax=6,
				xmin=-5,xmax=2,
				axis line style={draw=none},
				tick style={draw=none},
				xticklabels={,,},
				yticklabels={,,},
			]
			\draw[pviolindarkred,thick]
				(axis cs:0,2) -- (axis cs:0,6);
			\addplot+[
				color=black,
				fill=pviolindarkred,
				no marks,
				thick,
				smooth,
			]
			table[x={yplus},y={x},col sep=comma]
				{./violin_glycine_uniquac.csv};
			\addplot+[
				color=black,
				fill=pviolindarkred,
				no marks,
				thick,
				smooth,
			]
			table[x={yminus},y={x},col sep=comma]
				{./violin_glycine_uniquac.csv};
			\addplot+[
				color=black,
				very thick,
				mark=o,
			] coordinates {(0.0,4.714)};
			\end{axis}
		\end{tikzpicture}
		}
		\newsavebox{\violinalanineu}
		\savebox{\violinalanineu}{%
		\begin{tikzpicture}
			\begin{axis}[
				height=0.6\textwidth,
				width=0.8\textwidth,
				ymin=2,ymax=6,
				xmin=-6,xmax=1,
				axis line style={draw=none},
				tick style={draw=none},
				xticklabels={,,},
				yticklabels={,,},
			]
			%\draw[pviolinbrightred,thin,opacity=0.3]
				%(axis cs:0,2) -- (axis cs:0,6);
			\addplot+[
				color=black,
				fill=pviolinbrightred,
				fill opacity=0.6,
				no marks,
				thick,
				smooth,
			]
			table[x={yplus},y={x},col sep=comma]
				{./violin_alanine_uniquac.csv};
			\addplot+[
				color=black,
				fill=pviolinbrightred,
				fill opacity=0.6,
				no marks,
				thick,
				smooth,
			]
			table[x={yminus},y={x},col sep=comma]
				{./violin_alanine_uniquac.csv};
			\addplot+[
				color=black,
				very thick,
				mark=o,
			] coordinates {(0.0,4.683)};
			\end{axis}
		\end{tikzpicture}
		}
		\newsavebox{\violinfructoseu}
		\savebox{\violinfructoseu}{%
		\begin{tikzpicture}
			\begin{axis}[
				height=0.6\textwidth,
				width=0.8\textwidth,
				ymin=2,ymax=6,
				xmin=-3,xmax=4,
				axis line style={draw=none},
				tick style={draw=none},
				xticklabels={,,},
				yticklabels={,,},
			]
			\draw[pviolinbrightblue,thick]
				(axis cs:0,2) -- (axis cs:0,6);
			\addplot+[
				color=black,
				fill=pviolinbrightblue,
				no marks,
				thick,
				smooth,
			]
			table[x={yplus},y={x},col sep=comma]
				{./violin_fructose_uniquac.csv};
			\addplot+[
				color=black,
				fill=pviolinbrightblue,
				no marks,
				thick,
				smooth,
			]
			table[x={yminus},y={x},col sep=comma]
				{./violin_fructose_uniquac.csv};
			\addplot+[
				color=black,
				very thick,
				mark=o,
			] coordinates {(0.0,4.198)};
			\end{axis}
		\end{tikzpicture}
		}
		\newsavebox{\violinsucroseu}
		\savebox{\violinsucroseu}{%
		\begin{tikzpicture}
			\begin{axis}[
				height=0.6\textwidth,
				width=0.8\textwidth,
				ymin=2,ymax=6,
				xmin=-2,xmax=5,
				axis line style={draw=none},
				tick style={draw=none},
				xticklabels={,,},
				yticklabels={,,},
			]
			\draw[pviolinblue,thick]
				(axis cs:0,2) -- (axis cs:0,6);
			\addplot+[
				color=black,
				fill=pviolinblue,
				no marks,
				thick,
				smooth,
			]
			table[x={yplus},y={x},col sep=comma]
				{./violin_sucrose_uniquac.csv};
			\addplot+[
				color=black,
				fill=pviolinblue,
				no marks,
				thick,
				smooth,
			]
			table[x={yminus},y={x},col sep=comma]
				{./violin_sucrose_uniquac.csv};
			\addplot+[
				color=black,
				very thick,
				mark=o,
			] coordinates {(0.0,4.604)};
			\end{axis}
		\end{tikzpicture}
		}
		\newsavebox{\violinglucoseu}
		\savebox{\violinglucoseu}{%
		\begin{tikzpicture}
			\begin{axis}[
				height=0.6\textwidth,
				width=0.8\textwidth,
				ymin=2,ymax=6,
				xmin=-1,xmax=6,
				axis line style={draw=none},
				tick style={draw=none},
				xticklabels={,,},
				yticklabels={,,},
			]
			\draw[pviolindarkblue,thick]
				(axis cs:0,2) -- (axis cs:0,6);
			\addplot+[
				color=black,
				fill=pviolindarkblue,
				no marks,
				thick,
				smooth,
			]
			table[x={yplus},y={x},col sep=comma]
				{./violin_glucose_uniquac.csv};
			\addplot+[
				color=black,
				fill=pviolindarkblue,
				no marks,
				thick,
				smooth,
			]
			table[x={yminus},y={x},col sep=comma]
				{./violin_glucose_uniquac.csv};
			\addplot+[
				color=black,
				very thick,
				mark=o,
			] coordinates {(0.0,4.714)};
			\end{axis}
		\end{tikzpicture}
		}
	\begin{axis}[
			xmin = 0.0, xmax = 1.4, xlabel = {Substância},
			ymin = 2, ymax = 6, ylabel = {$u_\text{substância}$},
			height=0.59\textwidth,
			width=0.8\textwidth,
			xtick style={draw=none},
			xtick={0.2,0.4,0.6,1.0,1.2},
			xticklabels={Glicose,Sacarose,Frutose,Glicina,Alanina},
			ytick={2,3,4,5,6},
			yticklabels={$10^2$,$10^3$,$10^4$,$10^5$,$10^6$},
		]
		\draw (axis cs:0.7,4) node {\usebox{\violinfructoseu}};
		\draw (axis cs:0.7,4) node {\usebox{\violinsucroseu}};
		\draw (axis cs:0.7,4) node {\usebox{\violinglucoseu}};
		\draw (axis cs:0.7,4) node {\usebox{\violinglycineu}};
		\draw (axis cs:0.7,4) node {\usebox{\violinalanineu}};
	\end{axis}
	\end{tikzpicture}
	\caption{Valores obtidos para o parâmetro $u$ para diversas substâncias}
	\label{fig_violin_uniquac_u}
\end{figure}


\subsection{Modelo Virial}

As estimativas para os valores de $b_i$ apresentaram boa coerência (especialmente
entre conjuntos de dados provenientes das mesmas fontes), e foi possível obter
valores razoáveis para o ajuste. Exemplos para três carboidratos estão na figura
\ref{fig_violin_virial_carb}.

\begin{figure}[h]
	\centering
	\begin{tikzpicture}
		\newsavebox{\violinsucrose}
		\savebox{\violinsucrose}{%
		\begin{tikzpicture}
			\begin{axis}[
				height=0.6\textwidth,
				width=0.6\textwidth,
				ymin=-2,ymax=0,
				xmin=-4,xmax=12,
				axis line style={draw=none},
				tick style={draw=none},
				xticklabels={,,},
				yticklabels={,,},
			]
			\draw[pviolinblue,thick]
				(axis cs:0,0) -- (axis cs:0,-3);
			\addplot+[
				color=black,
				fill=pviolinbrightblue,
				no marks,
				thick,
				smooth,
			]
			table[x={yplus},y={x},col sep=comma]
				{./sucrose_violin.csv};
			\addplot+[
				color=black,
				fill=pviolinbrightblue,
				no marks,
				thick,
				smooth,
			]
			table[x={yminus},y={x},col sep=comma]
				{./sucrose_violin.csv};
			\addplot+[
				color=black,
				very thick,
				mark=o,
			] coordinates {(0.0,-1.065754)};
			\end{axis}
		\end{tikzpicture}
		}
		\newsavebox{\violinglucose}
		\savebox{\violinglucose}{%
		\begin{tikzpicture}
			\begin{axis}[
				height=0.6\textwidth,
				width=0.6\textwidth,
				ymin=-2,ymax=0,
				xmin=-8,xmax=8,
				axis line style={draw=none},
				tick style={draw=none},
				xticklabels={,,},
				yticklabels={,,},
			]
			\draw[pviolinblue,thick]
				(axis cs:0,0) -- (axis cs:0,-3);
			\addplot+[
				color=black,
				fill=pviolinblue,
				no marks,
				thick,
				smooth,
			]
			table[x={yplus},y={x},col sep=comma]
				{./glucose_violin.csv};
			\draw[pviolinblue,thick]
				(axis cs:0,0) -- (axis cs:0,-3);
			\addplot+[
				color=black,
				fill=pviolinblue,
				no marks,
				thick,
				smooth,
			]
			table[x={yminus},y={x},col sep=comma]
				{./glucose_violin.csv};
			\addplot+[
				color=black,
				very thick,
				mark=o,
			] coordinates {(0.0,-1.16826)};
			\end{axis}
		\end{tikzpicture}
		}
		\newsavebox{\violinfructose}
		\savebox{\violinfructose}{%
		\begin{tikzpicture}
			\begin{axis}[
				height=0.6\textwidth,
				width=0.6\textwidth,
				ymin=-2,ymax=0,
				xmin=-12,xmax=4,
				axis line style={draw=none},
				tick style={draw=none},
				xticklabels={,,},
				yticklabels={,,},
			]
			\draw[pviolindarkblue,thick]
				(axis cs:0,0) -- (axis cs:0,-3);
			\addplot+[
				color=black,
				fill=pviolindarkblue,
				no marks,
				thick,
				smooth,
			]
			table[x={yplus},y={x},col sep=comma]
				{./fructose_violin.csv};
			\addplot+[
				color=black,
				fill=pviolindarkblue,
				no marks,
				thick,
				smooth,
			]
			table[x={yminus},y={x},col sep=comma]
				{./fructose_violin.csv};
			\addplot+[
				color=black,
				very thick,
				mark=o,
			] coordinates {(0.0,-1.23545)};
			\end{axis}
		\end{tikzpicture}
		}
	\begin{axis}[
			xmin = 1.0, xmax = 3.0, xlabel = {Carboidrato},
			ymin = -2, ymax = 0, ylabel = {$b_\text{carboidrato}$},
			height=0.59\textwidth,
			width=0.6\textwidth,
			xtick style={draw=none},
			xtick={1.5,2,2.5},
			xticklabels={Sacarose,Glicose,Frutose},
		]
		\draw (axis cs:2,-1) node {\usebox{\violinsucrose}};
		\draw (axis cs:2,-1) node {\usebox{\violinglucose}};
		\draw (axis cs:2,-1) node {\usebox{\violinfructose}};
	\end{axis}
	\end{tikzpicture}
	\caption{Valores obtidos para o parâmetro $b$ para sacarose, %
		glucose e frutose.}
	\label{fig_violin_virial_carb}
\end{figure}

Entretanto, para os valores de $c_i$, a situação se torna diferente: diferentes
estudos (apesar de coerentes entre si) apresentaram estimativas ordens de grandeza
diferentes. Isso possivelmente está associado a temperatura ou diluição (já que
diferentes experimentos\footnote{%
	De fato, os valores de $c$ foram maiores para experimentos realizados
	sob maiores diluições. \cite{abderafi1994,velezmoro2000}
}
apresentam diferentes temperaturas e diluições). De toda forma o valor dos
parâmetros de iteração soluto-soluto acaba tendo pouco impacto nos desvios, e
o uso do modelo, simplificado de forma a assumir nulos seus valores, é uma boa
aproximação, comparável com os outros modelos avaliados. Isso pode ser visto
na tabela \ref{tab_vir_simpl}: os dois modelos viriais são os que apresentam,
em média, menores valores para os desvios.

\begin{tabularx}{\textwidth}{ X  r }
	\caption{\textit{Performance} do modelo virial simplificado}
	\label{tab_vir_simpl}\\
	\toprule
	Modelo & %
		$\frac{1}{n}\sqrt{\sum_{i=1}^N(\phi_{\text{exp}}-%
		\phi_{\text{calc}})^2}_\text{médio}$\\
	\midrule
	\endfirsthead
	\toprule
	Modelo & %
		$\frac{1}{n}\sqrt{\sum_{i=1}^N(\phi_{\text{exp}}-%
		\phi_{\text{calc}})^2}_\text{médio}$\\\hline
	\midrule
	\endhead
	\midrule
	\multicolumn{2}{r}{\footnotesize(Continua na página seguinte)}
	\endfoot
	\endlastfoot
	Raoult & 0.765376 \\
	Norrish & 0.516392 \\
	UNIQUAC & 0.344529 \\
	Virial (simplificado) & 0.243334 \\
	Virial (completo) & 0.215693 \\\hline
\end{tabularx}


\section{Capacidade preditiva dos modelos}

É interessante analisarmos a aplicação de parâmetros obtidos sobre um conjunto de
dados a outro conjunto de dados em condições semelhantes; para isso, dentre as
séries de dados avaliadas, separamos cada uma das dez maiores (em número de pontos)
em duas séries, cada uma com metade dos dados da série inicial, de form aleatória;
em seguida, ajustamos o modelo a uma dessas séries (dados de treino) e utilizamos
os valores obtidos para os parâmetros para uma avaliação da outra (dados de teste).

Para todos os modelos avaliados, o uso dos parâmetros obtidos na série de treino
foi um melhor ajuste à série de teste que o modelo de Raoult; em quase todos os
casos a diferença das funções objetivo entre a aplicação do modelo aos dados de
treino e teste foi uma ou mais ordens de grandeza inferior à diferença entre a
aplicação do modelo em questão e a aplicação do modelo de Raoult, aos dados de
teste.\footnote{A exceção reside em conjuntos de dados para os quais o modelo
de Norrish acaba convergindo para o modelo de Raoult, como discutido
anteriormente.}.

\begin{figure}[h]
	\centering
	\begin{tikzpicture}
		\begin{axis} [
				xlabel={Diferença entre os desvios (ideal/treino)},
				ylabel={Diferença entre os desvios (teste/treino)},
				xticklabel style={
					/pgf/number format/fixed,
					/pgf/number format/precision=3,
					/pgf/number format/fixed zerofill
				},
				scaled x ticks=false,
				yticklabel style={
					/pgf/number format/fixed,
					/pgf/number format/precision=3,
					/pgf/number format/fixed zerofill
				},
				scaled y ticks=false,
				legend pos=north east,
				xmin=0,ymin=0,
				xmax=1.1,ymax=1.1,
			]
			\addplot [
				color=black,
				fill=pverybrightblue,
				only marks,
				mark size=3pt,
			] table [
				x={raoult_minus_train},
				y={test_minus_train},
				col sep=comma
			] {./test_and_train_diff_norrish.csv};
			\addlegendentry{Modelo de Norrish};
			\addplot [
				color=black,
				fill=pblue,
				only marks,
				mark size=3pt,
			] table [
				x={raoult_minus_train},
				y={test_minus_train},
				col sep=comma
			] {./test_and_train_diff_virial.csv};
			\addlegendentry{Modelo Virial};
			\addplot [
				color=black,
				fill=pverydarkblue,
				only marks,
				mark size=3pt,
			] table [
				x={raoult_minus_train},
				y={test_minus_train},
				col sep=comma
			] {./test_and_train_diff_uniquac.csv};
			\addlegendentry{Modelo UNIQUAC};
			\fill[
				color=lightgray,
				opacity=0.5,
			]
				(axis cs:0,0) -- (axis cs:1.1,1.1)
				-- (axis cs:0,1.1) -- (axis cs:0,0);
			\draw (axis cs:0,0) -- (axis cs:1.1,1.1);
			\draw (axis cs:0.025,0.750)
				node[anchor=west,align=left]
					{Melhor ajuste com o\\
					modelo de Raoult};
			\draw (axis cs:1.075,0.500)
				node[anchor=east,align=right]
					{Melhor ajuste\\
					com os parâmetros\\
					obtidos no treino};
		\end{axis}
	\end{tikzpicture}
	\caption{Diferenças entre dados de teste e de treino}
	\label{fig_test_train}
\end{figure}

Isso pode ser visto na figura \ref{fig_test_train}; temos que para todos os conjuntos
de dados avaliados nessa seção, fizemos um gráfico da diferença entre os desvios
($\sqrt{\frac{1}{n}\sum_{i=1}^N\left(\phi_\text{exp}-\phi_\text{calc}\right)^2}$),
obtidos para o ajuste do modelo e para a aplicação do modelo com parâmetros
provenientes dos dados de treino aos dados de teste, em função da diferença entre
os desvios obtidos aplicando a Lei de Raoult e aplicando o modelo com esses
parâmetros aos dados de teste. Observe que quase nenhum ponto está na zona
hachurada, na figura, na qual os modelos com parâmetros obtidos para dados de
treino seriam inferiores ao modelo de Raoult, no ajuste de dados de teste.

Isso indica que os parâmetros obtidos para os modelos através das regressões são
boas estimativas para os parâmetros reais, como desejado, e não apenas informações
sobre propriedades intrínsecas de cada série de dados sobre as quais se faz a
regressão, já que nesse caso seria observada uma grande discrepância nos desvios
ao aplicarmos o modelo com os parâmetros obtidos nos dados de treino aos dados de
teste.


