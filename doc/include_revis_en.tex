\part{Bibliographic Revision}

\chapter{Water activity modelling}

\section{Analysed models}

Several models were analysed, ranging in complexity from simple as Raoult's Law
or Caurie's equation to complex as the iterative procedure derived from the
Zdanovskii's relation or the UNIQUAC model, for subsequent comparison with a
virial equation (as in \ref{eq:ad_pessoa}, adapted from \cite{pessoa2008}, in
which $b_i$ and $c_{ij}$ are adjustable parameters and $\theta$ is a measure
of concentration).

\begin{equation}
	\label{eq:ad_pessoa}
	\ln(a_w) = \sum_{i \neq w}\theta_i\Bigg(1 +%
	2b_i+3\sum_{j \neq w}c_{ij}\theta_j\Bigg)
\end{equation}

For obtaining $a_w$ values in binary solutions, one has several distinct models
to choose from. From those models, the following shall be analysed and compared:

\begin{itemize}
	\item Raoult's Law: Simplest model, for ideal solutions. Assumes
		water activity $a_w = x_w$, its molar fraction;
	\item Norrish's Equation \cite{norrish1966}: Raoult's Law, corrected.
		One can, with the molar fraction of water, $x_w$, molar
		fraction of the other substance, $x_s$, and one adjustable
		parameter ($K$), utilize the equation \ref{eq_norrish} to
		obtain the water activity of a solution.
		\ref{eq_norrish}:
		\begin{equation}
			\label{eq_norrish}
			a_w = x_w\exp\Big[\Big(\sqrt{K}x_s\Big)^2\Big]
		\end{equation}
	\item UNIQUAC (\textit{universal quasi-chemical}) Model
		\cite{abrams1975}: Possibly the most complex model among the
		chosen ones; requiring two adjustable parameters for each pair
		of substances present in the mixture (between solvents and
		solutes), can be used for binary or $n$-ary systems. Even though
		the nonlinear fitting for the model presents a high computational
		cost, as we are dealing with simple mixtures (with a maximum of
		four components), the complexity of the model is not a problem.
\end{itemize}

For $n$-ary solutions, a few other models were chosen for analysis and
comparisons:

\begin{itemize}
	\item Raoult's Law: water activity $a_w$ equals $x_w$,
		its molar fraction;
	\item Norrish's Equation: adopted with a small modification to the equation
		\ref{eq_norrish}, as shown in the equation \ref{eq_norrish_multi};
		\begin{equation}
			\label{eq_norrish_multi}
			a_w = x_w\exp\Big[\Big(\sum_{i \neq w}%
			\sqrt{K_i}x_i\Big)^2\Big]
		\end{equation}
	\item Caurie's Model \cite{caurie1986}: water activity as a correction
		of the product of the water activities of binary solutions
		in which each component's molalities are either zero or equal
		to its molality in the solution of interest, as presented
		in the equation \ref{eq_caurie}, in which $a_{wi}$ is the
		water activity in a binary solution in which the molality $m_i$
		of the substance $i$ remains the same and $n$ is the total number
		of solutes in the mixture.
		\begin{equation}
			\label{eq_caurie}
			a_w = \prod_{i \neq w}a_{wi}-\left[\cfrac{n}%
			{55.5^2}\sum_{\substack{i \neq j \\ i,j \neq w}}%
			m_im_j + \cfrac{(n+1)}{55.5^3}\sum_{%
			\substack{i\neq j,k \\ j \neq k \\  i,j,k \neq w}}%
			m_im_jm_k\right]
		\end{equation}
		To obtain the water activities of the binary solutions,
		we assumed ideal solution (Raoult's Law).
	\item Zdanovskii's Relation \cite{chen1973,sangster1973}:
		applied for ternary mixtures only, it's an iterative process
		based in the Zdanovskii relation; for a ternary mixture with
		two generic solutes 1 and 2, we adopt two binary solutions, one
		with 1 as its only solute, the other with 2, in osmotic equilibrium
		with the solution of interest. With $m_{01}$ and $m_{02}$ as the
		molalities of each solute in each the binary solution, and
		$m_1$ and $m_2$ as the molalities of each solute in the solution
		of interest, the Zdanovskii's relation ($m_1m_{01}^{-1} +
		m_2m_{02}^{-1} = 1$) generates an iterative procedure that
		returns the water activity of a ternary mixture from the
		curves $a_w=a_w(x_w)$ of binary mixtures of each solute.
	\item UNIQUAC (\textit{universal quasi-chemical}) Model, previously
		cited.
\end{itemize}

\section{Applicability of the models to the samples}

Since one is not only interested in fitting curves to data sets, but also in
assessing the possibility of predicting water activities for $n$-ary systems
based on data from binary systems, one can split the available models in two
great groups. For a thermodynamical model (as shown in equation
\ref{eq:geral_bigphi}) with $N$ parameters $(A_j)_{j \in \{1, 2, \ldots N\}}$
for the osmotic coefficient in a system of composition given by the molar
fractions of each component $(x_i)_\text{$i$ component}$, at temperature $T$
and pressure $P$, one can obtain estimates $(A^*_j)_{j \in \{1, 2, \ldots N\}}$
for the parameters through the use of nonlinear least-squares fitting,
as shown in equation \ref{eq:min_quad}, for a set of $K > N$ experimental data
points
$\{(\phi^k, T^k, P^k, (x_i)^k_\text{$i$ component}), k \in \{1,2,\ldots,K\}\}$.

\begin{equation}
	\label{eq:geral_bigphi}
	\phi = \Phi((A_j)_{j \in \{1, 2, \ldots, N\}}, (x_i)_\text{$i$ %
		component of the mixture}, T, P)
\end{equation}

\begin{equation}
	\label{eq:min_quad}
	\minimize_{(A_j^*)_{j \in \{1,2,\ldots,N\}} \in \mathbb{R}^N}%
	\left(\sum_{k \in \{1,2,\ldots,K\}}\left[\phi^k - \Phi((A^*_j)_{j%
	\in \{1, 2, \ldots, N\}}, (x^k_i)_\text{$i$ component},%
	T^k, P^k)\right]^2\right)
\end{equation}

For data obtained exclusively from binary systems, one may write:

\begin{equation}
	\forall k \in \{1,2,\ldots,K\} \;\; \exists i \neq w%
	\text{ component } : \forall \;\; i' \neq i, i' \neq w\;\; x^k_{i'} = 0
\end{equation}

When for all sets of data obtained exclusively from binary systems the equation
\ref{eq:no_bin_fit} is valid, one can say that the model presents an adjustable
parameter of component-component interaction. Otherwise, one can say that the model
has no parameters accounting for the interaction between two components.

\begin{equation}
	\label{eq:no_bin_fit}
	\exists j' \in \{1,2,\ldots,N\} : \forall \alpha \in \mathbb{R}\;\;%
	(A_j^*)_{j \in \{1,2,\ldots,N\}, A_{j'}^* = \alpha}%
	\text{ is a solution of \ref{eq:min_quad}}
\end{equation}

One must differentiate these two groups, for one of the aims of the project
is the assessment of the applicability of parameter estimates, obtained through
fitting curves through binary data, to $n$-ary data; this is important, primarily
because experimental data for binary systems are more abundant and easier to
obtain, but also due to the high computational cost in multivariate nonlinear
fitting.

An example of a model in the second category is the one proposed by Norrish; also,
one could trivially place models such as Zdanovskii's procedure and Caurie's
relation in the category. For the first one, more complex, the virial model defined
in the equation \ref{eq:ad_pessoa} or the UNIQUAC model can be considered as
examples.

\section{Limitations of the models and problems found}

One of the main problems found is the simple non-applicability of some models
to a few sets of data; as an example, one may check the intrinsical limitations
of the Norrish's model.

Norrish's model, as seen in the equations \ref{eq_norrish} and
\ref{eq_norrish_multi}, can be algebraically manipulated to yield equation
\ref{eq_phi_norrish} for the calculation of the osmotic coefficient $\phi$.

\begin{equation}
	\label{eq_phi_norrish}
	\phi = \cfrac{\ln(a_w)}{\ln(x_w)} = \cfrac{\ln(x_w) + \left(\sum_{ i\neq w}
	\sqrt{K_i}x_1\right)^2}{\ln(x_w)}
\end{equation}

As $x_w \le 1 \implies \ln(x_w) \le 0$, one can observe that the model can't
predict a value for $\phi$ greater than 1; however, this is not verified
in all real solutions, that can show several behaviors, ranging from
monotonically decreasing functions (as the model requires,\footnote{%
	Since $\lim_{x_w \to 1}\phi_\text{Norrish} = 1$.
} to monotonically increasing functions, the opposite of the desired
behavior, which can be seen in figure \ref{fig_atv_gamma_gluc}
and, more clearly, in figure \ref{fig_manose_phi}.

Another example is Caurie's equation; since $\lim_{x_w \to 0}m_i = +\infty$,
one can observe the existence of crtical dilutions, beyond which the
model simply fails to return a physically meaningful result, because
$a_w \in [0,1]$ and, as shown in the equations \ref{eq_caurie_fail},
there are values for the molar fractions of each component in the mixture
that, according to the model, must force the system to exhibit negative
values of $a_{w,\text{Caurie}}$. Besides, as the molar fraction of water
in a mixture decreases, $\phi$ must, in the limit, be as close as
desired to 0.

\begin{equation}
	\label{eq_caurie_fail}
	\begin{split}
		a_{w,\text{Caurie}} = \prod_{i \neq w}a_{w,i} - \left[\cfrac{n}%
			{55.5^2}\sum_{\substack{i \neq j \\ i,j \neq w}}%
			m_im_j + \cfrac{(n+1)}{55.5^3}\sum_{%
			\substack{i\neq j,k \\ j \neq k \\  i,j,k \neq w}}
			m_im_jm_k\right] <\\
		1 - \cfrac{n}{55.5^3}\left[55.5 \times%
			\sum_{\substack{i \neq j \\ i,j \neq w}}%
			m_im_j + \cfrac{n+1}{n}\sum_{%
			\substack{i\neq j,k \\ j \neq k \\  i,j,k \neq w}}
			m_im_jm_k\right]\implies\\
		\lim_{\substack{m_k \to +\infty \\%
				\forall n \neq k, \text{ $m_n$ constant}}}%
			a_{w,\text{Caurie}} = -\infty \implies%
		\lim_{x_w \to 0} a_{w,\text{Caurie}} = -\infty
	\end{split}
\end{equation}

This diversity of behaviors can be useful. For instance, for the virial model,
the measure chosen for the concentration was molar concentration, since one can
show that the data rarely follow the curves to which they are restricted when
one utilizes instead the molar fraction, as one would naïvely do. This can be
seen in the equations \ref{eq_phi_virial_mono_frac}
\footnote{%
	This can be affirmed since, $\forall w > 0$:
	\begin{equation*}
		\begin{split}
			e^w = 1 + w + \cfrac{w^2}{2!} + \ldots \ge w +1
			\stackrel{(z = w+1)}{\implies} e^z \ge ze \implies
				z \ge \ln(z) + 1
			\stackrel{\left(y=\cfrac{1}{z}\right)}{\implies}
				\cfrac{1}{y} \ge
				\ln\left(\cfrac{1}{y}\right) + 1 \implies\\
			1-y \ge y\ln\left(\cfrac{1}{y}\right)
			\stackrel{(x=1-y)}{\implies} x >
				(1-x)\ln\left(\cfrac{1}{1-x}\right)\implies
			\cfrac{x}{1-x} + \ln(1-x) > 0 \implies\\
				\cfrac{d}{dx}\left(\cfrac{x}{\ln(1-x)}\right) > 0
		\end{split}
	\end{equation*}
}
for binary systems, with $x$ as the molar fraction of the solute: the model
(with $\theta_i = x_i$) is obligatorily a monotonic function.

\begin{equation}
	\label{eq_phi_virial_mono_frac}
	\begin{split}
		\ln(a_w) = -x + 2bx \implies \phi =
			\cfrac{-x + 2bx}{\ln(1-x)}\implies\\
		\phi = (2b-1)\cfrac{x}{\ln(1-x)} \implies
			\forall (x_1,x_2) \in (0,1)\times(0,1)\;\;
			\cfrac{d\phi}{dx}\Big|_{x_1} \times
			\cfrac{d\phi}{dx}\Big|_{x_2} \ge 0
	\end{split}
\end{equation}

Another challenge (not intrinsical to the models, but specific to the data)
is the distribution of the data in small sets, compared to the amount of
data points required to fit a large model (as the complete UNIQUAC model,
for instance). Indeed, for a mixture of $n$ components, the model requires
$2n+n^2 \in \mathcal{O}(n^2)$ parameters (accounting for the interaction
energies $U_{ij}$ and the parameters $r_i$ and $q_i$. To solve this problem,
a few simplifications, from the literature, were adopted, as exhibited in
the equations \ref{eq_uniquac_simpl}.

\begin{equation}
	\label{eq_uniquac_simpl}
	\begin{cases}
		u_{ij} = \sqrt{u_{ii}u_{jj}}\\
		r_i = q_i
	\end{cases}\forall\ i, j\text{ component}
\end{equation}

Additionally, one must consider that a non-negligible amount of the datasets
exhibits colinearity-related problems, which, even if not a problem for correlation,
makes the values of the parameters adjusted to the data set less reliable. Finally,
for very diluted solutions, one can observe that the osmotic coefficient values
have big uncertainties, due to the relatively high uncertainties in the measures
of the logarithm of numbers close to 1.

\begin{figure}[h]
	\centering
	\begin{tikzpicture}
		\newsavebox{\minigrafmann}
		\savebox{\minigrafmann}{%
		\scalebox{0.5}{%
		\begin{tikzpicture}
			\begin{axis}[
			xmin=0.0,xmax=0.07,
			ymin=0.93,ymax=1.0,
			width=7cm,
			height=7cm,
			xlabel = {$x$},
			ylabel = {$a_w$},
			xlabel near ticks,
			ylabel near ticks,
			xticklabel style={
				/pgf/number format/fixed,
				/pgf/number format/precision=3,
				/pgf/number format/fixed zerofill
			},
			scaled x ticks=false,
			]
			\addplot+[
				color=pbrightred,
				mark=o,
				only marks,
				thick,
			]
			table[x={mannose},y={aw_exp},col sep=comma]
				{ebrahimi_mannose_norrish.csv};
			\addplot+[
				color=pverybrightblue,
				no marks,
				thick,
			]
			table[x={mannose},y={aw_calc},col sep=comma]
				{ebrahimi_mannose_norrish.csv};
			\addplot+[
				color=pblue,
				no marks,
				thick,
			]
			table[x={mannose},y={aw_calc},col sep=comma]
				{ebrahimi_mannose_virial.csv};
			\addplot+[
				color=pverydarkblue,
				no marks,
				thick,
			]
			table[x={mannose},y={aw_calc},col sep=comma]
				{ebrahimi_mannose_uniquac.csv};
		\end{axis}
	\end{tikzpicture}
	}}
	\begin{axis}[
			xmin = 0.0, xmax = 0.072, xlabel = {$x_\text{mannose}$},
			ymin = 0.975, ymax = 1.029, ylabel = {$\phi$},
			legend pos = south east,
			xlabel near ticks,
			ylabel near ticks,
			xticklabel style={
				/pgf/number format/fixed,
				/pgf/number format/precision=3,
				/pgf/number format/fixed zerofill
			},
			scaled x ticks=false,
			xtick = {0.015,0.030,0.045,0.060},
		]
		\addplot+[
			smooth,
			color=pverybrightblue,
			no marks,
			thick,
		]
		table[x={mannose},y={phi_calc},col sep=comma]
			{ebrahimi_mannose_norrish.csv};
		\addlegendentry{Norrish};
		\addplot+[
			smooth,
			color=pblue,
			no marks,
			thick,
		]
		table[x={mannose},y={phi_calc},col sep=comma]
			{ebrahimi_mannose_virial.csv};
		\addlegendentry{virial};
		\addplot+[
			smooth,
			color=pverydarkblue,
			no marks,
			thick,
		]
		table[x={mannose},y={phi_calc},col sep=comma]
			{ebrahimi_mannose_uniquac.csv};
		\addlegendentry{UNIQUAC};
		\addplot+[
			color=pbrightred,
			mark=o,
			very thick,
			only marks,
		]
		table[x={mannose},y={phi_exp},col sep=comma]
			{ebrahimi_mannose_uniquac.csv};
		\addlegendentry{experimental};
		\draw (axis cs:0.018,1.015) node{\usebox{\minigrafmann}};
		\end{axis}
	\end{tikzpicture}
	\caption{Osmotic coefficient $\phi$ e adjusted models for a mannose solution}
	\label{fig_manose_phi}
\end{figure}


\chapter{Experimental data sets analysed}

\label{sec_selec_data}

Due to external circunstances, laboratorial experiments are neither safe nor
practical; fortunately, they are also not needed: the previous research on the
area is rich enough to supply a good selection of experimental data for binary
and/or multicomponent systems. The data used in the project and its sources are
listed below.

\begin{itemize}
	\item For binary systems:
		\begin{itemize}
			\item Abderafi \& Bounahmidi (1994) \cite{abderafi1994};
			\item Bhandari \& Bareyre (2003) \cite{bhandari2003},
				direct measurements of $a_w$ for solutions of water
				and glucose at 25\textcelsius, with values between
				$x_w = 0.808$ and $x_w = 0.917$;
			\item Bonner (1982) \cite{bonner1982}, osmotic coefficient
				and activity measurements for lisine
				and arginine at a temperature of 298.15K.
			\item Chen (1987) \cite{chen1987}, water activity
				measurements of a few solutions of carbohydrates
				obtained through equilibrium relative humidity, at
				freezing temperatures.
			\item Cooke, Jónsdóttir \& Westh (2002) \cite{cooke2002a},
				vapor pressure measurements at temperatures of
				24.91\textcelsius\ and 44.84\textcelsius, for
				sucrose and other carbohydrates, with molar fractions
				between 0 and 0.24.
%			\item Dunning, Evans \& Taylor (1951) \cite{dunning1951},
%				medidas de pressão de vapor da água para soluções de
%				sacarose com temperaturas entre 60 e
%				95\textcelsius\ e frações molares $x_w$ entre 0.79
%				e 0.96.\footnote{%
%					Não puderam ser utilizados, tendo em vista
%					que foram, todos, obtidos a temperaturas
%					distintas, \textit{i.e.} não é coerente
%					ajustar esse conjunto de dados com uma
%					isoterma.
%				}
			\item Ebrahimi \& Sadeghi (2016) \cite{ebrahimi2016},
				osmotic coefficient data for several carbohydrates,
				including monosaccharides, dissacharides,
				trissacharides and polyols, at 308.15K.
			\item Ellerton, Reinfelds, Mulcahy \& Dunlop (1964)
				\cite{ellerton1964} osmotic pressure measurements
				obtained through the isopiestic method for five
				aminoacids at the temperature of 25\textcelsius.
			\item Himanshu, Priyanka \& Anakshi (2005)
				\cite{himanshu2005}, freezing point depression
				measurements were converted to osmotic coefficient
				data for diluted binary solutions of nine aminoacids
				(glycine, L-serine, L-proline, DL-valine,
				DL-alanine, L-threonine, hidroxy-L-proline,
				L-isoleucine and DL-methionine).
			\item Kiyosawa (1992) \cite{kiyosawa1992}, freezing
				point depression measurements of binary polyol
				solutions
			\item Kuramochi, Noritomi, Hoshino \& Nagahama (1997)
				\cite{kuramochi1997}, activity coefficient
				measurements for four aminoacids (glycine,
				L-alanine, L-serine and L-valine), at 298.15K
			\item Maximo, Meirelles \& Batista (2010) \cite{maximo2010},
				activity coefficient measurements obtained through
				boiling point elevation for sucrose, fructose and
				glucose at different pressures.
			\item Ninni \& Meirelles (2001) \cite{ninni2001}, direct
				measurements of water activity for binary solutions
				of four aminoacids (glycine, DL-alanine, L-proline
				and L-arginine), at 25\textcelsius.
			\item Pinho (2008) \cite{pinho2008}, direct water activity
				measurements for solutions of three aminoacids
				(glycine, DL-alanine and L-serine) at
				25\textcelsius\ and $x_w$ values close to 1.
			\item Romero \& González (2006) \cite{romero2006}, osmotic
				coefficient measurements obtained through the
				isopiestic method, for glycine, DL-$\alpha$-alanine
				and DL-$\alpha$-aminobutyric acid, at temperatures
				ranging from 288.15K and 303.15K.
			\item Tsurko, Neueder \& Kunz (2007) \cite{tsurko2007},
				osmotic coefficient data for three aminoacid
				solutions (glycine, glutamic acid and histidine),
				obtained through vapor-liquid equilibria, at
				temperatures of 298.15K and 310.15K.
			\item Velezmoro, Oliveira, Cabral \& Meirelles (2000)
				\cite{velezmoro2000}.
		\end{itemize}
	\item For $n$-ary systems:
		\begin{itemize}
			\item Abderafi \& Bounahmidi (1994) \cite{abderafi1994},
				boiling point elevation data for binary, ternary
				and quaternary solutions of sucrose, glucose and
				fructose, at low dilutions and atmospheric pressure.
			\item Norrish (1966) \cite{norrish1966}, equilibrium
				relative humidity measurements for ternary solutions
				of water, sucrose and sorbitol, glycerol or dextrose,
				at 25\textcelsius.
			\item Robinson \& Stokes (1961) \cite{stokes1961},
				osmotic equilibria data for ternary solutions
				of sucrose and mannitol at 25\textcelsius.
			\item Stokes \& Robinson (1966) \cite{stokes1966},
				osmotic equilibria data of ternary solutions with
				sucrose as a solute.
			\item Velezmoro, Oliveira, Cabral \& Meirelles (2000)
				\cite{velezmoro2000}, direct $a_w$ measurements
				at temperaturas of 25, 30 and 35\textcelsius\ for
				glucose, fructose, sucrose and maltose binary and
				quaternary solutions at high dilutions.
		\end{itemize}
\end{itemize}

\begin{tabularx}{\textwidth}{ X  c  X }
	\caption{Data points distribution by source (binary systems)}
	\label{tab_dados_pontos}\\
	\toprule
	Reference & \textnumero\ of points & Method utilized\\
	\midrule
	\endfirsthead
	\toprule
	Reference & \textnumero\ of points & Method utilized\\\hline
	\midrule
	\endhead
	\midrule
	\multicolumn{3}{r}{\footnotesize(Continue in the following page)}
	\endfoot
	\endlastfoot
	Abderafi \& Bounahmidi (1994) & 36 & Vapor-liquid equilibria\\
	Bhandari \& Bareyre (2003) & 10 & Direct $a_w$ measurements\\
	Bonner (1982) & 34 & Isopiestic method\\
	Chen (1987) & 14 & Vapor pressure measurements\\
	Cooke, Jónsdóttir \& Westh (2002) & 40 & Vapor pressure measurements\\
%	Dunning, Evans \& Taylor (1951) & 102 & Medições de pressão de vapor\\
	Ellerton, Reinfelds, Mulcahy \& Dunlop (1964) & 91 & Isopiestic method\\
	Ebrahimi \& Sadeghi (2016) & 246 & Vapor pressure osmometry\\
	Himanshu, Priyanka \& Anakshi (2005) & 45 &
		Freezing point depression\\
	Kiyosawa (1992) & 49 & Freezing point depression\\
	Kuramochi, Noritomi, Hoshino \& Nagahama (1997) & 44 &
		Vapor pressure measurements\\
	Maximo, Meirelles \& Batista (2010) & 168 & Boiling point elevation\\
	Ninni \& Meirelles (2001) & 31 & Vapor pressure measurements\\
	Pinho (2008) & 75 & Direct $a_w$ measurements\\
	Romero \& González & 144 & Isopiestic method\\
	Robinson \& Stokes (1961) & 43 & Isopiestic method\\
	Tsurko, Neueder \& Kunz (2007) & 57 & Vapor pressure measurements\\
	Velezmoro, Oliveira, Cabral \& Meirelles (2000) &
		145 & Vapor pressure measurements\\\hline
	Total & \multicolumn{2}{c}{1251}\\\hline
\end{tabularx}

\begin{tabularx}{\textwidth}{ X  c  X }
	\caption{Data points distribution by source ($n$-ary systems)}
	\label{tab_dados_multi_pontos}\\
	\toprule
	Reference & \textnumero\ of points & Method utilized\\
	\midrule
	\endfirsthead
	\toprule
	Reference & \textnumero\ of points & Method utilized\\\hline
	\midrule
	\endhead
	\midrule
	\multicolumn{3}{r}{\footnotesize(Continue in the following page)}
	\endfoot
	\endlastfoot
	Abderafi \& Bounahmidi (1994) & 174 & Boiling point elevation\\
	Norrish (1966) & 26 & Vapor pressure measurements\\
	Robinson \& Stokes (1961) & 74 & Isopiestic method\\
	Stokes \& Robinson (1966) & 24 & Isopiestic method\\
	Velezmoro, Oliveira, Cabral \& Meirelles (2000) & 135 &
		Vapor pressure measurements\\\hline
	Total & \multicolumn{2}{c}{358}\\\hline
\end{tabularx}

