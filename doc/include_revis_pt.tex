\part{Revisão Bibliográfica}

\chapter{Modelagem da atividade da água}

\section{Modelos avaliados}

Foram analisados diversos modelos, desde mais simples como a Lei de Raoult ou
equação de Caurie a mais complexos como o método iterativo originário da equação
de Zdanovskii, para posterior comparação com uma equação do tipo virial (como
\ref{eq:ad_pessoa}, a adaptada de \cite{pessoa2008}, na qual $b_i$ e
$c_{ij}$ são parâmetros ajustáveis e $\theta_i$ é uma medida de concentração,
com $i,j$ componentes da mistura).

\begin{equation}
	\label{eq:ad_pessoa}
	\ln(a_w) = \sum_{i \neq w}\theta_i\Bigg(1 +%
	2b_i+3\sum_{j \neq w}c_{ij}\theta_j\Bigg)
\end{equation}

Para obter os valores de $a_w$ para soluções com apenas um soluto, temos
diversos modelos que podem ser aplicados. Desses, serão analisados e comparados
os seguintes:

\begin{itemize}
	\item Lei de Raoult: Modelo mais simples, para soluções ideais. Assume
		atividade da água $a_w = x_w$, sua fração molar;
	\item Equação de Norrish \cite{norrish1966}: correção da lei de Raoult.
		Sendo $x_w$ a fração molar da água e $x_s$ a do soluto e
		tomando um parâmetro ajustável $K$, experimental, obtemos o
		valor de $a_w$ através da equação \ref{eq_norrish}.
		\begin{equation}
			\label{eq_norrish}
			a_w = x_w\exp\Big[\Big(\sqrt{K}x_s\Big)^2\Big]
		\end{equation}
	\item Modelo UNIQUAC (\textit{universal quasi-chemical}):
		\cite{abrams1975} Modelo mais refinado dentre os avaliados no
		projeto, demanda dois parâmetros para o ajuste para cada par de
		componentes da mistura (solvente e solutos), podendo ser tanto
		utilizado para sistemas binários quanto para sistemas $n$-ários.
		Embora claramente apresente custo computacional elevado para
		misturas com muitos componentes (já que o número de parâmetros
		é da ordem do número de componentes ao quadrado), os dados
		utilizados não são de misturas complexas o suficiente para que
		isso seja um problema.
\end{itemize}

Os modelos considerados para análise e comparação serão, para sistemas
multicomponente, os seguintes:

\begin{itemize}
	\item Lei de Raoult: atividade da água $a_w$ sendo igual a $x_w$,
		sua fração molar;
	\item Equação de Norrish: adotada pequena modificação para equação
		\ref{eq_norrish}, como visto na equação \ref{eq_norrish_multi};
		\begin{equation}
			\label{eq_norrish_multi}
			a_w = x_w\exp\Big[\Big(\sum_{i \neq w}%
			\sqrt{K_i}x_i\Big)^2\Big]
		\end{equation}
	\item Modelo de Caurie \cite{caurie1986}: atividade da água como uma
		correção do produto das atividades da água para uma solução
		com um único componente (equação \ref{eq_caurie}) com
		$m_i$ a molalidade do componente $i$ em solução, $a_{wi}$
		sendo a atividade da água em uma mistura com o componente
		$i$, apenas, tal que $m_i$ se mantenha e $n$ sendo o número de
		solutos em solução.
		\begin{equation}
			\label{eq_caurie}
			a_w = \prod_{i \neq w}a_{wi}-\left[\cfrac{n}%
			{55.5^2}\sum_{\substack{i \neq j \\ i,j \neq w}}%
			m_im_j + \cfrac{(n+1)}{55.5^3}\sum_{%
			\substack{i\neq j,k \\ j \neq k \\  i,j,k \neq w}}%
			m_im_jm_k\right]
		\end{equation}
		Para a obtenção das atividades da água de substâncias
		binárias, foi utilizada a Lei de Raoult.
	\item Relação de Zdanovskii \cite{chen1973,sangster1973}:
		para uma mistura ternária composta por água e mais dois
		componentes, digamos, 1 e 2, adotamos duas soluções binárias,
		cada uma com um componente, em equilíbrio osmótico com uma
		terceira, ternária. Sendo $m_{01}$ e $m_{02}$ as
		molalidades do componente 1, na solução na qual é o único
		soluto, e do componente 2, na solução na qual é o único soluto, e
		sendo $m_1$ e $m_2$ as molalidades de 1 e 2 na solução ternária,
		temos que $m_1m_{01}^{-1} + m_2m_{0_2}^{-1} = 1$, sendo esta a
		relação de Zdanovskii. Disso se obtém um procedimento iterativo,
		que possibilita obter, tendo as isotermas de cada componente, a
		atividade da água de uma mistura ternária.
	\item Modelo UNIQUAC (\textit{universal quasi-chemical}), citado
		anteriormente.
\end{itemize}

\section{Aplicabilidade dos modelos às amostras}

Como não temos interesse apenas em ajustar curvas aos dados para sistemas
$n$-ários, mas também queremos avaliar a possibilidade de prever valores de
$a_w$ para esses sistemas a partir de dados obtidos do ajuste de sistemas
binários, podemos dividir os modelos estudados em dois grandes grupos. Sendo um
modelo termodinâmico a $N$ parâmetros $(A_j)_{j \in \{1, 2, \ldots N\}}$ para o
coeficiente osmótico em um sistema de composição $(x_i)_\text{$i$ componente}$,
com temperatura $T$ e pressão $P$ dado pela equação \ref{eq:geral_bigphi},
podemos obter aproximações para os parâmetros $(A_j^*)_{j \in \{1, 2, \ldots, N\}}$
a partir do método dos mínimos quadrados, dado na expressão \ref{eq:min_quad},
para um conjunto de $K$ dados experimentais $\{(\phi^k, T^k, P^k,%
	(x_i)^k_\text{$i$ componente}), k \in \{1,2,\ldots,K\}\}$.

\begin{equation}
	\label{eq:geral_bigphi}
	\phi = \Phi((A_j)_{j \in \{1, 2, \ldots, N\}}, (x_i)_\text{$i$ %
		componente da mistura}, T, P)
\end{equation}

\begin{equation}
	\label{eq:min_quad}
	\minimize_{(A_j^*)_{j \in \{1,2,\ldots,N\}} \in \mathbb{R}^N}%
	\left(\sum_{k \in \{1,2,\ldots,K\}}\left[\phi^k - \Phi((A^*_j)_{j%
	\in \{1, 2, \ldots, N\}}, (x^k_i)_\text{$i$ componente},%
	T^k, P^k)\right]^2\right)
\end{equation}

Para dados obtidos exclusivamente de sistemas binários, temos que:

\begin{equation}
	\forall k \in \{1,2,\ldots,K\} \;\; \exists i \neq w%
	\text{ componente } : \forall \;\; i' \neq i, i' \neq w\;\; x^k_{i'} = 0
\end{equation}

Quando para todo conjunto de dados obtidos exclusivamente de sistemas binários
temos que é válida a expressão \ref{eq:no_bin_fit},
dizemos que o modelo apresenta parâmetro ajustável de interação entre os
solutos. Caso contrário, dizemos que o modelo não apresenta parâmetro
ajustável de interação entre os solutos.

\begin{equation}
	\label{eq:no_bin_fit}
	\exists j' \in \{1,2,\ldots,N\} : \forall \alpha \in \mathbb{R}\;\;%
	(A_j^*)_{j \in \{1,2,\ldots,N\}, A_{j'}^* = \alpha}%
	\text{ é solução de \ref{eq:min_quad}}
\end{equation}

Essa distinção se faz necessária a um dos objetivos do projeto, a saber, avaliar
a aplicabilidade de parâmetros obtidos a partir do ajuste de um modelo a dados
de sistemas binários à predição de dados de sistemas $n$-ários, aplicabilidade
essa que, caso verificada, traria grande vantagem ao modelo em questão, tendo em
vista a quantidade de dados disponível para um sistema binário qualquer
ser muito superior à quantidade de dados disponível para um sistema $n$-ário
qualquer. Além disso, também deve-se considerar que os experimentos de
laboratório necessários ao levantamento de dados representativos para sistemas
$n$-ários são muito mais custosos.

É exemplo de modelo da segunda categoria o proposto por Norrish; para a
primeira categoria, mais complexa, o modelo do tipo virial proposto na equação
\ref{eq:ad_pessoa} pode ser citado como exemplo. Além disso, claramente temos,
inclusos de forma trivial na segunda categoria, modelos sem parâmetro ajustável
algum, como a Lei de Raoult.

\section{Limitações dos modelos e principais problemas encontrados}

Um dos principais problemas encontrados é a simples não aplicabilidade de
alguns modelos em determinadas situações: para exemplificar essa situação,
analisaremos o modelo de Norrish e suas limitações intrínsecas.

O modelo de Norrish, como visto em \ref{eq_norrish} e \ref{eq_norrish_multi},
apresenta a equação \ref{eq_phi_norrish} para o cálculo do coeficiente
osmótico $\phi$:

\begin{equation}
	\label{eq_phi_norrish}
	\phi = \cfrac{\ln(a_w)}{\ln(x_w)} = \cfrac{\ln(x_w) + \left(\sum_{ i\neq w}
	\sqrt{K_i}x_1\right)^2}{\ln(x_w)}
\end{equation}

Como $x_w \le 1 \implies \ln(x_w) \le 0$, temos que o modelo prevê apenas
valores do coeficiente osmótico menores que 1. Isso definitivamente não se
verifica nos dados reais: temos diversos comportamentos possíveis, desde
dados que se assemelham a funções monótonas decrescentes, como exigido pelo
modelo,\footnote{%
	Já que $\lim_{x_w \to 1}\phi_\text{Norrish} = 1$.
} até dados que se assemelham a funções monótonas crescentes, o oposto do
desejado. Isso pode ser visto na própria figura \ref{fig_atv_gamma_gluc}, e,
mais claramente, na figura \ref{fig_manose_phi}.

Outro exemplo é a equação de Caurie: como $\lim_{x_w \to 0}m_i = +\infty$,
temos que existem concentrações a partir das quais o modelo simplesmente não
retorna resultados físicos, já que $a_w \in [0,1]$ e, como exibido na equação
\ref{eq_caurie_fail}, existem certos valores para as frações molares de cada
componente na mistura que retornam valores negativos de $a_{w,\text{Caurie}}$.
Além disso, conforme a fração molar de água na mistura decresce, $\phi$ deve
se aproximar de 0, no limite.

\begin{equation}
	\label{eq_caurie_fail}
	\begin{split}
		a_{w,\text{Caurie}} = \prod_{i \neq w}a_{w,i} - \left[\cfrac{n}%
			{55.5^2}\sum_{\substack{i \neq j \\ i,j \neq w}}%
			m_im_j + \cfrac{(n+1)}{55.5^3}\sum_{%
			\substack{i\neq j,k \\ j \neq k \\  i,j,k \neq w}}
			m_im_jm_k\right] <\\
		1 - \cfrac{n}{55.5^3}\left[55.5 \times%
			\sum_{\substack{i \neq j \\ i,j \neq w}}%
			m_im_j + \cfrac{n+1}{n}\sum_{%
			\substack{i\neq j,k \\ j \neq k \\  i,j,k \neq w}}
			m_im_jm_k\right]\implies\\
		\lim_{\substack{m_k \to +\infty \\%
				\forall n \neq k, \text{ $m_n$ constante}}}%
			a_{w,\text{Caurie}} = -\infty \implies%
		\lim_{x_w \to 0} a_{w,\text{Caurie}} = -\infty
	\end{split}
\end{equation}

Essa diversidade de comportamentos pode vir a ser útil. Por exemplo, para o
modelo virial, a medida de concentração escolhida foi a concentração molar, já
que é possível demonstrar que os dados dificilmente seguem a forma à qual são
restritos ao utilizarmos a fração molar (mais intuitiva) como medida. Isso pode
ser visto na equação \ref{eq_phi_virial_mono_frac}
\footnote{%
	Podemos afirmar isso já que temos, $\forall w > 0$:
	\begin{equation*}
		\begin{split}
			e^w = 1 + w + \cfrac{w^2}{2!} + \ldots \ge w +1
			\stackrel{(z = w+1)}{\implies} e^z \ge ze \implies
				z \ge \ln(z) + 1
			\stackrel{\left(y=\cfrac{1}{z}\right)}{\implies}
				\cfrac{1}{y} \ge
				\ln\left(\cfrac{1}{y}\right) + 1 \implies\\
			1-y \ge y\ln\left(\cfrac{1}{y}\right)
			\stackrel{(x=1-y)}{\implies} x >
				(1-x)\ln\left(\cfrac{1}{1-x}\right)\implies
			\cfrac{x}{1-x} + \ln(1-x) > 0 \implies\\
				\cfrac{d}{dx}\left(\cfrac{x}{\ln(1-x)}\right) > 0
		\end{split}
	\end{equation*}
}
para sistemas binários, sendo $x$ a fração molar do soluto: o modelo (com
$\theta_i = x_i$) implica em uma função monótona.

\begin{equation}
	\label{eq_phi_virial_mono_frac}
	\begin{split}
		\ln(a_w) = -x + 2bx \implies \phi =
			\cfrac{-x + 2bx}{\ln(1-x)}\implies\\
		\phi = (2b-1)\cfrac{x}{\ln(1-x)} \implies
			\forall (x_1,x_2) \in (0,1)\times(0,1)\;\;
			\cfrac{d\phi}{dx}\Big|_{x_1} \times
			\cfrac{d\phi}{dx}\Big|_{x_2} \ge 0
	\end{split}
\end{equation}

Outro problema (não intrínseco aos modelos, mas específico aos dados) é a
existência de pequenos conjuntos de dados, comparados ao que seria
necessário para ajustar o modelo UNIQUAC completo, por exemplo. De fato,
para uma mistura composta por $n$ substâncias, o modelo exige
$2n+n^2 \in \mathcal{O}(n^2)$ parâmetros (para contabilizar as energias de
interação $U_{ij}$ e os parâmetros $r_i$ e $q_i$). Dessa forma, algumas
simplificações (equações exibidas em \ref{eq_uniquac_simpl}), presentes
na literatura, foram adotadas.

\begin{equation}
	\label{eq_uniquac_simpl}
	\begin{cases}
		u_{ij} = \sqrt{u_{ii}u_{jj}}\\
		r_i = q_i
	\end{cases}\forall\ i, j\text{ componente}
\end{equation}

Devemos lembrar, além disso, que boa parte dos comjuntos de dados para sistemas
multicomponente apresenta problemas com colinearidade; isso não prejudica a
predição de novos dados, mas torna pouco confiáveis os valores obtidos para os
parâmetros. Por fim, existem problemas quanto a precisão dos dados de coeficiente
osmótico, devido às grandes incertezas nas medidas de logaritmos de números
próximos de 1.


\begin{figure}[h]
	\centering
	\begin{tikzpicture}
		\newsavebox{\minigrafmann}
		\savebox{\minigrafmann}{%
		\scalebox{0.5}{%
		\begin{tikzpicture}
			\begin{axis}[
			xmin=0.0,xmax=0.07,
			ymin=0.93,ymax=1.0,
			width=7cm,
			height=7cm,
			xlabel = {$x$},
			ylabel = {$a_w$},
			xlabel near ticks,
			ylabel near ticks,
			xticklabel style={
				/pgf/number format/fixed,
				/pgf/number format/precision=3,
				/pgf/number format/fixed zerofill
			},
			scaled x ticks=false,
			]
			\addplot+[
				color=pbrightred,
				mark=o,
				only marks,
				thick,
			]
			table[x={mannose},y={aw_exp},col sep=comma]
				{ebrahimi_mannose_norrish.csv};
			\addplot+[
				color=pverybrightblue,
				no marks,
				thick,
			]
			table[x={mannose},y={aw_calc},col sep=comma]
				{ebrahimi_mannose_norrish.csv};
			\addplot+[
				color=pblue,
				no marks,
				thick,
			]
			table[x={mannose},y={aw_calc},col sep=comma]
				{ebrahimi_mannose_virial.csv};
			\addplot+[
				color=pverydarkblue,
				no marks,
				thick,
			]
			table[x={mannose},y={aw_calc},col sep=comma]
				{ebrahimi_mannose_uniquac.csv};
		\end{axis}
	\end{tikzpicture}
	}}
	\begin{axis}[
			xmin = 0.0, xmax = 0.072, xlabel = {$x_\text{mannose}$},
			ymin = 0.975, ymax = 1.029, ylabel = {$\phi$},
			legend pos = south east,
			xlabel near ticks,
			ylabel near ticks,
			xticklabel style={
				/pgf/number format/fixed,
				/pgf/number format/precision=3,
				/pgf/number format/fixed zerofill
			},
			scaled x ticks=false,
			xtick = {0.015,0.030,0.045,0.060},
		]
		\addplot+[
			smooth,
			color=pverybrightblue,
			no marks,
			thick,
		]
		table[x={mannose},y={phi_calc},col sep=comma]
			{ebrahimi_mannose_norrish.csv};
		\addlegendentry{Norrish};
		\addplot+[
			smooth,
			color=pblue,
			no marks,
			thick,
		]
		table[x={mannose},y={phi_calc},col sep=comma]
			{ebrahimi_mannose_virial.csv};
		\addlegendentry{virial};
		\addplot+[
			smooth,
			color=pverydarkblue,
			no marks,
			thick,
		]
		table[x={mannose},y={phi_calc},col sep=comma]
			{ebrahimi_mannose_uniquac.csv};
		\addlegendentry{UNIQUAC};
		\addplot+[
			color=pbrightred,
			mark=o,
			very thick,
			only marks,
		]
		table[x={mannose},y={phi_exp},col sep=comma]
			{ebrahimi_mannose_uniquac.csv};
		\addlegendentry{experimental};
		\draw (axis cs:0.018,1.015) node{\usebox{\minigrafmann}};
		\end{axis}
	\end{tikzpicture}
	\caption{Coeficiente osmótico $\phi$ e ajuste de modelos
		para solução de manose}
	\label{fig_manose_phi}
\end{figure}


\chapter{Seleção de dados experimentais}

\label{sec_selec_data}

Devido às presentes circunstâncias, não é prático ou seguro fazer experimentos
laboratoriais para levantamento de dados. Felizmente, também não é necessário:
a literatura científica sobre o assunto é extensa o suficiente para possibilitar
a obtenção uma boa seleção de dados experimentais previamente publicados, para
sistemas multicomponente e binários, que se encontram resumidos nas listas e
tabelas que seguem.

\begin{itemize}
	\item Para sistemas binários:
		\begin{itemize}
			\item Abderafi \& Bounahmidi (1994) \cite{abderafi1994};
			\item Bhandari \& Bareyre (2003) \cite{bhandari2003},
				medidas diretas de $a_w$ para soluções de água
				e glicose a 25\textcelsius, com valores entre
				$x_w = 0.808$ e $x_w = 0.917$;
			\item Bonner (1982) \cite{bonner1982}, medidas de
				coeficiente osmótico e de atividade para lisina
				e arginina a temperatura de 298.15K.
			\item Chen (1987) \cite{chen1987}, medidas para atividade
				de água de algumas soluções de carboidratos obtidas
				a partir de dados de umidade relativa de equilíbrio,
				à temperatura de congelamento.
			\item Cooke, Jónsdóttir \& Westh (2002) \cite{cooke2002a},
				medidas de pressão de vapor em temperaturas de
				24.91\textcelsius, para a sacarose, e
				44.84\textcelsius, para frutose, e outros
				carboidratos, com frações molares entre 0 e 0.24 a
				depender da substância.
%			\item Dunning, Evans \& Taylor (1951) \cite{dunning1951},
%				medidas de pressão de vapor da água para soluções de
%				sacarose com temperaturas entre 60 e
%				95\textcelsius\ e frações molares $x_w$ entre 0.79
%				e 0.96.\footnote{%
%					Não puderam ser utilizados, tendo em vista
%					que foram, todos, obtidos a temperaturas
%					distintas, \textit{i.e.} não é coerente
%					ajustar esse conjunto de dados com uma
%					isoterma.
%				}
			\item Ebrahimi \& Sadeghi (2016) \cite{ebrahimi2016},
				medidas de coeficiente osmótico para diversos
				carboidratos entre monosacarídeos (pentoses e
				hexoses sortidas), dissacarídeos, trissacarídeos
				e polióis, à temperatura fixa de 308.15K.
			\item Ellerton, Reinfelds, Mulcahy \& Dunlop (1964)
				\cite{ellerton1964} medidas de pressão osmótica
				obtidas através do método isopiéstico para cinco
				aminoácidos à temperatura de 25\textcelsius.
			\item Himanshu, Priyanka \& Anakshi (2005)
				\cite{himanshu2005}, medidas de depressão de
				ponto de fusão foram convertidas em coeficientes
				osmóticos para soluções de nove aminoácidos
				(glicina, L-serina, L-prolina, DL-valina,
				DL-alanina, L-treonina, hidróxi-L-prolina,
				L-isoleucina e DL-metionina) para soluções binárias
				de baixa molalidade ($m < 1$mol/kg).
			\item Kiyosawa (1992) \cite{kiyosawa1992}, medidas de
				depressão de ponto de congelamento de soluções
				binárias de polióis.
			\item Kuramochi, Noritomi, Hoshino \& Nagahama (1997)
				\cite{kuramochi1997}, medidas de coeficiente de
				atividade para quatro aminoácidos (glicina,
				L-alanina, L-serina e L-valina) obtidos através
				de dados de pressão de vapor, para temperatura
				de 298.15K.
			\item Maximo, Meirelles \& Batista (2010) \cite{maximo2010},
				medidas de coeficiente de atividade obtidas através
				de elevação do ponto de ebulição para sacarose,
				frutose e glicose a diferentes pressões.
			\item Ninni \& Meirelles (2001) \cite{ninni2001}, medidas
				diretas de atividade da água para soluções binárias
				de quatro aminoácidos (glicina, DL-alanina, L-prolina
				e L-arginina) a uma temperatura de 25\textcelsius,
				com concentrações relativamente baixas
				($x_w > 0.85$), para sistemas com ou sem tampões
				ácidos ou básicos.
			\item Pinho (2008) \cite{pinho2008}, medidas de atividade da
				água obtidas de forma direta para três aminoácidos
				(glicina, DL-alanina e L-serina), a uma temperatura
				de 25\textcelsius\ e frações molares de água
				próximas de 1.
			\item Romero \& González (2006) \cite{romero2006}, medidas
				de coeficiente osmótico obtidas através do método
				isopiéstico, para glicina, DL-$\alpha$-alanina e
				ácido DL-$\alpha$-aminobutírico, para temperaturas
				entre 288.15 e 303.15K.
			\item Tsurko, Neueder \& Kunz (2007) \cite{tsurko2007},
				medidas de coeficiente osmótico de três aminoácidos
				(glicina, ácido glutâmico e histidina), obtidas
				através de dados de equilíbrio líquido-vapor, a
				temperaturas de 298.15 e 310.15K.
			\item Velezmoro, Oliveira, Cabral \& Meirelles (2000)
				\cite{velezmoro2000}.
		\end{itemize}
	\item Para sistemas $n$-ários:
		\begin{itemize}
			\item Abderafi \& Bounahmidi (1994) \cite{abderafi1994},
				medidas de elevação do ponto de ebulição
				para sistemas binários glicose-água,
				frutose-água, sacarose-água,
				glicose-frutose-água, glicose-sacarose-água,
				frutose-sacarose-água e
				glicose-frutose-sacarose-água, para ampla
				faixa de $x_w$ e pressão atmosférica.
			\item Norrish (1966) \cite{norrish1966}, medidas de
				humidade relativa de equilíbrio para soluções
				ternárias compostas por água e sacarose, com a
				adição de sorbitol, glicerol ou dextrose, a
				temperatura de 25\textcelsius.
			\item Robinson \& Stokes (1961) \cite{stokes1961},
				dados de equilíbrio osmótico para a solução
				ternária água-sacarose-manitol à temperatura
				de 25\textcelsius.
			\item Stokes \& Robinson (1966) \cite{stokes1966},
				dados de equilíbrio osmótico de soluções
				ternárias de duas substâncias (sacarose e
				arabinose, sacarose e glicose, sacarose e
				sorbitol, entre outras).
			\item Velezmoro, Oliveira, Cabral \& Meirelles (2000)
				\cite{velezmoro2000}, medidas diretas de $a_w$ nas
				temperaturas 25, 30 e 35\textcelsius\ para soluções
				de água e glicose, água e frutose, água e maltose,
				água e sacarose, e uma solução de glicose, frutose
				e sacarose em base aquosa, com baixas concentrações
				($x_w > 0.9$).
		\end{itemize}
\end{itemize}

\begin{tabularx}{\textwidth}{ X  c  X }
	\caption{Dados por estudo para sistemas binários}
	\label{tab_dados_pontos}\\
	\toprule
	Referência & \textnumero\ de pontos & Obtenção\\
	\midrule
	\endfirsthead
	\toprule
	Referência & \textnumero\ de pontos & Obtenção\\
	\midrule
	\endhead
	\midrule
	\multicolumn{3}{r}{\footnotesize(Continua na página seguinte)}
	\endfoot
	\endlastfoot
	Abderafi \& Bounahmidi (1994) & 36 & Equilíbrio líquido vapor\\
	Bhandari \& Bareyre (2003) & 10 & Medições diretas de $a_w$\\
	Bonner (1982) & 34 & Método isopiéstico\\
	Chen (1987) & 14 & Medições de pressão de vapor\\
	Cooke, Jónsdóttir \& Westh (2002) & 40 & Medições de pressão de vapor\\
%	Dunning, Evans \& Taylor (1951) & 102 & Medições de pressão de vapor\\
	Ellerton, Reinfelds, Mulcahy \& Dunlop (1964) & 91 & Método isopiéstico\\
	Ebrahimi \& Sadeghi (2016) & 246 & Osmometria de pressão de vapor\\
	Himanshu, Priyanka \& Anakshi (2005) & 45 &
		Depressão do ponto de congelamento\\
	Kiyosawa (1992) & 49 & Depressão do ponto de congelamento\\
	Kuramochi, Noritomi, Hoshino \& Nagahama (1997) & 44 &
		Medições de pressão de vapor\\
	Maximo, Meirelles \& Batista (2010) & 168 & Elevação do ponto de ebulição\\
	Ninni \& Meirelles (2001) & 31 & Medições de pressão de vapor\\
	Pinho (2008) & 75 & Medições diretas de $a_w$\\
	Romero \& González & 144 & Método isopiéstico\\
	Robinson \& Stokes (1961) & 43 & Método isopiéstico\\
	Tsurko, Neueder \& Kunz (2007) & 57 & Medições de pressão de vapor\\
	Velezmoro, Oliveira, Cabral \& Meirelles (2000) &
		145 & Medições de pressão de vapor\\\hline
	Total & \multicolumn{2}{c}{1251}\\\hline
\end{tabularx}

\begin{tabularx}{\textwidth}{ X  c  X }
	\caption{Dados por estudo para sistemas $n$-ários}
	\label{tab_dados_multi_pontos}\\
	\toprule
	Referência & \textnumero\ de pontos & Obtenção\\
	\midrule
	\endfirsthead
	\toprule
	Referência & \textnumero\ de pontos & Obtenção\\\hline
	\midrule
	\endhead
	\midrule
	\multicolumn{3}{r}{\footnotesize(Continua na página seguinte)}
	\endfoot
	\endlastfoot
	Abderafi \& Bounahmidi (1994) & 174 & Elevação do ponto de ebulição\\
	Norrish (1966) & 26 & Medições de pressão de vapor\\
	Robinson \& Stokes (1961) & 74 & Método isopiéstico\\
	Stokes \& Robinson (1966) & 24 & Método isopiéstico\\
	Velezmoro, Oliveira, Cabral \& Meirelles (2000) & 135 &
		Medições de pressão de vapor\\\hline
	Total & \multicolumn{2}{c}{358}\\\hline
\end{tabularx}

